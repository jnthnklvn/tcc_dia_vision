\chapter{Considerações Finais}
\label{ch:conclusion}

Este trabalho buscou contextualizar e fundamentar a problemática identificada pela dificuldade
no acesso à informações sobre o autocuidado com o DM por pacientes com DV\@. Para isso, foram introduzidos
o DM e suas complicações relacionadas à DV, bem como, foram apresentados dados que indicam o crescimento no
número de casos de ambos.

No trabalho de mestrado que acarretou na parceira para desenvolvimento deste projeto, foi possível um aprofundamento
sobre as necessidades e requisitos deste público-alvo. Nele foram identificadas a importância do autocuidado
no tratamento do DM e as principais funcionalidades utilizadas como solução no mercado \cite{Sobral2021}.

Outra problemática introduzida foi que, mesmo com o aumento da informatização e popularização dos \emph{smartphones},
PDV ainda enfrentam sérias dificuldades devido à falta de acessibilidade à DV em aplicações móveis \cite{Shera2021285}.
Diante disso, foi realizado um processo de MSL visando identificar as principais soluções que estão sendo adotadas.

A partir da análise dos resultados do MSL, estabeleceram-se as técnicas de acessibilidade que seriam utilizadas
no sistema DiaVision. Com isso, foram desenvolvidos um aplicativo móvel multiplataforma e uma aplicação
\emph{backend} com um \emph{dashboard web} para gerenciamento dos dados, atendendo aos requisitos levantados
com a PO no planejamento do projeto.

A usabilidade de uma versão inicial do \emph{app} foi avaliada com 77 pacientes com DV, os resultados
desse estudo mostraram que 90,1\% dos participantes gostariam de ter o \emph{app}, 62,3\% avaliaram-o com nota
máxima, 53,2\% utilizariam-o sempre e 35,1\% apontaram que precisava melhorias quanto à acessibilidade \cite{Sobral2022}.

Contudo, 7 das 10 soluções de acessibilidade propostas para o \emph{app} no Capítulo \ref{ch:mapping} foram implementadas,
parte delas após realização do estudo com a versão inicial do aplicativo. Assim, é previsto que em trabalhos futuros
seja realizada outra validação com as novas funcionalidades e melhorias relativas à acessibilidade.