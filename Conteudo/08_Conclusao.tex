\chapter{Considerações Finais}
\label{ch:conclusion}

Com 463 milhões de casos de DM no ano de 2019, o número de casos vem crescendo rapidamente nos últimos 20 anos.
Paralelo a isso, pelo menos 2,2 bilhões de pessoas no mundo vivem com DV\@, com complicações do DM sendo
a maior causa de novos casos de cegueira.

No trabalho de mestrado que acarretou na parceira para desenvolvimento deste projeto foram identificadas a importância
do autocuidado no tratamento do DM e as principais soluções que estão sendo utilizadas pelo mercado nesse sentido.

Porém, apesar do aumento da informatização e da popularização dos \emph{smartphones},
PDV ainda enfrentam sérias dificuldades devido à falta de acessibilidade à DV em aplicações móveis. Assim, um MSL foi realizado
neste trabalho, buscando identificar as principais soluções de acessibilidade que estão sendo adotadas para
aplicativos móveis na literatura. 

Com isso, foram estabelecidas as funcionalidades que fariam parte do desenvolvimento de um aplicativo
móvel proposto como solução. Como resultado, o DiaVision foi desenvolvido com as principais
funcionalidades encontradas para o autocuidado de pacientes com DM, implementando 7 das 15 principais soluções
de acessibilidade à DV identificadas no MSL\@.

A usabilidade de uma versão inicial do \emph{app} foi avaliada com 77 pacientes com DV e os resultados
desse estudo mostraram que 90,1\% dos participantes gostariam de ter o \emph{app}, 62,3\% avaliaram-o com nota
máxima, 53,2\% utilizariam-o sempre e 35,1\% apontaram a necessidade de melhorias quanto à acessibilidade.

Contudo, parte das soluções de acessibilidade propostas para o \emph{app} foram implementadas após realização
desse estudo com a versão inicial. Ademais, não foram realizados testes de acessibilidade utilizando
as ferramentas automatizadas também identificadas neste estudo.

Portanto, trabalhos futuros devem concentrar-se na validação das novas soluções de acessibilidade do \emph{app}
por meio de nova pesquisa com usuários e da utilização das ferramentas para testes automatizados de acessibilidade.