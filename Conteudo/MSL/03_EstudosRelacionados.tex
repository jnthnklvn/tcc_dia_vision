\section{Estudos Relacionados}

Durante o processo de seleção de artigos do MSL, foram encontrados alguns estudos secundários, estudo que realiza uma revisão de estudos primários relacionados a um tema específico \cite{Kitchenham2007}.
Embora tenham sido rejeitados no MSL, por se enquadrarem em algum dos critérios definidos na seção anterior, os estudos que realizaram revisões dentro do tema estudado neste trabalho foram considerados como estudos relacionados.

Assim, esta seção apresenta os principais problemas e propostas de soluções relacionados à acessibilidade de aplicações para dispositivos móveis identificados por esses estudos.
No \autoref{qua-art-rev-sis} estão listadas as informações de cada um desses estudos secundários.

\begin{quadro}[htb!]
  \caption{\label{qua-art-rev-sis}Estudos relacionados identificados no processo de MSL.}
  \begin{tabular}{|m{0.8cm} | m{8.2cm} | m{2.7cm} | m{2.5cm}|}
    %\hline
    \hline
    \textbf{Sigla} & \textbf{Título}                                                                                                             & \textbf{Referência}  & \textbf{Base de dados}     \\
    \hline
    AR1            & \emph{Accessibility of Mobile Applications: Evaluation by Users with Visual Impairment and by Automated Tools}              & \cite{Mateus2020}    & \emph{ACM Digital Library} \\
    \hline
    AR2            & \emph{Can Everyone use my app? An Empirical Study on Accessibility in Android Apps}                                         & \cite{Vendome201941} & \emph{Scopus}              \\
    \hline
    AR3            & \emph{Effect of UX Design Guideline on the information accessibility for the visually impaired in the mobile health apps}   & \cite{Kim20191103}   & \emph{Scopus}              \\
    \hline
    AR4            & \emph{Mobile Device Accessibility for the Visually Impaired: Problems Mapping and Empirical Study of Touch Screen Gestures} & \cite{Damaceno2016}  & \emph{ACM Digital Library} \\
    \hline
    AR5            & \emph{Observation Based Analysis on the Use of Mobile Applications for Visually Impaired Users}                             & \cite{Siebra2016}    & \emph{ACM Digital Library} \\
    \hline
    AR6            & \emph{Prioritization of mobile accessibility guidelines for visual impaired users}                                          & \cite{Quispe2020}    & \emph{Scopus}              \\
    \hline
    % \hline
  \end{tabular}
  \legend{Fonte: Autor}
\end{quadro}

% ---
\subsection{\emph{Accessibility of Mobile Applications: Evaluation by Users with Visual Impairment and by Automated Tools}}
% ---

O artigo apresenta um estudo comparativo de problemas de acessibilidade encontrados pelas ferramentas automatizadas MATE (\emph{Mobile Accessibility Testing}) e \emph{Accessibility Scanner}, com os problemas encontrados em um estudo anterior envolvendo 11 usuários com DV\@.
Além disso, o trabalho sumarizou e categorizou os problemas mais encontrados pelos usuários.
As principais categorias são listadas na \autoref{tab-cat-pro-1}.

\begin{table}[htb]
  \begin{center}
    \ABNTEXfontereduzida
    \caption{Categorias dos tipos de problemas mais identificados.}
    \label{tab-cat-pro-1}
    \begin{tabular}{p{2.0cm}|p{7cm}}
      %\hline
      \textbf{Código} & \textbf{Categoria}                       \\
      \hline
      CPF1            & Botões                                   \\
      \hline
      CPF2            & Características do Sistema               \\
      \hline
      CPF3            & Conteúdo e Significado                   \\
      \hline
      CPF4            & Controles, formulários e funcionalidades \\
      \hline
      CPF5            & Imagem                                   \\
      % \hline
    \end{tabular}
    \legend{Fonte: \citeonline{Christoph2020}}
  \end{center}
\end{table}

Na \autoref{tab-pro-blind-1} são listados os principais tipos de problemas, que apresentaram um total de pelo menos 10 observações.
As categorias, de acordo com a \autoref{tab-cat-pro-1}, e os números de observações totais e para cada cada tipo de DV\@ também são relacionados à cada tipo de problema.
Como o artigo só menciona os tipos problemas encontrados com maior frequência por cada tipo de usuário, o número de observações de alguns não estão presentes na \autoref{tab-pro-blind-1}.

\begin{table}[htb]
  \begin{center}
    \ABNTEXfontereduzida
    \caption{Problemas mais frequentes encontrados pelos usuários por tipo de DV.}
    \label{tab-pro-blind-1}
    \begin{tabular}{p{10.5cm}|p{1.4cm}|p{0.6cm}|p{0.6cm}|p{0.7cm}}
      %\hline
      \textbf{Problema}                                                       & \textbf{Categoria} & \textbf{DVT} & \textbf{DVP} & \textbf{Total} \\
      \hline
      \emph{Feedback} inapropriado                                            & CPF4               & 34           & 15           & 49             \\
      \hline
      Falta de informações                                                    & CPF1               & 22           & 8            & 30             \\
      \hline
      Usuários presumiram que era uma funcionalidade                          & CPF4               & 18           & 9            & 27             \\
      \hline
      Funcionalidades confusas ou não claras                                  & CPF4               & 25           & -            & 25             \\
      \hline
      Apresentação padrão de elementos de controle ou formulário não adequada & CPF4               & 11           & 12           & 23             \\
      \hline
      Sequências de interação confusas ou não claras                          & CPF4               & 15           & 6            & 21             \\
      \hline
      Usuários não entenderam sentido do conteúdo                             & CPF3               & 15           & 5            & 20             \\
      \hline
      Organização do conteúdo inconsistente                                   & CPF3               & 12           & 6            & 18             \\
      \hline
      Funcionalidade não funciona como esperado                               & CPF4               & 6            & 10           & 16             \\
      \hline
      Funcionalidades dos botões confusas ou não claras                       & CPF1               & 15           & -            & 15             \\
      \hline
      Expectativa de funcionalidade que não existe                            & CPF4               & 10           & 5            & 15             \\
      \hline
      Sem alternativa textual                                                 & CPF5               & 14           & -            & 14             \\
      \hline
      Sistema muito lento                                                     & CPF2               & -            & 11           & 11             \\
      \hline
      Significado no conteúdo está perdido                                    & CPF3               & 6            & 4            & 10             \\
      % \hline
    \end{tabular}
    \legend{Fonte: \citeonline{Christoph2020}}
  \end{center}
\end{table}

Os resultados do estudo mostraram que 36 tipos de problemas foram encontrados somente pelos usuários, 11 somente pelas ferramentas e 3 por ambos os métodos.
Evidenciando assim a necessidade de utilização de mais de um método para identificação dos problemas de acessibilidade.
Além disso, o estudo mostrou a importância da utilização dessas ferramentas automatizadas, visto que parte significativa dos problemas podem ser identificados ainda no processo de desenvolvimento, reduzindo o esforço e, consequentemente, o custo para solucioná-los.

% ---
\subsection{\emph{Can Everyone use my app? An Empirical Study on Accessibility in Android Apps}}
% ---

Esse trabalho realizou um estudo piloto onde foi observado que desenvolvedores de aplicativos móveis raramente utilizam as APIs de Acessibilidade e que o uso de descrições alternativas para elementos de \emph{interface} também é limitado.
Assim, visando entender a perspectiva desses desenvolvedores, o estudo também realizou uma investigação de postagens no \emph{Stack Overflow}, identificando os aspectos de acessibilidade que os desenvolvedores implementavam e os que experienciavam dificuldades.

O estudo investigou aspectos de acessibilidade no geral, baseado em 336 discussões de desenvolvedores \emph{Android} no \emph{Stack Overflow}, sendo 159 dessas sobre acessibilidade à DV\@.
Dessas 159 discussões, os principais aspectos discutidos foram sobre \emph{feedbacks} sonoros e legibilidade (114 e 24 postagens, respectivamente) como mostra a \autoref{tab-acc-asp-sta-flow}.

\begin{table}[htb]
  \begin{center}
    \ABNTEXfontereduzida
    \caption{Aspectos de acessibilidade à DV discutidos por \emph{devs Android} no \emph{Stack Overflow}.}
    \label{tab-acc-asp-sta-flow}
    \begin{tabular}{p{7.0cm}|p{3.5cm}}
      %\hline
      \textbf{Aspecto}                       & \textbf{Categoria}       \\
      \hline
      Alertas de acessibilidade              & \emph{Feedbacks} sonoros \\
      \hline
      Ampliação da tela                      & Legibilidade             \\
      \hline
      Aspectos não funcionais                & \emph{Feedbacks} sonoros \\
      \hline
      Consciência de contexto                & \emph{Feedbacks} sonoros \\
      \hline
      Conteúdos, ações e gestos customizados & \emph{Feedbacks} sonoros \\
      \hline
      \emph{Frameworks} de terceiros         & \emph{Feedbacks} sonoros \\
      \hline
      \emph{Mobile web apps}                 & \emph{Feedbacks} sonoros \\
      \hline
      Problemas com serviços                 & \emph{Feedbacks} sonoros \\
      \hline
      Sons e vibrações                       & \emph{Feedbacks} sonoros \\
      \hline
      Suporte à \emph{Braille}               & Teclados alternativos    \\
      \hline
      Tamanho de fonte                       & Legibilidade             \\
      \hline
      Teclado customizado                    & Teclados alternativos    \\
      \hline
      Transformações de cores                & Transformações de cores  \\
      % \hline
    \end{tabular}
    \legend{Fonte: \citeonline{Vendome201941}}
  \end{center}
\end{table}

No estudo piloto, o trabalho de \citeonline{Vendome201941} analisou 13.817 \emph{apps Android} de código aberto, descobrindo que cerca de 50\% deles tinham descrições alternativas para todos os elementos, enquanto cerca de 37\% não tinha nenhuma.
Além disso, o artigo apontou que apenas cerca de 2\% desses \emph{apps} utilizavam alguma API de acessibilidade no projeto.

% ---
\subsection{\emph{Effect of UX Design Guideline on the information accessibility for the visually impaired in the mobile health apps}}
% ---

Acessibilidade de informações visuais para DV raramente é considerada ao projetar aplicações móveis para saúde \cite{Kim20191103}.
O artigo propõe um guia de diretrizes de acessibilidade à DV, chamado UXDG (\emph{UX Design Guideline}), para resolver esse problema.
120 \emph{apps} na área de saúde foram analisados quanto à taxa de conformidade com o guia.

A \autoref{tab-acc-dir-uxd-1} lista as diretrizes do UXDG de acordo com as categorias.
Na análise dos 120 \emph{apps}, a média da taxa de conformidade com o guia foi de 39,24\%, com a diretriz XD7 apresentando a maior taxa, com 71,67\%, enquanto a XD9 apresentou a menor, com 5\%.

\begin{table}[htb]
  \begin{center}
    \ABNTEXfontereduzida
    \caption{Diretrizes do UXDG por categoria.}
    \label{tab-acc-dir-uxd-1}
    \begin{tabular}{p{1.0cm}|p{9.0cm}|p{4.5cm}}
      %\hline
      \textbf{Código} & \textbf{Diretriz}                                                   & \textbf{Categoria}             \\
      \hline
      XD1             & Destacar as mídias que disparam ação                                & Aquisição de informação        \\
      \hline
      XD2             & Destacar as principais imagens que o usuário pode acessar           & Aquisição de informação        \\
      \hline
      XD3             & Navegação intuitiva                                                 & Acessibilidade dos dados       \\
      \hline
      XD4             & Posicionar a caixa de pesquisa sempre no local                      & Busca de dados                 \\
      \hline
      XD5             & Posicionar resultados de buscas logo após a caixa de texto          & Busca de dados                 \\
      \hline
      XD6             & Reconhecimento de voz para entrada de texto                         & Busca de dados                 \\
      \hline
      XD7             & Resposta intuitiva do \emph{menu} de acordo com intenção do usuário & Acessibilidade dos dados       \\
      \hline
      XD8             & Suporte à esquemas de cores alternativos                            & Melhora na exposição dos dados \\
      \hline
      XD9             & Suporte de \emph{zoom in/out} para os principais conteúdos          & Melhora na exposição dos dados \\
      \hline
      XD10            & Suporte para outros métodos entrada além do toque                   & Acessibilidade dos dados       \\
      \hline
      XD11            & Uso de fontes com alta legibilidade                                 & Aquisição de informação        \\
      % \hline
    \end{tabular}
    \legend{Fonte: \citeonline{Kim20191103}}
  \end{center}
\end{table}

O estudo realizou testes, conduzidos com 23 PDV e 23 sem DV, comparando \emph{apps} selecionados da área da saúde antes e depois da aplicação do UXDG\@.
Os resultados apontam que houve um aumento na velocidade de reconhecimento das informações depois de aplicar as diretrizes.
De acordo com o experimento, esse aumento aconteceu tanto para usuários com DV, aumento de 13,68\%, quanto para os sem, de 32,41\%.

% ---
\subsection{\emph{Mobile Device Accessibility for the Visually Impaired: Problems Mapping and Empirical Study of Touch Screen Gestures}}
% ---

Esse artigo, através de um MSL, apresenta os problemas de acessibilidade enfrentados na utilização de dispositivos móveis por PDV encontrados na literatura.
A \autoref{tab-cat-pro-4} mostra, como categorias, 6 dos 7 grupos de problemas identificados no estudo,
desconsiderando o de ``borda não sensível ao toque'', visto que é um problema relativo aos dispositivos físicos.

\begin{table}[htb]
  \begin{center}
    \ABNTEXfontereduzida
    \caption{Categorias dos problemas mapeados na literatura.}
    \label{tab-cat-pro-4}
    \begin{tabular}{p{2.0cm}|p{5.0cm}}
      %\hline
      \textbf{Código} & \textbf{Categoria}   \\
      \hline
      CPM1            & Botões               \\
      \hline
      CPM2            & Comandos de voz      \\
      \hline
      CPM3            & Entrada de dados     \\
      \hline
      CPM4            & Interação por gestos \\
      \hline
      CPM5            & Leitor de tela       \\
      \hline
      CPM6            & Retorno ao usuário   \\
      % \hline
    \end{tabular}
    \legend{Fonte: \citeonline{Damaceno2016}}
  \end{center}
\end{table}

Na \autoref{tab-pro-1-2-6} são listados os problemas relacionados à botões (CPM1), comandos de voz (CPM2) e retorno do usuário (CPM6) e o número de citações, que corresponde ao número de estudos onde o problema foi identificado.
Sendo que os problemas relacionados aos botões físicos dos dispositivos foram desconsiderados, por estarem fora do controle da aplicação.

\begin{table}[htb]
  \begin{center}
    \ABNTEXfontereduzida
    \caption{Problemas relacionados às categorias CPM1, CPM2 e CPM6.}
    \label{tab-pro-1-2-6}
    \begin{tabular}{p{1.4cm}|p{11.7cm}|p{1.4cm}}
      %\hline
      \textbf{Categoria} & \textbf{Problema}                                                                               & \textbf{Citações} \\
      \hline
      CPM1               & A grande proximidade entre os botões virtuais dificulta a interação                             & 1                 \\
      \hline
      CPM1               & Os botões virtuais acarretam menor sensibilidade tátil                                          & 1                 \\
      \hline
      CPM2               & Apenas um comando de voz é reconhecido por vez                                                  & 2                 \\
      \hline
      CPM2               & Há baixa privacidade ao emitir comandos de voz                                                  & 1                 \\
      \hline
      CPM2               & Há diminuição do desempenho do reconhecimento em condições de ruído                             & 1                 \\
      \hline
      CPM2               & Há diminuição do desempenho do reconhecimento devido à entonação e à acentuação                 & 1                 \\
      \hline
      CPM2               & Há dificuldade para ativar comando de voz                                                       & 1                 \\
      \hline
      CPM2               & Há necessidade de mentalizar instrução por voz, aumentando carga de memória do indivíduo        & 1                 \\
      \hline
      CPM2               & O reconhecimento de voz funciona apenas em alguns aplicativos                                   & 1                 \\
      \hline
      CPM2               & O uso de comandos de voz é computacionalmente custoso                                           & 1                 \\
      \hline
      CPM6               & Há ausência de retorno ao usuário, ao interagir com alguns elementos de interface               & 1                 \\
      \hline
      CPM6               & Há dificuldade para compreender diferentes padrões vibratórios                                  & 1                 \\
      \hline
      CPM6               & Há dificuldade para compreender a orientação da interface, utilizando apenas o retorno auditivo & 1                 \\
      \hline
      CPM6               & Retorno auditivo é prejudicado em ambientes ruidosos                                            & 2                 \\
      \hline
      CPM6               & Usar apenas o retorno auditivo não é o suficiente para a interação                              & 1                 \\
      % \hline
    \end{tabular}
    \legend{Fonte: \citeonline{Damaceno2016}}
  \end{center}
\end{table}

\newpage

A \autoref{tab-pro-ent-dad-1} mostra os problemas relacionados à entrada de dados (CM3) com o número de citações para cada problema.
Os problemas que mencionavam teclado físico de dispositivos móveis foram desconsiderados, pois a aplicação a ser desenvolvida suporta apenas \emph{smartphones}.

\begin{table}[htb]
  \begin{center}
    \ABNTEXfontereduzida
    \caption{Problemas relacionados à entrada de dados (CM3).}
    \label{tab-pro-ent-dad-1}
    \begin{tabular}{p{13.1cm}|p{1.4cm}}
      %\hline
      \textbf{Problema}                                                                                                                  & \textbf{Citações} \\
      \hline
      A digitação de textos é lenta em teclados QWERTY virtuais                                                                          & 2                 \\
      \hline
      As teclas mais distantes das bordas são mais difíceis de encontrar do que as mais próximas das bordas, em teclados virtuais QWERTY & 1                 \\
      \hline
      É preciso conhecer previamente Braille para ter bom desempenho de digitação utilizando esta modalidade                             & 2                 \\
      \hline
      É preciso trocar o modo do teclado virtual, para acessar determinados caracteres                                                   & 1                 \\
      \hline
      Há ausência de marca tátil para o número 5, no teclado numérico virtual, e para as letras “F” e “J” no teclado QWERTY virtual      & 2                 \\
      \hline
      Há erros ao corrigir caracteres digitados equivocadamente, substituindo por fonemas semelhantes, em teclados virtuais              & 1                 \\
      \hline
      Há erros de omissão de caracteres, faltando um ou mais ao digitar palavras em teclados virtuais                                    & 1                 \\
      \hline
      Há necessidade de confirmação de cada caractere digitado em teclados virtuais                                                      & 1                 \\
      \hline
      Há necessidade de navegar pelo teclado virtual para localizar os caracteres desejados                                              & 1                 \\
      \hline
      Há um segundo de espera para entrar com cada tecla em teclados virtuais                                                            & 1                 \\
      \hline
      O teclado numérico virtual é denso dificultando, a interação                                                                       & 1                 \\
    \end{tabular}
    \legend{Fonte: \citeonline{Damaceno2016}}
  \end{center}
\end{table}

A \autoref{tab-pro-int-ges-1} lista os problemas relacionados à interação por gestos (CM4) com o número de citações para cada problema encontrado.

\begin{table}[htb]
  \begin{center}
    \ABNTEXfontereduzida
    \caption{Problemas relacionados à interação por gestos (CM4).}
    \label{tab-pro-int-ges-1}
    \begin{tabular}{p{13.1cm}|p{1.4cm}}
      %\hline
      \textbf{Problema}                                                                                                                        & \textbf{Citações} \\
      \hline
      A baixa flexibilidade de ângulo e velocidade dos gestos, por parte do sistema, dificultam seu reconhecimento                             & 1                 \\
      \hline
      Gestos representados pela forma da letra “L” são difíceis de fazer                                                                       & 2                 \\
      \hline
      Gestos representados por formas geométricas fechadas (círculo e triângulo) são difíceis de fazer                                         & 1                 \\
      \hline
      Gestos representados por formas geométricas são lentos de se fazer                                                                       & 1                 \\
      \hline
      Há conflito na desambiguação entre dois toques com um dedo e três toques com um dedo                                                     & 1                 \\
      \hline
      Há dificuldade para fazer gestos estando em movimento                                                                                    & 1                 \\
      \hline
      Há dificuldade para fazer gestos próximos à barra superior de sistemas                                                                   & 1                 \\
      \hline
      Há dificuldade para fazer gestos representados por símbolos, sendo maior no caso de pessoas com deficiência visual de nascença           & 2                 \\
      \hline
      Há dificuldade para fazer o gesto de dois toques com um dedo                                                                             & 1                 \\
      \hline
      Há dificuldade para se localizar na tela para realizar gestos                                                                            & 1                 \\
      \hline
      Há erros na identificação de gestos multitoque, já que o sistema, por vezes, falha para reconhecer mais de um dedo em contato com a tela & 1                 \\
      \hline
      Há falha de interpretação de gestos em geral, pelo sistema                                                                               & 4                 \\
      \hline
      Há mudança indevida de foco ao tentar fazer o gesto dois toques com um dedo                                                              & 1                 \\
      \hline
      Não é possível alterar mapeamento dos gestos às funções do sistema                                                                       & 1                 \\
      \hline
      Não há consistência de gestos entre diferentes sistemas                                                                                  & 1                 \\
      \hline
      Não há gestos que acionam as principais funções do sistema                                                                               & 1                 \\
      \hline
      O toque acidental na tela, com outro dedo, prejudica o reconhecimento de gestos                                                          & 1                 \\
      \hline
      Os manuais de explicação de como fazer gestos de toque não são eficientes                                                                & 3                 \\
      \hline
      Quando um aplicativo aceita gestos de toque próprios, há conflito entre estes gestos e os do leitor de tela do sistema                   & 1                 \\
      % \hline
    \end{tabular}
    \legend{Fonte: \citeonline{Damaceno2016}}
  \end{center}
\end{table}

\newpage

Por fim, são listados, na \autoref{tab-pro-lei-tel-1}, os problemas relacionados a leitores de tela (CM5) com o número de citações.

\begin{table}[htb]
  \begin{center}
    \ABNTEXfontereduzida
    \caption{Problemas relacionados a leitores de tela (CM5).}
    \label{tab-pro-lei-tel-1}
    \begin{tabular}{p{13.1cm}|p{1.4cm}}
      %\hline
      \textbf{Problema}                                                                                & \textbf{Citações} \\
      \hline
      A leitura é linear, demorando para se ter noção global da interface                              & 2                 \\
      \hline
      A pronúncia de algumas palavras é problemática                                                   & 1                 \\
      \hline
      A voz do leitor de tela é artificial                                                             & 1                 \\
      \hline
      Alguns elementos de interface não são lidos                                                      & 3                 \\
      \hline
      Há baixa familiaridade com o leitor de tela de dispositivos móveis                               & 1                 \\
      \hline
      Há conflito ao usar o leitor de tela do sistema em conjunto com o leitor embutido em aplicativos & 2                 \\
      \hline
      Há desconforto ao ouvir o leitor de tela em ambientes ruidosos                                   & 2                 \\
      \hline
      Há leitura de apenas o que está em foco                                                          & 1                 \\
      \hline
      Não há controle de velocidade de leitura                                                         & 2                 \\
      \hline
      Não há um botão para interromper a leitura imediatamente                                         & 1                 \\
      \hline
      O foco do leitor de tela muda indevidamente                                                      & 2                 \\
      \hline
      O foco do leitor de tela não possui uma ordem de navegação lógica                                & 2                 \\
      \hline
      O leitor de tela é lento                                                                         & 1                 \\
      \hline
      O texto lido é, por vezes, inadequado                                                            & 1                 \\
      % \hline
    \end{tabular}
    \legend{Fonte: \citeonline{Damaceno2016}}
  \end{center}
\end{table}

% ---
\subsection{\emph{Observation Based Analysis on the Use of Mobile Applications for Visually Impaired Users}}
% ---

O estudo realizou uma análise, envolvendo 5 PDV, com o objetivo de validar se a falta dos requisitos de acessibilidade levantados em um trabalho anterior realmente impactavam na utilização de \emph{apps} móveis por PDV.

\begin{table}[htb]
  \begin{center}
    \ABNTEXfontereduzida
    \caption{Categorias dos requisitos encontrados.}
    \label{tab-cat-req-enc-5}
    \begin{tabular}{p{2.0cm}|p{5.0cm}}
      %\hline
      \textbf{Código} & \textbf{Categoria}                \\
      \hline
      CRED1           & \emph{Feedbacks} audíveis         \\
      \hline
      CRED2           & Adaptação das informações visuais \\
      \hline
      CRED3           & Navegação                         \\
      % \hline
    \end{tabular}
    \legend{Fonte: \citeonline{Siebra2016}}
  \end{center}
\end{table}

Os requisitos foram divididos em 3 categorias, como mostra a \autoref{tab-cat-req-enc-5}.
Baseados na análise dos resultados, o estudo qualificou os requisitos em 3 níveis (Essencial, Desejável e Não observado).
Como os requisitos ``não observados'', de acordo com o artigo, não foram mencionados pelos participantes dos testes, apenas os requisitos essenciais e desejáveis são listados na \autoref{tab-req-ess-des-1}.
Somente um requisito foi classificado como desejável, o RED7, o restante foi classificado como essencial.

\begin{table}[htb]
  \begin{center}
    \ABNTEXfontereduzida
    \caption{Requisitos essenciais e desejáveis focados em DV.}
    \label{tab-req-ess-des-1}
    \begin{tabular}{p{1.0cm}|p{12.1cm}|p{1.4cm}}
      %\hline
      \textbf{Código} & \textbf{Problema}                                                                                                & \textbf{Categoria} \\
      \hline
      RED1            & O nome do caractere que está sendo digitado deve ser ouvido                                                      & CRED1              \\
      \hline
      RED2            & Nomes de elementos e imagens na tela devem ser ouvidos ao serem tocados ou selecionados                          & CRED1              \\
      \hline
      RED3            & \emph{Feedback} de ações/interações devem ser claros e fornecidos de forma tátil, voz ou eventos sonoros         & CRED1              \\
      \hline
      RED4            & Estratégias para o uso de leitores de tela (ex.\@: atalhos para navegar na tela de forma mais eficiente)         & CRED1              \\
      \hline
      RED5            & Prover uma chave ``home'' tátil de acesso fácil e rápido para que um usuário possa retornar a um lugar conhecido & CRED2              \\
      \hline
      RED6            & Prover documentação em formatos alternativos, utilizando fontes grandes                                          & CRED3              \\
      \hline
      RED7            & Permitir customizações pelo usuário e evitar que essas preferências sejam perdidas                               & CRED3              \\
      \hline
      RED8            & Apresentar amplificador com \emph{zoom} ajustável                                                                & CRED3              \\
      \hline
      RED9            & Prover equivalências textuais claras para evitar erros quando os textos são lidos na tela                        & CRED3              \\
      \hline
      RED10           & Brilho, contrate e cores ajustáveis                                                                              & CRED3              \\
      \hline
      RED11           & Prover alertas informativos por outros canais além do visual (ex.\@: voz)                                        & CRED3              \\
      % \hline
    \end{tabular}
    \legend{Fonte: \citeonline{Siebra2016}}
  \end{center}
\end{table}

% ---
\subsection{\emph{Prioritization of mobile accessibility guidelines for visual impaired users}}
% ---

O artigo apresenta uma proposta de priorização de diretrizes de acessibilidade que resultaram de estudos anteriores.
Essas diretrizes foram baseadas no eMAG, porém diretrizes como as da BCC (\emph{BBC Mobile Accessibility Guidelines}) e recomendações da plataforma \emph{Android} também foram consideradas.
Para criação do \emph{ranking}, o estudo utilizou um questionário que foi respondido 103 vezes, sendo 66 com DV, onde a análise se concentrou.

O estudo dividiu as diretrizes em 6 categorias que podem ser visualizadas na \autoref{tab-cat-dir-acc-5}.

\begin{table}[htb]
  \begin{center}
    \ABNTEXfontereduzida
    \caption{Categorias das diretrizes de acessibilidade \emph{mobile} baseadas no eMAG.}
    \label{tab-cat-dir-acc-5}
    \begin{tabular}{p{1.5cm}|p{4.5cm}}
      %\hline
      \textbf{Código} & \textbf{Categoria}         \\
      \hline
      EGE             & Estrutura                  \\
      \hline
      EGC             & Comportamento              \\
      \hline
      EGCI            & Conteúdo/Informação        \\
      \hline
      EGAD            & Apresentação/\emph{Design} \\
      \hline
      EGM             & Multimídia                 \\
      \hline
      EGF             & Formulários                \\
      % \hline
    \end{tabular}
    \legend{Fonte: \citeonline{Quispe2020}}
  \end{center}
\end{table}

O estudo considerou a priorização para 4 grupos diferentes, baseados no tipo de DV (baixa visão, visão parcial e os 2 tipos de cegueira: legal e total).
E os resultados mostraram que existiam diferenças notáveis na percepção das diretrizes entre os grupos.

Assim, a partir desses resultados, o trabalho relacionou as diretrizes com as percepções de cada grupo
e criou a lista de priorização que pode ser vista no \autoref{qua-pri-acc-gui}.
Onde a coluna id informa a ordem de priorização e os códigos que estão nas outras colunas são listados na \autoref{tab-dir-acc-mob-1} junto com as diretrizes.

\begin{quadro}[htb!]
  \begin{center}
    \ABNTEXfontereduzida
    \caption{\label{qua-pri-acc-gui}Priorização de diretrizes de acessibilidade para usuários com DV.}
    \begin{tabular}{|m{0.5cm} | m{2.4cm} | m{2.4cm} | m{2.8cm} | m{3.0cm} | m{2.4cm}|}
      %\hline
      \hline
      \textbf{Id} & \textbf{Visão parcial} & \textbf{Baixa visão} & \textbf{Cegueira legal} & \textbf{Cegueira total}        & \textbf{Todas as DV} \\
      \hline
      1           & EGC2, G28              & EGC4                 & EGAD3, G28              & EGC3, EGAD1, EGAD2             & EGC3                 \\
      \hline
      2           & EGM1                   & EGC6, G28            & EGC2, EGC6, EGAD1, EGM1 & EGE1, EGC1, EGM1, G28          & EGM1                 \\
      \hline
      3           & EGAD3                  & EGM1                 & EGE1, EGM3              & EGC2, EGC4, EGCI4, EGAD3, EGM3 & EGC2, EGC4, EGAD1    \\
      \hline
      4           & EGC4, EGAD1            & EGC2, EGAD1          & EGC4                    & EGC6                           & EGC6, EGAD3          \\
      \hline
      5           & EGC6                   & EGC5                 & EGC5                    & EGC5                           & EGM3                 \\
      \hline
      6           & EGC5, EGM3             & EGCI4                & EGC3, EGAD2             & \-                             & EGE1                 \\
      \hline
      7           & EGC3, EGCI4            & EGE1, EGC1, EGM3     & EGC1                    & \-                             & EGC3                 \\
      \hline
      8           & EGC1                   & EGC3                 & EGCI4                   & \-                             & EGC1, EGAD2          \\
      \hline
      9           & EGE1                   & EGAD2, EGAD3         & \-                      & \-                             & EGC5, EGCI4          \\
      \hline
      10          & EGAD2                  & \-                   & \-                      & \-                             & \-                   \\
      \hline
      % \hline
    \end{tabular}
    \legend{Fonte: \citeonline{Quispe2020}}
  \end{center}
\end{quadro}

\begin{table}[htb]
  \begin{center}
    \ABNTEXfontereduzida
    \caption{Diretrizes de acessibilidade \emph{mobile} baseadas no eMAG.}
    \label{tab-dir-acc-mob-1}
    \begin{tabular}{p{1.0cm}|p{13.0cm}}
      %\hline
      \textbf{Código} & \textbf{Diretriz}                                                                                                                            \\
      \hline
      EGE1            & Elementos de tela devem ser organização de maneira lógica e semântica.                                                                       \\
      \hline
      EGE2            & As telas devem apresentar sequência lógica de leitura para navegação entre \emph{links}, controles de formulário e outros elementos.         \\
      \hline
      EGE3            & \emph{Links} na tela devem ser organizados para evitar confusão.                                                                             \\
      \hline
      EGE4            & Informações devem ser divididas em grupos específicos para facilitar a procura e leitura dos conteúdos.                                      \\
      \hline
      EGE5            & Usuários devem ser informados se \emph{links} abrem novas telas para poderem decidir se querem ou não sair da tela atual.                    \\
      \hline
      EGC1            & Todas as funcionalidades na tela devem estar disponíveis a partir do teclado.                                                                \\
      \hline
      EGC2            & Todos os elementos de \emph{interface} na tela devem ser acessíveis.                                                                         \\
      \hline
      EGC3            & Redirecionamento automático de telas não deve acontecer.                                                                                     \\
      \hline
      EGC4            & Em telas com limite de tempo, deve haver opções para desligar ou ajustar o tempo.                                                            \\
      \hline
      EGC5            & Não deve haver efeitos visuais piscantes, intermitentes ou cintilantes na tela.                                                              \\
      \hline
      EGC6            & Conteúdos animados não devem iniciar automaticamente.                                                                                        \\
      \hline
      EGCI1           & A linguagem utilizada na tela deve ser especificada.                                                                                         \\
      \hline
      EGCI2           & Mudanças na linguagem dos conteúdos sempre devem ser especificadas.                                                                          \\
      \hline
      EGCI3           & Títulos de telas devem ser descritivos, informativos e representativos com relação ao conteúdo principal.                                    \\
      \hline
      EGCI4           & Deve haver algum mecanismo para indicar ao usuário onde ele está no momento, no conjunto de telas.                                           \\
      \hline
      EGCI5           & Alvos de \emph{links} devem ser identificados claramente, incluindo informações sobre se estão funcionando ou se direcionam para outra tela. \\
      \hline
      EGCI6           & Todas as imagens devem possuir descrição textual.                                                                                            \\
      \hline
      EGCI7           & Documentos em formatos acessíveis devem estar disponíveis.                                                                                   \\
      \hline
      EGCI8           & Quando uma tabela é utilizada na tela, título e sumário apropriados devem ser fornecidos.                                                    \\
      \hline
      EGCI9           & Os textos nas telas devem ser fáceis de ler e entender.                                                                                      \\
      \hline
      EGCI10          & Totos as siglas, abreviações e palavras incomuns na tela devem possuir explicação.                                                           \\
      \hline
      EGAD1           & Deve haver uma taxa miníma de contraste entre as cores de fundo e as de frente.                                                              \\
      \hline
      EGAD2           & Características sensoriais (ex.\: cores, formas e sons) não podem ser o único significado para distinguir elementos de tela.                 \\
      \hline
      EGAD3           & O elemento ou área em foco deve ser evidente visualmente.                                                                                    \\
      \hline
      EGM1            & Vídeos que não incluem áudio devem fornecer alternativas como legendas.                                                                      \\
      \hline
      EGM2            & Deve haver alternativas a conteúdo de áudio (ex.\: transcrição ou linguagem de sinais).                                                      \\
      \hline
      EGM3            & Conteúdos visuais que não estão disponíveis como áudio devem ser descritos.                                                                  \\
      \hline
      EGM4            & Devem haver mecanismos para controlar áudios da aplicação.                                                                                   \\
      \hline
      EGM5            & Devem haver mecanismos para controlar animações que iniciam automaticamente.                                                                 \\
      \hline
      EGF1            & Botões de imagem ou conteúdos de áudio em formulários devem possuir alternativas textuais.                                                   \\
      \hline
      EGF2            & Todos os campos do formulário devem ser identificados.                                                                                       \\
      \hline
      EGF3            & Uma ordem lógica na navegação pelo formulário deve ser garantida.                                                                            \\
      \hline
      EGF4            & Não devem haver mudanças automáticas quando um elemento do formulário é focado, para não confundir ou desorientar o usuário.                 \\
      \hline
      EGF5            & Formulários devem possuir instruções de preenchimento.                                                                                       \\
      \hline
      EGF6            & Erros de entrada devem sempre ser descritos e as submissões de dados confirmadas.                                                            \\
      % \hline
    \end{tabular}
    \legend{Fonte: \citeonline{Quispe2020}}
  \end{center}
\end{table}
