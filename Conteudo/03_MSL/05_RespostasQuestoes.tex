\newpage

\section{Respostas das Questões do Protocolo}

Nesta seção são respondidas as questões definidas no Protocolo de MSL, apresentado no início deste capítulo, levando em consideração a análise dos resultados da seção anterior.
Assim, seguem:

\begin{enumerate}
    \item Quais são as principais soluções de acessibilidade para PDV utilizadas
          no desenvolvimento de aplicações móveis?
          \begin{enumerate}
              \item Cumprimento de diretrizes de acessibilidade para aplicações móveis propostas por entidades
                    como \emph{Google}\footnote{\url{https://developer.android.com/guide/topics/ui/accessibility/apps}},
                    SIDI\footnote{\url{https://www.sidi.org.br/guiadeacessibilidade}} e
                    \emph{BBC}\footnote{\url{https://www.bbc.co.uk/accessibility/forproducts/guides/mobile}}
                    durante o processo de desenvolvimento;
              \item Realização de estudos visando identificar problemas enfrentados por usuários com DV no uso de aplicativos móveis e definir diretrizes para solucionar cada problema;
              \item Utilização de diretrizes de acessibilidade da \emph{W3C}\footnote{\url{https://www.w3.org/TR/mobile-bp/summary}} adaptadas da \emph{web} para o contexto de aplicações móveis;
              \item Utilização de ferramentas para realização de testes de acessibilidade automatizados.
          \end{enumerate}
    \item Quais são as tecnologias utilizadas no desenvolvimento dessas soluções?
          \begin{enumerate}
              \item A linguagem \emph{Java} no desenvolvimento de aplicações \emph{Android};
              \item Os \emph{frameworks} \emph{React Native}, \emph{Cordova} e \emph{MD²} no desenvolvimento multiplataforma;
              \item As IDEs \emph{Android Studio} e \emph{Eclipse};
              \item \emph{Unity 3D} no desenvolvimento de jogos para \emph{Android};
              \item As ferramentas \emph{Accessibility Scanner App}, \emph{MATE} e \emph{Test Lab} para realização de testes de acessibilidade automatizados.
          \end{enumerate}
    \item Para quais plataformas as soluções foram propostas?
          \begin{enumerate}
              \item 9 apenas para \emph{Android};
              \item 2 apenas para \emph{iOS};
              \item 4 multiplataforma, para \emph{Android} e \emph{iOS}.
          \end{enumerate}
    \item Quem são os públicos alvos dessas soluções? \\
          As soluções visaram atender diversas necessidades de PDV\@.
          Assim, foram criadas aplicações para diferentes públicos com DV,
          como crianças e estudantes, e com diversos tópicos específicos,
          como livros, medicamentos e clima.
\end{enumerate}
