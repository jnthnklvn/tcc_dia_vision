\section{Estudos Relacionados}

Durante o processo de seleção de artigos do MSL, foram encontrados alguns estudos secundários, estes que realizam revisões de estudos primários
relacionados a um tema específico \cite{Kitchenham2007}. Embora tenham sido rejeitados no MSL, por se enquadrarem em algum dos critérios definidos
na seção anterior, os estudos que realizaram revisões dentro do tema estudado neste trabalho foram considerados como estudos relacionados.

Assim, esta seção apresenta os principais problemas e propostas de soluções relacionados à acessibilidade de aplicações para dispositivos móveis
identificados por esses estudos. No \autoref{qua-art-rev-sis} estão listadas as informações de cada um desses estudos secundários.

\begin{quadro}[htb!]
  \caption{\label{qua-art-rev-sis}Estudos relacionados identificados no processo de MSL.}
  \begin{tabular}{|m{1.2cm} | m{8.1cm} | m{2.7cm} | m{2.5cm}|}
    %\hline
    \hline
    \textbf{Código} & \textbf{Título}                                                                                                             & \textbf{Referência}  & \textbf{Base de dados}     \\
    \hline
    AR1             & \emph{Accessibility of Mobile Applications: Evaluation by Users with Visual Impairment and by Automated Tools}              & \cite{Mateus2020}    & \emph{ACM Digital Library} \\
    \hline
    AR2             & \emph{Can Everyone use my app? An Empirical Study on Accessibility in Android Apps}                                         & \cite{Vendome201941} & \emph{Scopus}              \\
    \hline
    AR3             & \emph{Effect of UX Design Guideline on the information accessibility for the visually impaired in the mobile health apps}   & \cite{Kim20191103}   & \emph{Scopus}              \\
    \hline
    AR4             & \emph{Mobile Device Accessibility for the Visually Impaired: Problems Mapping and Empirical Study of Touch Screen Gestures} & \cite{Damaceno2016}  & \emph{ACM Digital Library} \\
    \hline
    AR5             & \emph{Observation Based Analysis on the Use of Mobile Applications for Visually Impaired Users}                             & \cite{Siebra2016}    & \emph{ACM Digital Library} \\
    \hline
    AR6             & \emph{Prioritization of mobile accessibility guidelines for visual impaired users}                                          & \cite{Quispe2020}    & \emph{Scopus}              \\
    \hline
    % \hline
  \end{tabular}
  \legend{Fonte: Autor.}
\end{quadro}

% ---
\subsection{\emph{Accessibility of Mobile Applications: Evaluation by Users with Visual Impairment and by Automated Tools}}
% ---

O artigo apresenta um estudo comparativo de problemas de acessibilidade encontrados pelas ferramentas automatizadas MATE e \emph{Accessibility Scanner}, com os problemas encontrados em um estudo anterior envolvendo 11 usuários com DV\@.
Além disso, o trabalho sumarizou e categorizou os problemas mais encontrados pelos usuários.
As principais categorias são listadas na \autoref{tab-cat-pro-1}.

\begin{table}[htb]
  \begin{center}
    \ABNTEXfontereduzida
    \caption{Categorias dos tipos de problemas mais identificados.}
    \label{tab-cat-pro-1}
    \begin{tabular}{p{2.0cm}|p{7cm}}
      %\hline
      \textbf{Código} & \textbf{Categoria}                       \\
      \hline
      CPF1            & Botões                                   \\
      \hline
      CPF2            & Características do Sistema               \\
      \hline
      CPF3            & Conteúdo e Significado                   \\
      \hline
      CPF4            & Controles, formulários e funcionalidades \\
      \hline
      CPF5            & Imagem                                   \\
      % \hline
    \end{tabular}
    \legend{Fonte: \citeonline{Christoph2020}.}
  \end{center}
\end{table}

Na \autoref{tab-pro-blind-1} são listados os principais tipos de problemas, que apresentaram um total de pelo menos 10 observações.
As categorias, de acordo com a \autoref{tab-cat-pro-1}, e o número total de observações para cada tipo de DV\@ (total ou parcial) também são relacionados à cada tipo de problema.
Como o artigo só menciona os tipos problemas encontrados com maior frequência por cada tipo de usuário, o número de observações de alguns não estão presentes na \autoref{tab-pro-blind-1}.

\begin{table}[htb]
  \begin{center}
    \ABNTEXfontereduzida
    \caption{Problemas mais frequentes encontrados pelos usuários por tipo de DV.}
    \label{tab-pro-blind-1}
    \begin{tabular}{p{1.2cm}|p{8.7cm}|p{1.4cm}|p{0.6cm}|p{0.6cm}|p{0.7cm}}
      %\hline
      \textbf{Código} & \textbf{Problema}                                                       & \textbf{Categoria} & \textbf{DVT} & \textbf{DVP} & \textbf{Total} \\
      \hline
      AR1P1           & \emph{Feedback} inapropriado                                            & CPF4               & 34           & 15           & 49             \\
      \hline
      AR1P2           & Falta de informações                                                    & CPF1               & 22           & 8            & 30             \\
      \hline
      AR1P3           & Usuários presumiram que era uma funcionalidade                          & CPF4               & 18           & 9            & 27             \\
      \hline
      AR1P4           & Funcionalidades confusas ou não claras                                  & CPF4               & 25           & -            & 25             \\
      \hline
      AR1P5           & Apresentação padrão de elementos de controle ou formulário não adequada & CPF4               & 11           & 12           & 23             \\
      \hline
      AR1P6           & Sequências de interação confusas ou não claras                          & CPF4               & 15           & 6            & 21             \\
      \hline
      AR1P7           & Usuários não entenderam sentido do conteúdo                             & CPF3               & 15           & 5            & 20             \\
      \hline
      AR1P8           & Organização do conteúdo inconsistente                                   & CPF3               & 12           & 6            & 18             \\
      \hline
      AR1P9           & Funcionalidade não funciona como esperado                               & CPF4               & 6            & 10           & 16             \\
      \hline
      AR1P10          & Funcionalidades dos botões confusas ou não claras                       & CPF1               & 15           & -            & 15             \\
      \hline
      AR1P11          & Expectativa de funcionalidade que não existe                            & CPF4               & 10           & 5            & 15             \\
      \hline
      AR1P12          & Sem alternativa textual                                                 & CPF5               & 14           & -            & 14             \\
      \hline
      AR1P13          & Sistema muito lento                                                     & CPF2               & -            & 11           & 11             \\
      \hline
      AR1P14          & Significado no conteúdo está perdido                                    & CPF3               & 6            & 4            & 10             \\
      % \hline
    \end{tabular}
    \legend{Fonte: \citeonline{Christoph2020}.}
  \end{center}
\end{table}

Os resultados do estudo mostraram que 36 tipos de problemas foram encontrados somente pelos usuários, 11 somente pelas ferramentas e 3 por ambos os métodos.
Evidenciando assim a necessidade de utilização de mais de um método para identificação dos problemas de acessibilidade.
Além disso, o estudo mostrou a importância da utilização dessas ferramentas automatizadas, visto que parte significativa dos problemas podem ser identificados ainda no processo de desenvolvimento, reduzindo o esforço e, consequentemente, o custo para solucioná-los.

% ---
\subsection{\emph{Can Everyone use my app? An Empirical Study on Accessibility in Android Apps}}
% ---

Esse trabalho realizou um estudo piloto no qual foi observado que desenvolvedores de aplicativos móveis raramente utilizam as APIs de Acessibilidade e que o uso de descrições alternativas para elementos de \emph{interface} também é limitado.
Assim, visando entender a perspectiva desses desenvolvedores, o estudo também realizou uma investigação de postagens no \emph{Stack Overflow}, identificando os aspectos de acessibilidade que os desenvolvedores implementavam e os que experienciavam dificuldades.

O estudo investigou aspectos de acessibilidade no geral, baseado em 336 discussões de desenvolvedores Android no \emph{Stack Overflow}, sendo 159 dessas sobre acessibilidade à DV\@.
Dessas 159 discussões, os principais aspectos discutidos foram sobre \emph{feedbacks} sonoros e legibilidade (114 e 24 postagens, respectivamente) como mostra a \autoref{tab-acc-asp-sta-flow}.

\begin{table}[htb]
  \begin{center}
    \ABNTEXfontereduzida
    \caption{Aspectos de acessibilidade à DV discutidos por \emph{devs Android} no \emph{Stack Overflow}.}
    \label{tab-acc-asp-sta-flow}
    \begin{tabular}{p{1.2cm}|p{7.0cm}|p{3.8cm}}
      %\hline
      \textbf{Código} & \textbf{Aspecto}                       & \textbf{Categoria}       \\
      \hline
      AR2P1           & Alertas de acessibilidade              & \emph{Feedbacks} sonoros \\
      \hline
      AR2P2           & Ampliação da tela                      & Legibilidade             \\
      \hline
      AR2P3           & Aspectos não funcionais                & \emph{Feedbacks} sonoros \\
      \hline
      AR2P4           & Consciência de contexto                & \emph{Feedbacks} sonoros \\
      \hline
      AR2P5           & Conteúdos, ações e gestos customizados & \emph{Feedbacks} sonoros \\
      \hline
      AR2P6           & \emph{Frameworks} de terceiros         & \emph{Feedbacks} sonoros \\
      \hline
      AR2P7           & \emph{Mobile web apps}                 & \emph{Feedbacks} sonoros \\
      \hline
      AR2P8           & Problemas com serviços                 & \emph{Feedbacks} sonoros \\
      \hline
      AR2P9           & Sons e vibrações                       & \emph{Feedbacks} sonoros \\
      \hline
      AR2P10          & Suporte à \emph{Braille}               & Teclados alternativos    \\
      \hline
      AR2P11          & Tamanho de fonte                       & Legibilidade             \\
      \hline
      AR2P12          & Teclado customizado                    & Teclados alternativos    \\
      \hline
      AR2P13          & Transformações de cores                & Transformações de cores  \\
      % \hline
    \end{tabular}
    \legend{Fonte: \citeonline{Vendome201941}.}
  \end{center}
\end{table}

No estudo piloto, o trabalho de \citeonline{Vendome201941} analisou 13.817 \emph{apps Android} de código aberto, descobrindo que cerca de 50\% deles tinham descrições alternativas para todos os elementos, enquanto cerca de 37\% não tinha nenhuma.
Além disso, o artigo apontou que apenas cerca de 2\% desses \emph{apps} utilizavam alguma API de acessibilidade no projeto.

% ---
\subsection{\emph{Effect of UX Design Guideline on the information accessibility for the visually impaired in the mobile health apps}}
% ---

Acessibilidade de informações visuais para DV raramente é considerada ao projetar aplicações móveis para saúde \cite{Kim20191103}.
O artigo propõe um guia de diretrizes de acessibilidade à DV, chamado UXDG (\emph{UX Design Guideline}), para resolver esse problema.
120 \emph{apps} na área de saúde foram analisados quanto à taxa de conformidade com o guia.

A \autoref{tab-acc-dir-uxd-1} lista as diretrizes do UXDG de acordo com as categorias.
Na análise dos 120 \emph{apps}, a média da taxa de conformidade com o guia foi de 39,24\%, com a diretriz AR3D7 apresentando
a maior taxa, com 71,67\%, enquanto a AR3D9 apresentou a menor, com 5\%.

\begin{table}[htb]
  \begin{center}
    \ABNTEXfontereduzida
    \caption{Diretrizes do UXDG por categoria.}
    \label{tab-acc-dir-uxd-1}
    \begin{tabular}{p{1.2cm}|p{8.8cm}|p{4.5cm}}
      %\hline
      \textbf{Código} & \textbf{Diretriz}                                                   & \textbf{Categoria}             \\
      \hline
      AR3D1           & Destacar as mídias que disparam ação                                & Aquisição de informação        \\
      \hline
      AR3D2           & Destacar as principais imagens que o usuário pode acessar           & Aquisição de informação        \\
      \hline
      AR3D3           & Navegação intuitiva                                                 & Acessibilidade dos dados       \\
      \hline
      AR3D4           & Posicionar a caixa de pesquisa sempre no mesmo local                & Busca de dados                 \\
      \hline
      AR3D5           & Posicionar resultados de buscas logo após a caixa de texto          & Busca de dados                 \\
      \hline
      AR3D6           & Reconhecimento de voz para entrada de texto                         & Busca de dados                 \\
      \hline
      AR3D7           & Resposta intuitiva do \emph{menu} de acordo com intenção do usuário & Acessibilidade dos dados       \\
      \hline
      AR3D8           & Suporte à esquemas de cores alternativos                            & Melhora na exposição dos dados \\
      \hline
      AR3D9           & Suporte de \emph{zoom in/out} para os principais conteúdos          & Melhora na exposição dos dados \\
      \hline
      AR3D10          & Suporte para outros métodos entrada além do toque                   & Acessibilidade dos dados       \\
      \hline
      AR3D11          & Uso de fontes com alta legibilidade                                 & Aquisição de informação        \\
      % \hline
    \end{tabular}
    \legend{Fonte: \citeonline{Kim20191103}.}
  \end{center}
\end{table}

O estudo realizou testes, conduzidos com 23 PDV e 23 sem DV, comparando \emph{apps} selecionados da área da saúde antes e depois da aplicação do UXDG\@.
Os resultados apontam que houve um aumento na velocidade de reconhecimento das informações depois de aplicar as diretrizes.
De acordo com o experimento, esse aumento aconteceu tanto para usuários com DV, aumento de 13,68\%, quanto para os sem, de 32,41\%.

% ---
\subsection{\emph{Mobile Device Accessibility for the Visually Impaired: Problems Mapping and Empirical Study of Touch Screen Gestures}}
% ---

Esse artigo, mediante um MSL, apresenta os problemas de acessibilidade enfrentados na utilização de dispositivos móveis por PDV encontrados na literatura.
A \autoref{tab-cat-pro-4} mostra, como categorias, 6 dos 7 grupos de problemas identificados no estudo,
desconsiderando o de ``borda não sensível ao toque'', visto que é um problema relativo aos dispositivos físicos.

\begin{table}[htb]
  \begin{center}
    \ABNTEXfontereduzida
    \caption{Categorias dos problemas mapeados na literatura.}
    \label{tab-cat-pro-4}
    \begin{tabular}{p{2.0cm}|p{5.0cm}}
      %\hline
      \textbf{Código} & \textbf{Categoria}   \\
      \hline
      CPM1            & Botões               \\
      \hline
      CPM2            & Comandos de voz      \\
      \hline
      CPM3            & Entrada de dados     \\
      \hline
      CPM4            & Interação por gestos \\
      \hline
      CPM5            & Leitor de tela       \\
      \hline
      CPM6            & Retorno ao usuário   \\
      % \hline
    \end{tabular}
    \legend{Fonte: \citeonline{Damaceno2016}.}
  \end{center}
\end{table}

Na \autoref{tab-pro-1-2-6} são listados os problemas relacionados à botões (CPM1), comandos de voz (CPM2) e retorno do usuário (CPM6), e o número de citações, que corresponde ao número de estudos
onde o problema foi identificado. Sendo que os problemas relacionados aos botões físicos dos dispositivos foram desconsiderados, por estarem fora do controle da aplicação.

\begin{table}[htb]
  \begin{center}
    \ABNTEXfontereduzida
    \caption{Problemas relacionados às categorias CPM1, CPM2 e CPM6.}
    \label{tab-pro-1-2-6}
    \begin{tabular}{p{1.2cm}|p{10.0cm}|p{1.4cm}|p{1.4cm}}
      %\hline
      \textbf{Código} & \textbf{Problema}                                                                               & \textbf{Categoria} & \textbf{Citações} \\
      \hline
      AR4P1           & A grande proximidade entre os botões virtuais dificulta a interação                             & CPM1               & 1                 \\
      \hline
      AR4P2           & Os botões virtuais acarretam menor sensibilidade tátil                                          & CPM1               & 1                 \\
      \hline
      AR4P3           & Apenas um comando de voz é reconhecido por vez                                                  & CPM2               & 2                 \\
      \hline
      AR4P4           & Há baixa privacidade ao emitir comandos de voz                                                  & CPM2               & 1                 \\
      \hline
      AR4P5           & Há diminuição do desempenho do reconhecimento em condições de ruído                             & CPM2               & 1                 \\
      \hline
      AR4P6           & Há diminuição do desempenho do reconhecimento devido à entonação e à acentuação                 & CPM2               & 1                 \\
      \hline
      AR4P7           & Há dificuldade para ativar comando de voz                                                       & CPM2               & 1                 \\
      \hline
      AR4P8           & Há necessidade de mentalizar instrução por voz, aumentando carga de memória do indivíduo        & CPM2               & 1                 \\
      \hline
      AR4P9           & O reconhecimento de voz funciona apenas em alguns aplicativos                                   & CPM2               & 1                 \\
      \hline
      AR4P10          & O uso de comandos de voz é computacionalmente custoso                                           & CPM2               & 1                 \\
      \hline
      AR4P11          & Há ausência de retorno ao usuário, ao interagir com alguns elementos de interface               & CPM6               & 1                 \\
      \hline
      AR4P12          & Há dificuldade para compreender diferentes padrões vibratórios                                  & CPM6               & 1                 \\
      \hline
      AR4P13          & Há dificuldade para compreender a orientação da interface, utilizando apenas o retorno auditivo & CPM6               & 1                 \\
      \hline
      AR4P14          & Retorno auditivo é prejudicado em ambientes ruidosos                                            & CPM6               & 2                 \\
      \hline
      AR4P15          & Usar apenas o retorno auditivo não é o suficiente para a interação                              & CPM6               & 1                 \\
      % \hline
    \end{tabular}
    \legend{Fonte: \citeonline{Damaceno2016}.}
  \end{center}
\end{table}

\newpage

A \autoref{tab-pro-ent-dad-1} mostra os problemas relacionados à entrada de dados (CPM3) com o número de citações para cada problema.
Os problemas que mencionavam teclado físico de dispositivos móveis foram desconsiderados, pois a aplicação a ser desenvolvida suporta apenas \emph{smartphones}.

\begin{table}[htb]
  \begin{center}
    \ABNTEXfontereduzida
    \caption{Problemas relacionados à entrada de dados (CPM3).}
    \label{tab-pro-ent-dad-1}
    \begin{tabular}{p{1.2cm}|p{12.0cm}|p{1.3cm}}
      %\hline
      \textbf{Código} & \textbf{Problema}                                                                                                                  & \textbf{Citações} \\
      \hline
      AR4P16          & A digitação de textos é lenta em teclados QWERTY virtuais                                                                          & 2                 \\
      \hline
      AR4P17          & As teclas mais distantes das bordas são mais difíceis de encontrar do que as mais próximas das bordas, em teclados virtuais QWERTY & 1                 \\
      \hline
      AR4P18          & É preciso conhecer previamente Braille para ter bom desempenho de digitação utilizando esta modalidade                             & 2                 \\
      \hline
      AR4P19          & É preciso trocar o modo do teclado virtual, para acessar determinados caracteres                                                   & 1                 \\
      \hline
      AR4P20          & Há ausência de marca tátil para o número 5, no teclado numérico virtual, e para as letras “F” e “J” no teclado QWERTY virtual      & 2                 \\
      \hline
      AR4P21          & Há erros ao corrigir caracteres digitados equivocadamente, substituindo por fonemas semelhantes, em teclados virtuais              & 1                 \\
      \hline
      AR4P22          & Há erros de omissão de caracteres, faltando um ou mais ao digitar palavras em teclados virtuais                                    & 1                 \\
      \hline
      AR4P23          & Há necessidade de confirmação de cada caractere digitado em teclados virtuais                                                      & 1                 \\
      \hline
      AR4P24          & Há necessidade de navegar pelo teclado virtual para localizar os caracteres desejados                                              & 1                 \\
      \hline
      AR4P25          & Há um segundo de espera para entrar com cada tecla em teclados virtuais                                                            & 1                 \\
      \hline
      AR4P26          & O teclado numérico virtual é denso dificultando, a interação                                                                       & 1                 \\
    \end{tabular}
    \legend{Fonte: \citeonline{Damaceno2016}.}
  \end{center}
\end{table}

A \autoref{tab-pro-int-ges-1} lista os problemas relacionados à interação por gestos (CPM4) com o número de citações para cada problema encontrado.

\begin{table}[htb]
  \begin{center}
    \ABNTEXfontereduzida
    \caption{Problemas relacionados à interação por gestos (CPM4).}
    \label{tab-pro-int-ges-1}
    \begin{tabular}{p{1.2cm}|p{12.0cm}|p{1.2cm}}
      %\hline
      \textbf{Código} & \textbf{Problema}                                                                  & \textbf{Citações} \\
      \hline
      AR4P27          & Baixa flexibilidade de ângulo e velocidade dos gestos dificultam o reconhecimento  & 1                 \\
      \hline
      AR4P28          & Gestos com forma da letra “L” são difíceis de fazer                                & 2                 \\
      \hline
      AR4P29          & Gestos com formas geométricas fechadas (círculo e triângulo) são difíceis de fazer & 1                 \\
      \hline
      AR4P30          & Gestos com formas geométricas são lentos de se fazer                               & 1                 \\
      \hline
      AR4P31          & Conflito na desambiguação entre dois toques com um dedo e três toques com um dedo  & 1                 \\
      \hline
      AR4P32          & Dificuldade para fazer gestos estando em movimento                                 & 1                 \\
      \hline
      AR4P33          & Dificuldade para fazer gestos próximos à barra superior de sistemas                & 1                 \\
      \hline
      AR4P34          & Dificuldade para fazer gestos representados por símbolos                           & 2                 \\
      \hline
      AR4P35          & Dificuldade para fazer o gesto de dois toques com um dedo                          & 1                 \\
      \hline
      AR4P36          & Dificuldade para se localizar na tela para realizar gestos                         & 1                 \\
      \hline
      AR4P37          & Erros na detecção de gestos multitoque                                             & 1                 \\
      \hline
      AR4P38          & Falha de interpretação de gestos em geral, pelo sistema                            & 4                 \\
      \hline
      AR4P39          & Mudança indevida de foco ao tentar fazer o gesto dois toques com um dedo           & 1                 \\
      \hline
      AR4P40          & Não é possível alterar mapeamento dos gestos às funções do sistema                 & 1                 \\
      \hline
      AR4P41          & Não há consistência de gestos entre diferentes sistemas                            & 1                 \\
      \hline
      AR4P42          & Não há gestos que acionam as principais funções do sistema                         & 1                 \\
      \hline
      AR4P43          & O toque acidental na tela, com outro dedo, prejudica o reconhecimento de gestos    & 1                 \\
      \hline
      AR4P44          & Os manuais de explicação de como fazer gestos de toque não são eficientes          & 3                 \\
      \hline
      AR4P45          & Conflito entre do aplicativo gestos e os do leitor de tela do sistema              & 1                 \\
      % \hline
    \end{tabular}
    \legend{Fonte: \citeonline{Damaceno2016}.}
  \end{center}
\end{table}

\newpage

Por fim, são listados, na \autoref{tab-pro-lei-tel-1}, os problemas relacionados a leitores de tela (CPM5) com o número de citações.

\begin{table}[htb]
  \begin{center}
    \ABNTEXfontereduzida
    \caption{Problemas relacionados a leitores de tela (CPM5).}
    \label{tab-pro-lei-tel-1}
    \begin{tabular}{p{1.2cm}|p{12.0cm}|p{1.4cm}}
      %\hline
      \textbf{Código} & \textbf{Problema}                                                                                & \textbf{Citações} \\
      \hline
      AR4P46          & A leitura é linear, demorando para se ter noção global da interface                              & 2                 \\
      \hline
      AR4P47          & A pronúncia de algumas palavras é problemática                                                   & 1                 \\
      \hline
      AR4P48          & A voz do leitor de tela é artificial                                                             & 1                 \\
      \hline
      AR4P49          & Alguns elementos de interface não são lidos                                                      & 3                 \\
      \hline
      AR4P50          & Há baixa familiaridade com o leitor de tela de dispositivos móveis                               & 1                 \\
      \hline
      AR4P51          & Há conflito ao usar o leitor de tela do sistema em conjunto com o leitor embutido em aplicativos & 2                 \\
      \hline
      AR4P52          & Há desconforto ao ouvir o leitor de tela em ambientes ruidosos                                   & 2                 \\
      \hline
      AR4P53          & Há leitura de apenas o que está em foco                                                          & 1                 \\
      \hline
      AR4P54          & Não há controle de velocidade de leitura                                                         & 2                 \\
      \hline
      AR4P55          & Não há um botão para interromper a leitura imediatamente                                         & 1                 \\
      \hline
      AR4P56          & O foco do leitor de tela muda indevidamente                                                      & 2                 \\
      \hline
      AR4P57          & O foco do leitor de tela não possui uma ordem de navegação lógica                                & 2                 \\
      \hline
      AR4P58          & O leitor de tela é lento                                                                         & 1                 \\
      \hline
      AR4P59          & O texto lido é, por vezes, inadequado                                                            & 1                 \\
      % \hline
    \end{tabular}
    \legend{Fonte: \citeonline{Damaceno2016}.}
  \end{center}
\end{table}

% ---
\subsection{\emph{Observation Based Analysis on the Use of Mobile Applications for Visually Impaired Users}}
% ---

O estudo realizou uma análise, envolvendo 5 PDV, com o objetivo de validar se a falta dos requisitos de acessibilidade levantados em um trabalho anterior realmente impactavam na utilização de \emph{apps} móveis por PDV.

\begin{table}[htb]
  \begin{center}
    \ABNTEXfontereduzida
    \caption{Categorias dos requisitos encontrados.}
    \label{tab-cat-req-enc-5}
    \begin{tabular}{p{2.0cm}|p{5.0cm}}
      %\hline
      \textbf{Código} & \textbf{Categoria}                \\
      \hline
      CRED1           & \emph{Feedbacks} audíveis         \\
      \hline
      CRED2           & Adaptação das informações visuais \\
      \hline
      CRED3           & Navegação                         \\
      % \hline
    \end{tabular}
    \legend{Fonte: \citeonline{Siebra2016}.}
  \end{center}
\end{table}

Os requisitos foram divididos em 3 categorias, como mostra a \autoref{tab-cat-req-enc-5}.
Baseados na análise dos resultados, o estudo qualificou os requisitos em 3 níveis (Essencial, Desejável e Não observado).
Como os requisitos ``não observados'', de acordo com o artigo, não foram mencionados pelos participantes dos testes, apenas os requisitos essenciais e desejáveis são listados na \autoref{tab-req-ess-des-1}.
Somente um requisito foi classificado como desejável pelo estudo, o AR5R7, o restante foi classificado como essencial.

\begin{table}[htb]
  \begin{center}
    \ABNTEXfontereduzida
    \caption{Requisitos essenciais e desejáveis focados em DV.}
    \label{tab-req-ess-des-1}
    \begin{tabular}{p{1.2cm}|p{12.1cm}|p{1.4cm}}
      %\hline
      \textbf{Código} & \textbf{Requisito}                                                                                               & \textbf{Categoria} \\
      \hline
      AR5R1           & O nome do caractere que está sendo digitado deve ser ouvido                                                      & CRED1              \\
      \hline
      AR5R2           & Nomes de elementos e imagens na tela devem ser ouvidos ao serem tocados ou selecionados                          & CRED1              \\
      \hline
      AR5R3           & \emph{Feedback} de ações/interações devem ser claros e fornecidos de forma tátil, voz ou eventos sonoros         & CRED1              \\
      \hline
      AR5R4           & Estratégias para o uso de leitores de tela (ex.\@: atalhos para navegar na tela de forma mais eficiente)         & CRED1              \\
      \hline
      AR5R5           & Prover uma chave ``home'' tátil de acesso fácil e rápido para que um usuário possa retornar a um lugar conhecido & CRED2              \\
      \hline
      AR5R6           & Prover documentação em formatos alternativos, utilizando fontes grandes                                          & CRED3              \\
      \hline
      AR5R7           & Permitir customizações pelo usuário e evitar que essas preferências sejam perdidas                               & CRED3              \\
      \hline
      AR5R8           & Apresentar amplificador com \emph{zoom} ajustável                                                                & CRED3              \\
      \hline
      AR5R9           & Prover equivalências textuais claras para evitar erros quando os textos são lidos na tela                        & CRED3              \\
      \hline
      AR5R10          & Brilho, contrate e cores ajustáveis                                                                              & CRED3              \\
      \hline
      AR5R11          & Prover alertas informativos por outros canais além do visual (ex.\@: voz)                                        & CRED3              \\
      % \hline
    \end{tabular}
    \legend{Fonte: \citeonline{Siebra2016}.}
  \end{center}
\end{table}

% ---
\subsection{\emph{Prioritization of mobile accessibility guidelines for visual impaired users}}
% ---

O artigo apresenta uma proposta de priorização de diretrizes de acessibilidade que resultaram de estudos anteriores.
Essas diretrizes foram baseadas no eMAG, porém diretrizes como as da BCC (\emph{BBC Mobile Accessibility Guidelines}) e recomendações da plataforma Android também foram consideradas.
Para criação do \emph{ranking}, o estudo utilizou um questionário que foi respondido 103 vezes, sendo 66 dessas respostas de PDV, nas quais a análise se concentrou.

O estudo dividiu as diretrizes em 6 categorias que podem ser visualizadas na \autoref{tab-cat-dir-acc-5}.

\begin{table}[htb]
  \begin{center}
    \ABNTEXfontereduzida
    \caption{Categorias das diretrizes de acessibilidade \emph{mobile} baseadas no eMAG.}
    \label{tab-cat-dir-acc-5}
    \begin{tabular}{p{1.5cm}|p{4.5cm}}
      %\hline
      \textbf{Código} & \textbf{Categoria}         \\
      \hline
      AR6CE           & Estrutura                  \\
      \hline
      AR6CC           & Comportamento              \\
      \hline
      AR6CCI          & Conteúdo/Informação        \\
      \hline
      AR6CAD          & Apresentação/\emph{Design} \\
      \hline
      AR6CM           & Multimídia                 \\
      \hline
      AR6CF           & Formulários                \\
      % \hline
    \end{tabular}
    \legend{Fonte: \citeonline{Quispe2020}.}
  \end{center}
\end{table}

O estudo considerou a priorização para 4 grupos diferentes, baseados no tipo de DV (baixa visão, visão parcial e os 2 tipos de cegueira: legal e total).
E os resultados mostraram que existiam diferenças notáveis na percepção das diretrizes entre os grupos.

\begin{quadro}[htb!]
  \begin{center}
    \ABNTEXfontereduzida
    \caption{\label{qua-pri-acc-gui}Priorização de diretrizes de acessibilidade para usuários com DV.}
    \begin{tabular}{|m{0.5cm} | m{2.4cm} | m{2.4cm} | m{2.8cm} | m{3.0cm} | m{2.4cm}|}
      %\hline
      \hline
      \textbf{Id} & \textbf{Visão parcial} & \textbf{Baixa visão} & \textbf{Cegueira legal}       & \textbf{Cegueira total}              & \textbf{Todas as DV} \\
      \hline
      1           & AR6D7, AR6D28          & AR6D9                & AR6D24, AR6D28                & AR6D8, AR6D22, AR6D23                & AR6D8                \\
      \hline
      2           & AR6D25                 & AR6D11, AR6D28       & AR6D7, AR6D11, AR6D22, AR6D25 & AR6D1, AR6D6, AR6D25, AR6D28         & AR6D25               \\
      \hline
      3           & AR6D24                 & AR6D25               & AR6D1, AR6D27                 & AR6D7, AR6D9, AR6D15, AR6D24, AR6D27 & AR6D7, AR6D9, AR6D22 \\
      \hline
      4           & AR6D9, AR6D22          & AR6D7, AR6D22        & AR6D9                         & AR6D11                               & AR6D11, AR6D24       \\
      \hline
      5           & AR6D11                 & AR6D10               & AR6D10                        & AR6D10                               & AR6D27               \\
      \hline
      6           & AR6D10, AR6D27         & AR6D15               & AR6D8, AR6D23                 & \-                                   & AR6D1                \\
      \hline
      7           & AR6D8, AR6D15          & AR6D1, AR6D6, AR6D27 & AR6D6                         & \-                                   & AR6D8                \\
      \hline
      8           & AR6D6                  & AR6D8                & AR6D15                        & \-                                   & AR6D6, AR6D23        \\
      \hline
      9           & AR6D1                  & AR6D23, AR6D24       & \-                            & \-                                   & AR6D10, AR6D15       \\
      \hline
      10          & AR6D23                 & \-                   & \-                            & \-                                   & \-                   \\
      \hline
      % \hline
    \end{tabular}
    \legend{Fonte: \citeonline{Quispe2020}.}
  \end{center}
\end{quadro}

Assim, a partir desses resultados, o trabalho relacionou as diretrizes com as percepções de cada grupo
e criou a lista de priorização que pode ser vista no \autoref{qua-pri-acc-gui}.
No qual a coluna ``Id'' informa a ordem de priorização e os códigos que estão nas demais são listados na \autoref{tab-dir-acc-mob-1} junto com as diretrizes.

\begin{table}[htb]
  \begin{center}
    \ABNTEXfontereduzida
    \caption{Diretrizes de acessibilidade \emph{mobile} baseadas no eMAG.}
    \label{tab-dir-acc-mob-1}
    \begin{tabular}{p{1.2cm}|p{12.0cm}|p{1.5cm}}
      %\hline
      \textbf{Código} & \textbf{Diretriz}                                                                                                                            & \textbf{Categoria} \\
      \hline
      AR6D1           & Elementos de tela devem ser organizados de maneira lógica e semântica.                                                                       & AR6CE              \\
      \hline
      AR6D2           & As telas devem apresentar sequência lógica de leitura para navegação entre \emph{links}, controles de formulário e outros elementos.         & AR6CE              \\
      \hline
      AR6D3           & \emph{Links} na tela devem ser organizados para evitar confusão.                                                                             & AR6CE              \\
      \hline
      AR6D4           & Informações devem ser divididas em grupos específicos para facilitar a procura e leitura dos conteúdos.                                      & AR6CE              \\
      \hline
      AR6D5           & Usuários devem ser informados se \emph{links} abrem novas telas para poderem decidir se querem ou não sair da tela atual.                    & AR6CE              \\
      \hline
      AR6D6           & Todas as funcionalidades na tela devem estar disponíveis a partir do teclado.                                                                & AR6CC              \\
      \hline
      AR6D7           & Todos os elementos de \emph{interface} na tela devem ser acessíveis.                                                                         & AR6CC              \\
      \hline
      AR6D8           & Redirecionamento automático de telas não deve acontecer.                                                                                     & AR6CC              \\
      \hline
      AR6D9           & Em telas com limite de tempo, deve haver opções para desligar ou ajustar o tempo.                                                            & AR6CC              \\
      \hline
      AR6D10          & Não deve haver efeitos visuais piscantes, intermitentes ou cintilantes na tela.                                                              & AR6CC              \\
      \hline
      AR6D11          & Conteúdos animados não devem iniciar automaticamente.                                                                                        & AR6CC              \\
      \hline
      AR6D12          & A linguagem utilizada na tela deve ser especificada.                                                                                         & AR6CCI             \\
      \hline
      AR6D13          & Mudanças na linguagem dos conteúdos sempre devem ser especificadas.                                                                          & AR6CCI             \\
      \hline
      AR6D14          & Títulos de telas devem ser descritivos, informativos e representativos com relação ao conteúdo principal.                                    & AR6CCI             \\
      \hline
      AR6D15          & Deve haver algum mecanismo para indicar ao usuário onde ele está no momento, no conjunto de telas.                                           & AR6CCI             \\
      \hline
      AR6D16          & Alvos de \emph{links} devem ser identificados claramente, incluindo informações sobre se estão funcionando ou se direcionam para outra tela. & AR6CCI             \\
      \hline
      AR6D17          & Todas as imagens devem possuir descrição textual.                                                                                            & AR6CCI             \\
      \hline
      AR6D18          & Documentos em formatos acessíveis devem estar disponíveis.                                                                                   & AR6CCI             \\
      \hline
      AR6D19          & Quando uma tabela é utilizada na tela, título e sumário apropriados devem ser fornecidos.                                                    & AR6CCI             \\
      \hline
      AR6D20          & Os textos nas telas devem ser fáceis de ler e entender.                                                                                      & AR6CCI             \\
      \hline
      AR6D21          & Totos as siglas, abreviações e palavras incomuns na tela devem possuir explicação.                                                           & AR6CCI             \\
      \hline
      AR6D22          & Deve haver uma taxa miníma de contraste entre as cores de fundo e as de frente.                                                              & AR6CAD             \\
      \hline
      AR6D23          & Características sensoriais (ex.\: cores, formas e sons) não podem ser o único significado para distinguir elementos de tela.                 & AR6CAD             \\
      \hline
      AR6D24          & O elemento ou área em foco deve ser evidente visualmente.                                                                                    & AR6CAD             \\
      \hline
      AR6D25          & Vídeos que não incluem áudio devem fornecer alternativas como legendas.                                                                      & AR6CM              \\
      \hline
      AR6D26          & Deve haver alternativas a conteúdo de áudio (ex.\: transcrição ou linguagem de sinais).                                                      & AR6CM              \\
      \hline
      AR6D27          & Conteúdos visuais que não estão disponíveis como áudio devem ser descritos.                                                                  & AR6CM              \\
      \hline
      AR6D28          & Devem haver mecanismos para controlar áudios da aplicação.                                                                                   & AR6CM              \\
      \hline
      AR6D29          & Devem haver mecanismos para controlar animações que iniciam automaticamente.                                                                 & AR6CM              \\
      \hline
      AR6D30          & Botões de imagem ou conteúdos de áudio em formulários devem possuir alternativas textuais.                                                   & AR6CF              \\
      \hline
      AR6D31          & Todos os campos do formulário devem ser identificados.                                                                                       & AR6CF              \\
      \hline
      AR6D32          & Uma ordem lógica na navegação pelo formulário deve ser garantida.                                                                            & AR6CF              \\
      \hline
      AR6D33          & Não devem haver mudanças automáticas quando um elemento do formulário é focado, para não confundir ou desorientar o usuário.                 & AR6CF              \\
      \hline
      AR6D34          & Formulários devem possuir instruções de preenchimento.                                                                                       & AR6CF              \\
      \hline
      AR6D35          & Erros de entrada devem sempre ser descritos e as submissões de dados confirmadas.                                                            & AR6CF              \\
      % \hline
    \end{tabular}
    \legend{Fonte: \citeonline{Quispe2020}.}
  \end{center}
\end{table}
