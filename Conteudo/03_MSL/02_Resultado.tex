% !TeX root = ..\Modelo-TCC-DCOMP.tex
\newpage{}

\section{Resultados Encontrados}

Nesta seção são apresentados os resumos com as principais características relacionadas ao tema deste trabalho,
incluindo um quadro com características dos aplicativos desenvolvidos, identificadas nos artigos selecionados
na fase de extração, visando encontrar respostas para as questões levantadas na definição do protocolo de MSL\@.

\subsection{\emph{A Mobile Educational Game Accessible to All, Including Screen Reading Users on a Touch-Screen Device}}

O estudo realizado por \citeonline{Leporini2017} teve como objetivo levantar informações e possíveis soluções para as dificuldades levantadas por um grupo composto por 6 pessoas cegas ao responder questões de tarefas interativas.
E investigou, por meio de tarefas interativas como exercícios e questionários, a acessibilidade e usabilidade de gestos e leitores de tela em dispositivos móveis com \emph{touch-screen}.

No artigo é apresentado um \emph{game} que envolveu duas pessoas cegas com experiência na utilização de \emph{smartphones} na fase inicial do planejamento do protótipo.
O jogo funciona como se fosse um ``sistema solar'' com oito planetas, com cada planeta representando um conjunto de questões e exercícios.
O jogador recebe determinada pontuação cada vez que joga de acordo com os acertos e erros.
As principais funcionalidades do \emph{app} relativas à acessibilidade identificadas foram:

\begin{enumerate}
  \item Contraste de cor para garantir diferentes níveis de acessibilidade;
  \item Apresentações de conteúdos de forma auditiva e visual;
  \item Interação via gestos ou toques;
  \item Suporte auditivo com descrições dos elementos.
\end{enumerate}

Por meio da avaliação desse protótipo, por cegos, o estudo investigou o suporte de acessibilidade \emph{mobile} multiplataforma do conjunto de especificações
técnicas WAI-Aria\footnote{A Iniciativa de Acessibilidade na Web (WAI, do inglês \emph{Web Accessibility Initiative})
  define uma maneira de tonar aplicações e conteúdos web mais acessíveis às pessoas com deficiência por meio do conjunto
  Aplicações para a Internet Ricas em Acessibilidade (ARIA, do inglês \emph{Accessible Rich Internet Applications Suite}).},
observando problemas na detecção de elementos devido às suas posições na tela e conteúdos difíceis
de identificar na interação com leitores de tela. Notando também que houve alguma dificuldade por
conta de gestos implementados no \emph{app} diferirem dos habituais utilizados pelos usuários no \emph{VoiceOver} do iOS.

Apesar dos problemas encontrados, o artigo aponta que o \emph{feedback} foi positivo e os resultados mostraram que os exercícios puderam ser realizados facilmente, por pessoas cegas,
fazendo uso de simples gestos com auxilio dos leitores de tela.

O \autoref{qua-car-am1} mostra as principais características do desenvolvimento do aplicativo apresentado nesse trabalho.

\begin{quadro}[htb!]
  \caption{\label{qua-car-am1}Características do Desenvolvimento do Aplicativo do AM1.}
  \begin{tabular}{|c|c|}
    \hline
    \textbf{Tecnologias Utilizadas} & Cordova Framework \\ \hline
    \textbf{Plataformas}            & Android e iOS     \\ \hline
    \textbf{Público-alvo}           & PDV               \\
    \hline
  \end{tabular}
  \legend{Fonte: \citeonline{Leporini2017}.}
\end{quadro}

\subsection{\emph{A Model-Driven Approach to Cross-Platform Development of Accessible Business Apps}}

Um procedimento comum no processo de desenvolvimento de \emph{software} é considerar a acessibilidade para PDV apenas na etapa final.
Além disso, muitos desenvolvedores não estão cientes de técnicas de software para atender esse grupo, pois o domínio de aplicativos
móveis multiplataforma tem recebido uma atenção limitada por pesquisadores. Foi nesse sentido, que o estudo de \citeonline{Christoph2020}
buscou identificar desafios, requisitos e soluções técnicas de acessibilidade, selecionando 28 requisitos a respeito de acessibilidade para
aplicações móveis por meio de uma RSL\@.

O artigo apresenta uma abordagem orientada a modelos que integra conceitos de acessibilidade no desenvolvimento de aplicações móveis multiplataforma em conjunto com protótipos
acessíveis à PDV\@. Com isso, apresenta também uma aplicação com foco em fornecer informações sobre chuvas fortes e inundações, na qual
os usuários podem ter uma visão de eventos de inundações próximos e compartilhar novos incidentes.

O estudo comparou uma versão da aplicação desenvolvida nativamente que necessitou de 3,400 linhas de código \emph{Java} e 3,200 linhas de código \emph{XML}
(gerado de forma semiautomática) com outra versão, com um conjunto similar de funcionalidades. A nova versão do \emph{app} consistiu em 445 linhas de código \emph{MD²}, \emph{framework}
baseado na abordagem orientada a modelos para desenvolvimento móvel multiplataforma com a linguagem de alto nível \emph{Xtend}\footnote{
  Linguagem de programação de alto nível para a JVM (\emph{Java Virtual Machine}) com foco em injeção de dependências e geração de código Java. \url{https://www.eclipse.org/xtend/}}.
Principais funcionalidades sobre acessibilidade identificadas:

\begin{enumerate}
  \item Adaptação da \emph{interface} de acordo com as necessidades do usuário;
  \item Integração com os leitores de tela graças ao fornecimento de descrições em texto para elementos não textuais;
  \item Personalização do contorno de foco do \emph{TalkBack}.
\end{enumerate}

Segundo o artigo, o estudo de caso mostrou que \emph{apps} acessíveis podem ser gerados a partir do modelo de alto nível \emph{MD²}, implementando as técnicas de integração adequadas em
cada ponto. Embora o autor afirme isso, o estudo também deixa claro que ainda havia uma pendência de validação centrada no usuário, visto que o trabalho não implementou todas as técnicas
e a solução proposta não foi testada com PDV\@.

As tecnologias utilizadas no desenvolvimento do aplicativo desse estudo, plataformas e público-alvo podem ser vistos
no \autoref{qua-car-am2}.

\begin{quadro}[htb!]
  \caption{\label{qua-car-am2}Características do Desenvolvimento do Aplicativo do AM2.}
  \begin{tabular}{|c|c|}
    \hline
    \textbf{Tecnologias Utilizadas} & Xtend, Java e Eclipse                          \\ \hline
    \textbf{Plataformas}            & Android e iOS                                  \\ \hline
    \textbf{Público-alvo}           & PDV com interesse em eventos climáticos locais \\
    \hline
  \end{tabular}
  \legend{Fonte: \citeonline{Christoph2020}.}
\end{quadro}

\subsection{\emph{An Accessible Roller Coaster Simulator for Touchscreen Devices: An Educational Game for the Visually Impaired}}

O trabalho de \citeonline{Biase2018} apresenta um \emph{app} simulador de montanha russa, baseado em simuladores educacionais já existentes
e adaptado para \emph{smartphones}, para ser utilizado em disciplinas de Educação Física por pessoas com e sem DV\@.
A aplicação foi desenvolvida para auxiliar no estudo de Energia Mecânica e trás as interações por áudio e tátil como alternativas à visual.
As principais funcionalidades sobre acessibilidade identificadas no \emph{app} foram:

\begin{enumerate}
  \item Os elementos visuais possuem descrições textuais para integração com leitores de tela;
  \item \emph{Feedback} por meio de ``texto para voz'' (TTS, do inglês \emph{text-to-speech}) e vibração ao clicar em determinados elementos
        na tela, mesmo com o modo de acessibilidade desativado;
  \item Efeitos sonoros característicos que ilustram os resultados da simulação ao longo do percurso.
\end{enumerate}

Com taxas de 73\% eficácia, 77\% de eficiência e 66\% satisfação do usuário com relação a aplicação desenvolvida, os testes de usabilidade
demonstraram que as estratégias de interação propostas são viáveis, com grande potencial para serem utilizadas em propósitos educacionais.

Contudo, alguns problemas de acessibilidade afetaram a taxa de satisfação dos usuários, a mantendo em 66\%, tais como
dificuldades em seguir a trilha da montanha com apenas um dedo, não ser possível detectar quando o carro está voltando
no trilho e falha no comando que altera o foco dos elementos, alterando para o elemento errado.

Características do desenvolvimento desse aplicativo, como tecnologias, plataforma e público-alvo são listadas
no \autoref{qua-car-am3}.

\begin{quadro}[htb!]
  \caption{\label{qua-car-am3}Características do Desenvolvimento do Aplicativo do AM3.}
  \begin{tabular}{|c|c|}
    \hline
    \textbf{Tecnologias Utilizadas} & Unity 3D             \\ \hline
    \textbf{Plataformas}            & Android              \\ \hline
    \textbf{Público-alvo}           & Pessoas com e sem DV \\
    \hline
  \end{tabular}
  \legend{Fonte: \citeonline{Biase2018}.}
\end{quadro}

\subsection{\emph{Application for the Configuration and Adaptation of the Android Operating System for the Visually Impaired}}

Apesar das vantagens dos dispositivos móveis, alguns desafios da interação de PDV com os sistemas operacionais (SOs) desses dispositivos precisam ser superados, para que a tecnologia alcance
um número significativo nesse grupo. Assim, o estudo de \citeonline{Oliveira2018} visou planejar e desenvolver uma aplicação que automatize as configurações do SO Android de acordo com
as preferências de acessibilidade de cada PDV, por meio de comandos de voz. O artigo apresenta algumas funcionalidades e técnicas relacionadas a acessibilidade que são listadas a seguir:

\begin{enumerate}
  \item Escala de Usabilidade do Sistema (SUS, do inglês \emph{System Usability Scale}) para avaliação de usabilidade da aplicação;
  \item \emph{SpeechRecognizer} do Android para reconhecimento de voz;
  \item Eurísticas de Usabilidade de Nielsen (do inglês, \emph{Nielsen Usability Heuristics}) para evitar problemas de acessibilidade já mapeados.
\end{enumerate}

Um protótipo foi desenvolvido e mostrou potencial para ser utilizado como ferramenta para PDV, trazendo benefícios com a possibilidade do uso de comando de voz.
Os testes foram realizados com seis voluntárias com DV, sendo duas parcial e quatro total.
Das quais três já possuíam experiência com comandos de voz e apenas duas das seis pessoas já haviam realizado a configuração do dispositivo alguma vez.

Por fim, as voluntárias expressaram avaliações positivas quanto à autonomia, satisfação e usabilidade da aplicação.
E o tempo gasto para realizar as configurações de acessibilidade foi mais curto no \emph{app} desenvolvido que na aplicação padrão do Android.

As principais características do desenvolvimento desse aplicativo são apresentadas no \autoref{qua-car-am4}.

\begin{quadro}[htb!]
  \caption{\label{qua-car-am4}Características do Desenvolvimento do Aplicativo do AM4.}
  \begin{tabular}{|c|c|}
    \hline
    \textbf{Tecnologias Utilizadas} & Android Studio 2.0 \\ \hline
    \textbf{Plataformas}            & Android            \\ \hline
    \textbf{Público-alvo}           & PDV                \\
    \hline
  \end{tabular}
  \legend{Fonte: \citeonline{Oliveira2018}.}
\end{quadro}

\newpage

\subsection{\emph{Blind and visually impaired user interface to solve accessibility problems}}

Este estudo realizou uma RSL e testes em várias aplicações móveis para PDV, e dividiu os problemas encontrados em três categorias: organização, apresentação e comportamento (OAC).
Uma aplicação móvel, chamada ``\emph{Read Master}'', também foi desenvolvida no trabalho de \citeonline{Shera2021285}, incorporando soluções para os principais problemas de OAC\@.

\begin{table}[htb]
  \begin{center}
    \ABNTEXfontereduzida
    \caption{Categorias dos problemas identificados.}
    \label{tab-cat-pro-enc-ar5}
    \begin{tabular}{p{2.0cm}|p{5.0cm}}
      %\hline
      \textbf{Código} & \textbf{Categoria} \\
      \hline
      CRR1            & Apresentação       \\
      \hline
      CRR2            & Comportamento      \\
      \hline
      CRR3            & Organizacional     \\
      % \hline
    \end{tabular}
    \legend{Fonte: \citeonline{Shera2021285}.}
  \end{center}
\end{table}

Na tabela \autoref{tab-pro-enc-ar5} estão listados os problemas identificados pelo estudo.
Os códigos na coluna ``Categoria'' referem-se às categorias da \autoref{tab-cat-pro-enc-ar5}.

\begin{table}[htb]
  \begin{center}
    \ABNTEXfontereduzida
    \caption{Problemas de acessibilidade encontrados por categoria.}
    \label{tab-pro-enc-ar5}
    \begin{tabular}{p{1.2cm}|p{11.5cm}|p{1.5cm}}
      %\hline
      \textbf{Código} & \textbf{Problema}                                                                                 & \textbf{Categoria} \\
      \hline
      AM1P1           & Falta de consistência no \emph{layout} e terminologias                                            & CRR1               \\
      \hline
      AM1P2           & Leitor de tela fornecendo \emph{feedbacks} confusos                                               & CRR1               \\
      \hline
      AM1P3           & Leitor de tela quebrando                                                                          & CRR2               \\
      \hline
      AM1P4           & Ouvir cabeçalhos e títulos das páginas de forma redundante antes de detectar o conteúdo das telas & CRR2               \\
      \hline
      AM1P5           & Conteúdos na tela da aplicação móvel                                                              & CRR3               \\
      \hline
      AM1P6           & Fluxo de tarefas                                                                                  & CRR3               \\
      \hline
      AM1P7           & Não entendimento da sequência natural de leitura                                                  & CRR3               \\
      \hline
      AM1P8           & Não entendimento do fluxo natural de tarefas                                                      & CRR3               \\
      \hline
      AM1P9           & Problemas de navegação                                                                            & CRR3               \\
      \hline
      AM1P10          & Sobrecarga de informações                                                                         & CRR3               \\
      % \hline
    \end{tabular}
    \legend{Fonte: \citeonline{Shera2021285}.}
  \end{center}
\end{table}

O \emph{app} desenvolvido consistiu em duas funcionalidades principais: fornecer informações cientificas e \emph{quizzes} de múltipla escolha.
As principais técnicas e funcionalidades identificadas no estudo para o suporte de acessibilidade foram:

\begin{enumerate}
  \item \emph{SUS} para avaliação de usabilidade da aplicação;
  \item Leitor de tela embutido por meio de TTS\@.
  \item Levantamento e categorização dos principais problemas de acessibilidade em \emph{apps} móveis.
\end{enumerate}

Uma avaliação de usabilidade do \emph{app}, com 56 PDV, foi conduzida e validada com foco na experiência de usuários com DV\@.
Os resultados mostraram que a organização da aplicação estava 100\% efetiva tanto para os usuários cegos quanto para os com DV parcial.
Já com relação à eficiência dos usuários, a dos com DV parcial mostrou-se maior que a dos cegos.

O nível mais alto de satisfação quanto às 3 categorias de problemas avaliados estava na apresentação com
87,62\% para usuários com DV total, enquanto para os com visão parcial estava tanto na organização quanto
na apresentação com 89,21\%.
No geral, o estudo indica que a aplicação reduziu a gravidade dos problemas de OPB, oferecendo alta usabilidade.

Embora o estudo não informe algumas características do processo de desenvolvimento do aplicativo, algumas foram
informadas e estão listadas no \autoref{qua-car-am5}.

\begin{quadro}[htb!]
  \caption{\label{qua-car-am5}Características do Desenvolvimento do Aplicativo do AM5.}
  \begin{tabular}{|c|c|}
    \hline
    \textbf{Tecnologias Utilizadas} & Não informado \\ \hline
    \textbf{Plataformas}            & Android       \\ \hline
    \textbf{Público-alvo}           & PDV           \\
    \hline
  \end{tabular}
  \legend{Fonte: \citeonline{Shera2021285}.}
\end{quadro}

\subsection{\emph{Design and development of a mobile app of drug information for people with visual impairment}}

Esse trabalho foi desenvolvido na Colombia, onde a falta de acesso à informações acessíveis
de rótulos de medicamentos como contraindicações, armazenamento, data de validade e dosagem foi identificada como uma
das principais barreiras no uso de medicamentos por PDV \cite{Amariles2020}.

Nesse contexto, uma aplicação \emph{mobile} chamada \emph{FarmaceuticApp} foi desenvolvida no estudo.
A principal funcionalidade do \emph{app} é a de busca por informações de medicamentos, apresentando-as
ao usuário de forma acessível e podendo ser realizada por vários meios.

As principais técnicas e funcionalidades identificadas, relacionadas à acessibilidade e utilizadas no desenvolvimento dessa solução, foram:

\begin{enumerate}
  \item Tamanho da fonte das letras personalizável;
  \item Vibração e sons para alertar o usuário do resultado da busca;
  \item \emph{Tutorial} com possibilidade de ser visto novamente;
  \item Possibilidade de busca por \emph{barcode} e \emph{qrcode}, foto, comando de voz e texto;
  \item Possibilidade de ativar e desativar o assistente de voz do \emph{app}.
\end{enumerate}

O estudo envolveu 48 PDV, das quais 69\% necessitavam de assistência para o uso de medicamentos e 90\% possuíam celulares, sendo 93\%  deles com o SO Android.
Na avaliação final, 100\% dos usuários disseram utilizariam o \emph{app} e o avaliaram entre 4 e 5 estrelas (bom e muito bom).

As diversas tecnologias utilizadas no desenvolvimento desse aplicativo são mostradas no \autoref{qua-car-am6}
junto com outras características.

\begin{quadro}[htb!]
  \caption{\label{qua-car-am6}Características do Desenvolvimento do Aplicativo do AM6.}
  \begin{tabular}{|c|c|}
    \hline
    \textbf{Tecnologias Utilizadas} & Java, Accessibility Scanner App e Firebase Test Lab   \\ \hline
    \textbf{Plataformas}            & Android                                               \\ \hline
    \textbf{Público-alvo}           & PDV que buscam informações de rótulos de medicamentos \\
    \hline
  \end{tabular}
  \legend{Fonte: \citeonline{Amariles2020}.}
\end{quadro}

\subsection{\emph{Designing multimodal mobile interaction for a text messaging application for visually impaired users}}

Apesar da inclusão de opções de acessibilidade, os SOs móveis ainda enfrentam uma falta de suporte adequado para alguns tipos de atividades e contextos, como é o exemplo da escrita de textos para PDV, uma tarefa que acaba consumindo muito tempo.
Além disso, os usuários geralmente necessitam utilizar as duas mãos para escrever mensagens, o que mostra ser um problema para cegos que necessitam carregar bengala ou possuem cão guia, assim restando apenas uma mão livre.

Nesse contexto, a abordagem proposta no estudo de \citeonline{Duarte2017}, por meio do protótipo de um \emph{app} para envio de mensagens, visou uma interação com o \emph{smartphone}
com as mãos livres, por meio de técnicas multimodais, especialmente o uso de gestos em combinação com comandos de voz.

Os gestos são utilizados como gatilhos para ações, logo quando um gesto é reconhecido, alguma função como o ``reconhecedor de fala'' ou o TTS\@ é acionada.
Por exemplo, existe um gesto para a ação de adicionar uma nova mensagem, ao reconhece-lo, o \emph{app} ativa o reconhecedor de fala para que o usuário dite o que deve ser escrito na mensagem.

Um outro gesto aciona a função para revisão da mensagem escrita por meio do TTS, em que a mensagem é lida palavra a palavra.
As principais características relacionadas à acessibilidade identificadas nessa solução foram:

\begin{enumerate}
  \item Reconhecimento de voz;
  \item Reconhecimento de gestos;
  \item Síntese de fala.
  \item Possibilidade de revisar as mensagens escritas de maneira acessível;
  \item Possibilidade de parar a narração durante a revisão da mensagem e editar palavras especificas;
  \item Aplicação de questionário da escala SUS\@.
\end{enumerate}

Uma pesquisa foi realizada com 9 usuários com DV e resultou em \emph{feedbacks} positivos, principalmente a respeito da interação por gestos.
O estudo também trouxe comparativo de performance dos usuários na realização de tarefas no \emph{app} de envio de mensagem padrão com o \emph{app} desenvolvido.
Na avaliação da usabilidade das aplicações, fazendo uso da escala SUS, ambas atingiram 74 pontos, considerada uma alta pontuação.

Os resultados mostraram que na realização de tarefas fáceis a performance do \emph{app} era pouco superior a alternativa padrão do sistema.
Porém, passa-se a notar grandes diferenças a favor do \emph{app} desenvolvido em tarefas consideradas normais e difíceis, com cerca de 30\% e 50\% mais performance, respectivamente, para a solução desenvolvida em relação ao \emph{app} padrão.

O artigo não informa as tecnologias utilizadas no desenvolvimento do aplicativo, porém, a plataforma e o público-alvo são
mencionados, como mostra o \autoref{qua-car-am7}.

\begin{quadro}[htb!]
  \caption{\label{qua-car-am7}Características do Desenvolvimento do Aplicativo do AM7.}
  \begin{tabular}{|c|c|}
    \hline
    \textbf{Tecnologias Utilizadas} & Não informado \\ \hline
    \textbf{Plataformas}            & Android       \\ \hline
    \textbf{Público-alvo}           & PDV           \\
    \hline
  \end{tabular}
  \legend{Fonte: \citeonline{Duarte2017}.}
\end{quadro}

\subsection{\emph{Do You like My Outfit? Cromnia, a Mobile Assistant for Blind Users}}

O objetivo do estudo de \citeonline{Giuliana2018} foi projetar uma solução assistiva que pudesse prover autonomia à pessoas cegas em suas atividades diárias.
Especialistas na área de deficiência visual, de clínicos à profissionais de reabilitação vocacional e operadores do campo de cuidados sociais, participaram do estudo.

O processo de análise e projeto envolveu, desde o início, a participação de 4 pessoas cegas da \emph{Italian Blind Union}, que se voluntariaram para colaborar com a equipe de \emph{design} de usabilidade.
Entre as tarefas diárias que mais se esperava autonomia a de vestir-se com uma combinação de cores e roupas adequadas mostrou-se ser o maior interesse para as PDV, essas que geralmente dependem de ajudantes para isso.

O estudo levantou que já existiam soluções no mercado para esse problema, porém a ideia de uma ferramenta paga não foi bem aceita
pelos entrevistados, observando que muitos não poderiam pagar.
Diante disso, uma aplicação \emph{mobile} foi projetada visando a autonomia de PDV total ou parcial nesse ato cotidiano de vestir-se.
O \emph{app} é bem simples e consiste em uma única \emph{interface}, parecida com a padrão da câmera do sistema iOS\@.

As principais soluções de acessibilidade utilizadas no desenvolvimento foram:

\begin{enumerate}
  \item Integração com leitores de tela;
  \item Tamanho de fontes e \emph{labels} adaptáveis de acordo com o tipo de deficiência;
  \item Sistema de notificações simples e imediato;
  \item Resposta em tempo real.
\end{enumerate}

Como resultado do estudo uma aplicação chamada \emph{Cromnia} foi desenvolvida, esta que possibilita que os usuários reconheçam
cores, padrões e combinações de cores, considerando a iluminação do ambiente.

Os testes envolveram 6 PDV com parcial e 6 com DV total.
Os participantes gostaram dos benefícios do \emph{app} e mostraram-se ansiosos para experimentar novas versões,
pensando em quando poderão utilizar o aplicativo de fato no dia-a-dia.
O \emph{app} está disponível para iOS, como mostra o \autoref{qua-car-am8}, na \emph{AppStore}
e conta com alto número de \emph{downloads}.

\begin{quadro}[htb!]
  \caption{\label{qua-car-am8}Características do Desenvolvimento do Aplicativo do AM8.}
  \begin{tabular}{|c|c|}
    \hline
    \textbf{Tecnologias Utilizadas} & Não informado \\ \hline
    \textbf{Plataformas}            & iOS           \\ \hline
    \textbf{Público-alvo}           & PDV           \\
    \hline
  \end{tabular}
  \legend{Fonte: \citeonline{Giuliana2018}.}
\end{quadro}

\subsection{\emph{Improved and Accessible E-Book Reader Application for Visually Impaired People}}

Embora livros digitais já estejam estabelecidos internacionalmente, não são satisfatórios em termos de acessibilidade e \emph{interface}.
Por conta disso, o estudo de \citeonline{Heesook2017} apresenta um aplicativo leitor de \emph{e-book} acessível à PDV, que tem o objetivo
de suprimir limitações como falta de novos livros, ausência de textos alternativos e navegação desconfortável dos atuais formatos
acessíveis (áudio e \emph{Braille}).

Um levantamento de requisitos de usuário foi realizado por meio de questionário e cerca 70\% dos requisitos foram implementados.
O \emph{app} possibilita a realização de busca, \emph{download} e leitura de conteúdos no formato \emph{EPUB3} e possui controles para inciar, parar, avançar e retroceder a leitura.
Quanto à acessibilidade, foram identificadas as seguintes soluções:

\begin{enumerate}
  \item Suporte para comandos de voz;
  \item Configurações de alto contraste;
  \item Sintese de voz para leitura dos \emph{e-books};
  \item Tamanho dos botões e espaçamentos adequados à PDV\@.
\end{enumerate}

Nos resultados dos testes, realizados com 12 PDV (7 experientes e 5 sem experiência), o estudo mostrou que a média de satisfação
dos usuários foi de aproximadamente 75\% nos testes de usabilidade, realizados em 3 fases, com usuários com e sem experiência.
E o tempo médio de execução das tarefas foi de 92 segundos para usuários não experientes e 82 segundos para experientes.

Entretanto, usuários experientes acabaram enfrentando erros relacionados a \emph{login}, configuração e busca por tentarem
utilizar suas próprias abordagens baseadas nas interações com outras aplicações.

Esse artigo também não informa as tecnologias utilizadas no desenvolvimento do aplicativo, assim, o \autoref{qua-car-am6}
mostra as demais características que foram informadas.

\begin{quadro}[htb!]
  \caption{\label{qua-car-am9}Características do Desenvolvimento do Aplicativo do AM9.}
  \begin{tabular}{|c|c|}
    \hline
    \textbf{Tecnologias Utilizadas} & Não informado            \\ \hline
    \textbf{Plataformas}            & iOS                      \\ \hline
    \textbf{Público-alvo}           & PDV que gostam de livros \\
    \hline
  \end{tabular}
  \legend{Fonte: \citeonline{Heesook2017}.}
\end{quadro}

\subsection{\emph{MathMelodies 2: A Mobile Assistive Application for People with Visual Impairments Developed with React Native}}

Esse artigo apresenta a experiência do desenvolvimento do \emph{MathMelodies 2}, uma aplicação para ajudar crianças de 1 a 5 anos com DV no estudo de matemática.
A aplicação apresenta 13 tipos de exercícios e diferentes níveis de dificuldade.
Esses exercícios passam-se dentro de contos de fantasia, os quais a criança tem que resolver para avançar na história.

A primeira versão foi desenvolvida em 2013, financiada por uma campanha de \emph{crowdfunding} e lançada para \emph{iPad} de forma gratuita.
O \emph{design} do novo \emph{app} seguiu princípios derivados da experiência e do \emph{feedback} dos usuários da versão anterior, dos quais
uma das demandas mais frequentes foi a de disponibilização do \emph{app} para outras plataformas, Android e iOS\@.

Assim, nesse trabalho, \citeonline{Ducci2018}, desenvolve essa nova versão como um protótipo, utilizando \emph{React Native} para reduzir o esforço de desenvolvimento.
As principais técnicas e funcionalidades para acessibilidade, utilizadas nesse estudo, são listadas a seguir:

\begin{enumerate}
  \item Implementação nativa para iOS e Android de componentes não acessíveis no \emph{React Native};
  \item Elementos chave de interação sempre posicionados na mesma parte da tela, em locais de fácil acesso;
  \item Tamanho dos ícones e componentes adaptáveis de acordo com tamanho da tela;
  \item Todos os elementos visíveis na tela sem necessidade de rolagem;
  \item Cores de fundo uniformes e neutras;
  \item Interações por gestos simples.
\end{enumerate}

Embora as funcionalidades básicas tenham sido contempladas pelo \emph{framework} utilizado, uma funcionalidade avançada que foi requerida não era suportada.
Por conta disso, foi necessário desenvolver componentes adicionais nativamente, isto é, utilizando as tecnologias especificas para cada plataforma.

Por fim, o artigo conclui que as tecnologias utilizadas, mostradas no \autoref{qua-car-am10}, são uma escolha válida para o desenvolvimento
de aplicações acessíveis e multiplataforma, com base nos resultados de testes preliminares, com participação de duas pessoas (uma com DV
parcial e outra total), sugerindo que a aplicação estava totalmente acessível.

\begin{quadro}[htb!]
  \caption{\label{qua-car-am10}Características do Desenvolvimento do Aplicativo do AM10.}
  \begin{tabular}{|c|c|}
    \hline
    \textbf{Tecnologias Utilizadas} & React Native    \\ \hline
    \textbf{Plataformas}            & Android e iOS   \\ \hline
    \textbf{Público-alvo}           & Crianças com DV \\
    \hline
  \end{tabular}
  \legend{Fonte: \citeonline{Ducci2018}.}
\end{quadro}

\subsection{\emph{Object Recognition and Hearing Assistive Technology Mobile Application Using Convolutional Neural Network}}

A falta de aplicações móveis que atendam pelo menos as necessidades mais comuns de PDV motivou a realização do trabalho de \citeonline{Caballero2020}, que desenvolveu
uma aplicação com objetivo de atender as necessidades desse grupo utilizando tecnologias de Reconhecimento de Objetos (RO) e TTS\@.

O \emph{app} utiliza algoritmos de \emph{Convolutional Neural Network} (CNN), solução de aprendizado de máquina reconhecida como um poderoso método para reconhecimento de
imagens, para identificar detalhes em imagens e narrá\@-los para o usuário por meio do TTS\@.

O artigo concentra-se na apresentação da API utilizada para o RO, mostrando pouco sobre a aplicação \emph{mobile},
ainda assim, foram identificadas as seguintes características de acessibilidade:

\begin{enumerate}
  \item Reconhecimento de detalhes de imagens;
  \item Sintese dos resultados do RO por voz.
\end{enumerate}

O estudo realizou a revisão de diferentes estudos e tecnologias que utilizam CNN, um dos principais estudos citados
foi publicado em 2015 na Conferência Brasileira de Sistemas Inteligentes (BRACIS), este que utiliza RO para um sistema
de navegação inteligente que possibilita que robôs interajam e determinem o comportamento de objetos.
Fazendo uso de informações identificadas nos trabalhos relacionados, o trabalho aplicou o RO direcionado à inclusão social de PDV\@.

Os resultados mostraram que CNN tem potencial para classificar coisas vivas e objetos em ambientes interiores e exteriores com alta precisão, fazendo uso de imagens públicas
como base de treinamento, possibilitando um desempenho funcional e confiável do sistema em benefício das PDV graças ao \emph{app} desenvolvido.

O artigo concentra-se em descrever os algoritmos de CNN utilizados e acaba não apresentando características importantes
do aplicativo desenvolvido, como as tecnologias utilizadas, como mostra o \autoref{qua-car-am11}.

\begin{quadro}[htb!]
  \caption{\label{qua-car-am11}Características do Desenvolvimento do Aplicativo do AM11.}
  \begin{tabular}{|c|c|}
    \hline
    \textbf{Tecnologias Utilizadas} & Não informado \\ \hline
    \textbf{Plataformas}            & Android       \\ \hline
    \textbf{Público-alvo}           & PDV           \\
    \hline
  \end{tabular}
  \legend{Fonte: \citeonline{Caballero2020}.}
\end{quadro}

\subsection{\emph{QUIMIVOX MOBILE 2.0: Application for Helping Visually Impaired People in Learning Periodic Table and Electron Configuration}}

Muito ainda precisa ser feito quanto à inclusão de PDV no processo de ensino e aprendizagem de química, por requerer de muitos recursos visuais.
E, embora exista uma quantidade significativa de \emph{apps} que auxiliam no ensino de química, os mesmos não são acessíveis aos DV, mesmo com o uso de leitores de tela.

Foi nesse sentido que o estudo de \citeonline{Oliveira2019} introduziu uma nova versão do \emph{``Quimivox Mobile 2.0''},
aplicativo que apresenta informações acessíveis à PDV sobre a tabela periódica e, na nova versão, a configuração eletrônica
dos elementos químicos. A interação do \emph{app} é baseada em gestos e comandos de voz, com as informações sendo apresentadas
graficamente e por síntese de voz, graças ao \emph{TalkBack}.

A aplicação utiliza de técnicas de gestos já utilizadas em outras ferramentas que consistem em deslizar com os dedos em quatro
direções. Esses gestos foram complementados com outros específicos para a realização de ações na aplicação, tais como a ativação
do reconhecimento de voz e uma opção para retornar à tela anterior.
Segue abaixo as principais técnicas e funcionalidades para acessibilidade identificadas no estudo:

\begin{enumerate}
  \item Interação por reconhecimento de voz e gestos;
  \item Tamanhos de fontes de letras ampliados;
  \item Alto contraste (fundos pretos e textos brancos);
  \item Possibilidade de escolha de cores do \emph{app} para melhorar a legibilidade para pessoas daltônicas;
  \item \emph{Feedback} sonoro mesmo com \emph{Talkback} desativado.
\end{enumerate}

Os usuários apontaram o comando de voz como a funcionalidade que mais facilitou na utilização da \emph{app}.
Na avaliação de uma das PDV, participante dos testes, o desenvolvimento de manual poderia contribuir com melhor entendimento do funcionamento do aplicativo.
Outras sugestões foram a ampliação dos tipos de toques na tela e o aumento na velocidade da voz sintetizada.

O artigo conclui que os participantes aprovaram a nova versão, avaliando positivamente o \emph{app}, indicando que a maior dificuldade estava na pouca prática no uso de dispositivos móveis por parte de alguns DV\@.
E relata que essa dificuldade estava relacionada aos gestos, com a maioria fazendo algum comentário negativo, citando 5 desses participantes.

Porém, o autor supõe que, com a prática no uso dos gestos, essa dificuldade poderia ser diminuída significativamente, citando o reconhecimento da falta de experiência na utilização de dispositivos móveis por 4 participantes como justificativa, sendo que apenas um deles, chamado P10, fazia parte dos 5 participantes citados pelos comentários negativos.

Por fim, as características do aplicativo desenvolvido são exibidas no \autoref{qua-car-am12}.

\begin{quadro}[htb!]
  \caption{\label{qua-car-am12}Características do Desenvolvimento do Aplicativo do AM12.}
  \begin{tabular}{|c|c|}
    \hline
    \textbf{Tecnologias Utilizadas} & Java, Android Studio e API Airy            \\ \hline
    \textbf{Plataformas}            & Android                                    \\ \hline
    \textbf{Público-alvo}           & PDV interessadas no aprendizado de Química \\
    \hline
  \end{tabular}
  \legend{Fonte: \citeonline{Oliveira2019}.}
\end{quadro}

\subsection{\emph{``Talkin' about the weather'': Incorporating TalkBack functionality and sonifications for accessible app design}}

Informações a respeito do clima atual e previsões são especialmente importantes para PDV, visto que podem afetar suas as decisões
do cotidiano, como escolhas de rotas, roupas e tecnologias assistivas que impactam significativamente seu trajeto.
Mesmo assim, essas pessoas enfrentam péssimas experiencias tentando buscar informações sobre o clima nos dispositivos móveis.

Esses problemas costumam ocorrer devido a erros entre as informações na tela e a ordem em que os leitores de tela as apresentam, além dos
\emph{apps} serem cheios de imagens e ícones que não apresentam descrições para o usuário, a menos que possa enxergá-las.

Diante disso, em \citeonline{Tomlinson2016377}, foi projetado um \emph{app} de clima para ser acessível a usuários que dependem
de leitores de tela. No qual o estudo realizou uma análise das necessidades dos usuários com DV, levantando quais eram as informações
importantes e em qual ordem eles gostariam de consumi-las.

As principais soluções quanto à acessibilidade identificadas foram:

\begin{enumerate}
  \item Alternativa aos ícones padrões utilizados com os chamados ``Ícones audíveis'';
  \item Utilização constante do \emph{TalkBack} durante o processo de desenvolvimento;
  \item Interface com alto contraste (textos brancos em fundo preto), visando a experiência de usuário (UX) de PDV\@;
  \item Integração com \emph{Talkback} seguindo as Diretrizes de Acessibilidade do Google.
\end{enumerate}

``Ícones audíveis'' emitem sons breves, baseados nos sons reais do cotidiano, e servem alternativa para representação dos ícones
visuais de clima, como o ícone de chuva, representado por sons que remetem ao evento.

Nos testes de usabilidade, 7 participantes responderam que utilizaram o \emph{app} por pelo menos seis dias durante a semana
e, no geral, reportaram terem obtido experiência tão boa ou melhor que nos \emph{apps} de clima que já utilizaram anteriormente.

Assim como alguns dos artigos anteriores, esse também não informa as tecnologias utilizadas no desenvolvimento do aplicativo, 
porém, as demais são apresentadas no \autoref{qua-car-am13}.

\begin{quadro}[htb!]
  \caption{\label{qua-car-am13}Características do Desenvolvimento do Aplicativo do AM13.}
  \begin{tabular}{|c|c|}
    \hline
    \textbf{Tecnologias Utilizadas} & Não informado                                   \\ \hline
    \textbf{Plataformas}            & Android                                         \\ \hline
    \textbf{Público-alvo}           & PDV que necessitam de informações sobre o clima \\
    \hline
  \end{tabular}
  \legend{Fonte: \citeonline{Tomlinson2016377}.}
\end{quadro}

\subsection{\emph{Users’ perception on usability aspects of a braille learning mobile application ‘mBRAILLE’}}

Estudantes com DV enfrentam dificuldades ou incapacidade, a depender do nível de DV, para obter informações visuais, o que torna
o processo de aprendizagem deles mais difícil que o dos outros. Nesse artigo, \citeonline{Nahar2019100}, apresenta o \emph{mBRAILLE},
\emph{app} que foi desenvolvido em \emph{Bangladesh} para auxiliar PDV no processo de autoaprendizagem de \emph{Braille}, sem ou com
dependência mínima de outras pessoas.

Embora a publicação não apresente muitos detalhes do processo de desenvolvimento, sequer
mencionam leitores de tela, algumas características relacionadas à acessibilidade utilizadas na solução foram identificadas, seguem:

\begin{enumerate}
  \item \emph{Tutorial} para auxiliar o usuário na utilização do \emph{app};
  \item \emph{Feedback} por vibração e áudio;
\end{enumerate}

O trabalho avaliou 4 aspectos de usabilidade (aprendizagem, interface e funcionalidades, acessibilidade e auto descritividade)
do \emph{app} por meio de testes com 5 usuários com DV, que realizaram a avaliação após utilizarem a aplicação por 2 semanas,
mostrando resultados de avaliação média satisfatórios, de 6 ou acima, numa escala de 0 a 7.

O estudo teve a uma limitação de apenas 5 participantes, sendo todos experientes em \emph{Braille}.
Contudo, o artigo menciona que trabalhos futuros irão concentrar-se na avaliação e testes da efetividade do
aprendizado de \emph{Braille} por meio do \emph{app} com um grande número de participantes de diferentes escolas.

O público-alvo e a plataforma do aplicativo desenvolvido são listados no \autoref{qua-car-am14}, mas, as tecnologias utilizadas
não foram relatadas no estudo.

\begin{quadro}[htb!]
  \caption{\label{qua-car-am14}Características do Desenvolvimento do Aplicativo do AM14.}
  \begin{tabular}{|c|c|}
    \hline
    \textbf{Tecnologias Utilizadas} & Não informado                   \\ \hline
    \textbf{Plataformas}            & Android                         \\ \hline
    \textbf{Público-alvo}           & Estudantes de Bangladesh com DV \\
    \hline
  \end{tabular}
  \legend{Fonte: \citeonline{Nahar2019100}.}
\end{quadro}

\subsection{\emph{WordMelodies: Supporting Children with Visual Impairment in Learning Literacy}}

As ferramentas educacionais de escolas primarias frequentemente não são acessíveis para crianças com DV\@.
Além disso, os livros costumam ser ricos em conteúdos gráficos com o intuito de engajar os alunos, impactando na acessibilidade mesmo quando estão disponíveis no formato digital.
Da mesma forma, \emph{apps} educacionais constantemente apresentam conteúdos gráficos interativos de maneira inacessível à PDV\@.

Visando amenizar esses problemas, o artigo de \citeonline{Mascetti2019} apresenta o \emph{WordMelodies}, uma aplicação \emph{mobile}
inclusiva e multiplataforma que teve o objetivo de ajudar crianças com DV na aquisição de habilidades básicas de literatura com 8
tipos de exercícios.

A aplicação foi projetada e avaliada por 3 especialistas no domínio de tecnologias assistivas e educação
para crianças com DV\@. As principais características relativas à acessibilidade encontradas no artigo foram:

\begin{enumerate}
  \item Elementos chave de interação sempre posicionados na mesma parte da tela, priorizando os cantos da tela;
  \item Interações por gestos como ``arrastar e soltar'' com descrição auditiva;
  \item Descrição alternativa em texto dos elementos de tela para integração com leitores de tela.
\end{enumerate}

Exceto por um problema que afetou a utilização do usuário ao navegar entre os elementos utilizando leitores de tela, o \emph{app}
mostrou-se totalmente acessível na avaliação dos especialistas. Nessa navegação, a ordem dos elementos não corresponde com a ordem
lógica apresentada na tela, problema que ocorreu por uma limitação do \emph{kit} de ferramentas da tecnologia de desenvolvimento
utilizada, mostrada no \autoref{qua-car-am15}.

\begin{quadro}[htb!]
  \caption{\label{qua-car-am15}Características do Desenvolvimento do Aplicativo do AM15.}
  \begin{tabular}{|c|c|}
    \hline
    \textbf{Tecnologias Utilizadas} & React Native    \\ \hline
    \textbf{Plataformas}            & Android e iOS   \\ \hline
    \textbf{Público-alvo}           & Crianças com DV \\
    \hline
  \end{tabular}
  \legend{Fonte: \citeonline{Mascetti2019}.}
\end{quadro}

Um dos principais desafios no desenvolvimento foi alcançar uma funcionalidade de ``arrastar e soltar'' acessível e fácil de utilizar,
pois no \emph{React Native} esse componente não fornece suporte à acessibilidade, sendo necessário o desenvolvimento de um componente
nativo tanto no iOS como no Android, para prover informações auditivas ao usuário durante o uso do componente.

\section{Estudos Relacionados}

Durante o processo de seleção de artigos do MSL, foram encontrados alguns estudos secundários, tipo de estudo que realiza revisão de estudos primários
relacionados a um tema específico \cite{Kitchenham2007}. Embora tenham sido rejeitados no MSL, por se enquadrarem em algum dos critérios definidos
na seção anterior, os estudos que realizaram essas revisões dentro do tema abordado neste trabalho foram considerados como estudos relacionados.

Assim, esta seção apresenta os principais problemas e propostas de soluções relacionados à acessibilidade de aplicações para dispositivos móveis
identificados por esses estudos. No \autoref{qua-art-rev-sis} estão listadas as informações de cada um desses estudos secundários.

\begin{quadro}[htb!]
  \caption{\label{qua-art-rev-sis}Estudos relacionados identificados no processo de MSL.}
  \begin{tabular}{|m{1.2cm} | m{8.1cm} | m{2.7cm} | m{2.5cm}|}
    %\hline
    \hline
    \textbf{Código} & \textbf{Título}                                                                                                             & \textbf{Referência}  & \textbf{Base de dados}     \\
    \hline
    AR1             & \emph{Accessibility of Mobile Applications: Evaluation by Users with Visual Impairment and by Automated Tools}              & \cite{Mateus2020}    & \emph{ACM Digital Library} \\
    \hline
    AR2             & \emph{Can Everyone use my app? An Empirical Study on Accessibility in Android Apps}                                         & \cite{Vendome201941} & \emph{Scopus}              \\
    \hline
    AR3             & \emph{Effect of UX Design Guideline on the information accessibility for the visually impaired in the mobile health apps}   & \cite{Kim20191103}   & \emph{Scopus}              \\
    \hline
    AR4             & \emph{Mobile Device Accessibility for the Visually Impaired: Problems Mapping and Empirical Study of Touch Screen Gestures} & \cite{Damaceno2016}  & \emph{ACM Digital Library} \\
    \hline
    AR5             & \emph{Observation Based Analysis on the Use of Mobile Applications for Visually Impaired Users}                             & \cite{Siebra2016}    & \emph{ACM Digital Library} \\
    \hline
    AR6             & \emph{Prioritization of mobile accessibility guidelines for visual impaired users}                                          & \cite{Quispe2020}    & \emph{Scopus}              \\
    \hline
    % \hline
  \end{tabular}
  \legend{Fonte: Autor.}
\end{quadro}

% ---
\subsection{\emph{Accessibility of Mobile Applications: Evaluation by Users with Visual Impairment and by Automated Tools}}
% ---

O artigo apresenta um estudo comparativo de problemas de acessibilidade encontrados pelas ferramentas automatizadas MATE e \emph{Accessibility Scanner}, com os problemas encontrados em um estudo anterior envolvendo 11 usuários com DV\@.
Além disso, o trabalho sumarizou e categorizou os problemas mais encontrados pelos usuários.
As principais categorias são listadas na \autoref{tab-cat-pro-1}.

\begin{table}[htb]
  \begin{center}
    \ABNTEXfontereduzida
    \caption{Categorias dos tipos de problemas mais identificados.}
    \label{tab-cat-pro-1}
    \begin{tabular}{p{2.0cm}|p{7cm}}
      %\hline
      \textbf{Código} & \textbf{Categoria}                       \\
      \hline
      CPF1            & Botões                                   \\
      \hline
      CPF2            & Características do Sistema               \\
      \hline
      CPF3            & Conteúdo e Significado                   \\
      \hline
      CPF4            & Controles, formulários e funcionalidades \\
      \hline
      CPF5            & Imagem                                   \\
      % \hline
    \end{tabular}
    \legend{Fonte: \citeonline{Christoph2020}.}
  \end{center}
\end{table}

Na \autoref{tab-pro-blind-1} são listados os principais tipos de problemas, que apresentaram um total de pelo menos 10 observações.
As categorias, de acordo com a \autoref{tab-cat-pro-1}, e o número total de observações para cada tipo de DV\@ (total ou parcial) também são relacionados à cada tipo de problema.
Como o artigo só menciona os tipos problemas encontrados com maior frequência por cada tipo de usuário, o número de observações de alguns não estão presentes na \autoref{tab-pro-blind-1}.

\begin{table}[htb]
  \begin{center}
    \ABNTEXfontereduzida
    \caption{Problemas mais frequentes encontrados pelos usuários por tipo de DV.}
    \label{tab-pro-blind-1}
    \begin{tabular}{p{1.2cm}|p{8.7cm}|p{1.4cm}|p{0.6cm}|p{0.6cm}|p{0.7cm}}
      %\hline
      \textbf{Código} & \textbf{Problema}                                                        & \textbf{Categoria} & \textbf{DVT} & \textbf{DVP} & \textbf{Total} \\
      \hline
      AR1P1           & \emph{Feedback} inapropriado.                                            & CPF4               & 34           & 15           & 49             \\
      \hline
      AR1P2           & Falta de informações.                                                    & CPF1               & 22           & 8            & 30             \\
      \hline
      AR1P3           & Usuários presumiram que era uma funcionalidade.                          & CPF4               & 18           & 9            & 27             \\
      \hline
      AR1P4           & Funcionalidades confusas ou não claras.                                  & CPF4               & 25           & -            & 25             \\
      \hline
      AR1P5           & Apresentação padrão de elementos de controle ou formulário não adequada. & CPF4               & 11           & 12           & 23             \\
      \hline
      AR1P6           & Sequências de interação confusas ou não claras.                          & CPF4               & 15           & 6            & 21             \\
      \hline
      AR1P7           & Usuários não entenderam sentido do conteúdo.                             & CPF3               & 15           & 5            & 20             \\
      \hline
      AR1P8           & Organização do conteúdo inconsistente.                                   & CPF3               & 12           & 6            & 18             \\
      \hline
      AR1P9           & Funcionalidade não funciona como esperado.                               & CPF4               & 6            & 10           & 16             \\
      \hline
      AR1P10          & Funcionalidades dos botões confusas ou não claras.                       & CPF1               & 15           & -            & 15             \\
      \hline
      AR1P11          & Expectativa de funcionalidade que não existe.                            & CPF4               & 10           & 5            & 15             \\
      \hline
      AR1P12          & Sem alternativa textual.                                                 & CPF5               & 14           & -            & 14             \\
      \hline
      AR1P13          & Sistema muito lento.                                                     & CPF2               & -            & 11           & 11             \\
      \hline
      AR1P14          & Significado no conteúdo está perdido.                                    & CPF3               & 6            & 4            & 10             \\
      % \hline
    \end{tabular}
    \legend{Fonte: \citeonline{Christoph2020}.}
  \end{center}
\end{table}

Os resultados do estudo mostraram que 36 tipos de problemas foram encontrados somente pelos usuários, 11 somente pelas ferramentas
e 3 por ambos os métodos. Evidenciando assim a necessidade de utilização de mais de um método para identificação dos problemas de
acessibilidade.

Além disso, o estudo mostrou a importância da utilização dessas ferramentas automatizadas, visto que parte significativa dos problemas
podem ser identificados ainda no processo de desenvolvimento, reduzindo o esforço e, consequentemente, o custo para solucioná-los.

% ---
\subsection{\emph{Can Everyone use my app? An Empirical Study on Accessibility in Android Apps}}
% ---

Esse trabalho realizou um estudo piloto no qual foi observado que desenvolvedores de aplicativos móveis raramente utilizam as APIs de Acessibilidade e que o uso de descrições alternativas para elementos de \emph{interface} também é limitado.
Ademais, visando entender a perspectiva desses desenvolvedores, o estudo também realizou uma investigação de postagens no Stack Overflow, identificando os aspectos de acessibilidade que os desenvolvedores implementavam e os que experienciavam dificuldades.

O estudo investigou aspectos de acessibilidade no geral, baseado em 336 discussões de desenvolvedores Android no Stack Overflow, sendo 159 dessas sobre acessibilidade à DV\@.
Dessas 159 discussões, os principais aspectos discutidos foram sobre \emph{feedbacks} sonoros e legibilidade (114 e 24 postagens, respectivamente) como mostra a \autoref{tab-acc-asp-sta-flow}.

\begin{table}[htb]
  \begin{center}
    \ABNTEXfontereduzida
    \caption{Aspectos de acessibilidade à DV discutidos por \emph{devs} Android no Stack Overflow.}
    \label{tab-acc-asp-sta-flow}
    \begin{tabular}{p{1.2cm}|p{7.0cm}|p{3.8cm}}
      %\hline
      \textbf{Código} & \textbf{Aspecto}                       & \textbf{Categoria}       \\
      \hline
      AR2P1           & Alertas de acessibilidade              & \emph{Feedbacks} sonoros \\
      \hline
      AR2P2           & Ampliação da tela                      & Legibilidade             \\
      \hline
      AR2P3           & Aspectos não funcionais                & \emph{Feedbacks} sonoros \\
      \hline
      AR2P4           & Consciência de contexto                & \emph{Feedbacks} sonoros \\
      \hline
      AR2P5           & Conteúdos, ações e gestos customizados & \emph{Feedbacks} sonoros \\
      \hline
      AR2P6           & \emph{Frameworks} de terceiros         & \emph{Feedbacks} sonoros \\
      \hline
      AR2P7           & \emph{Mobile web apps}                 & \emph{Feedbacks} sonoros \\
      \hline
      AR2P8           & Problemas com serviços                 & \emph{Feedbacks} sonoros \\
      \hline
      AR2P9           & Sons e vibrações                       & \emph{Feedbacks} sonoros \\
      \hline
      AR2P10          & Suporte à \emph{Braille}               & Teclados alternativos    \\
      \hline
      AR2P11          & Tamanho de fonte                       & Legibilidade             \\
      \hline
      AR2P12          & Teclado customizado                    & Teclados alternativos    \\
      \hline
      AR2P13          & Transformações de cores                & Transformações de cores  \\
      % \hline
    \end{tabular}
    \legend{Fonte: \citeonline{Vendome201941}.}
  \end{center}
\end{table}

O trabalho de \citeonline{Vendome201941} analisou 13.817 \emph{apps} Android de código aberto, descobrindo que cerca de 50\% deles tinham descrições alternativas para todos os elementos, enquanto cerca de 37\% não tinha nenhuma.
Além disso, o artigo apontou que apenas cerca de 2\% desses \emph{apps} utilizavam alguma API de acessibilidade no projeto.

% ---
\subsection{\emph{Effect of UX Design Guideline on the information accessibility for the visually impaired in the mobile health apps}}
% ---

Acessibilidade de informações visuais para DV raramente é considerada ao projetar aplicações móveis para saúde \cite{Kim20191103}.
Diante disso, o artigo propõe um guia de diretrizes de acessibilidade à DV, chamado UXDG (\emph{UX Design Guideline}), para resolver esse problema.
A \autoref{tab-acc-dir-uxd-1} lista as diretrizes do UXDG de acordo com as categorias.

Como parte da validação do guia, 120 \emph{apps} da área de saúde foram analisados quanto à taxa de conformidade com o guia.
Na análise desses \emph{apps}, a média da taxa de conformidade com o guia foi de 39,24\%, com a diretriz AR3D7 apresentando
a maior taxa, com 71,67\%, enquanto a AR3D9 apresentou a menor, com 5\%.

\begin{table}[htb]
  \begin{center}
    \ABNTEXfontereduzida
    \caption{Diretrizes do UXDG por categoria.}
    \label{tab-acc-dir-uxd-1}
    \begin{tabular}{p{1.2cm}|p{8.8cm}|p{4.5cm}}
      %\hline
      \textbf{Código} & \textbf{Diretriz}                                                    & \textbf{Categoria}             \\
      \hline
      AR3D1           & Destacar as mídias que disparam ação.                                & Aquisição de informação        \\
      \hline
      AR3D2           & Destacar as principais imagens que o usuário pode acessar.           & Aquisição de informação        \\
      \hline
      AR3D3           & Navegação intuitiva.                                                 & Acessibilidade dos dados       \\
      \hline
      AR3D4           & Posicionar a caixa de pesquisa sempre no mesmo local.                & Busca de dados                 \\
      \hline
      AR3D5           & Posicionar resultados de buscas logo após a caixa de texto.          & Busca de dados                 \\
      \hline
      AR3D6           & Reconhecimento de voz para entrada de texto.                         & Busca de dados                 \\
      \hline
      AR3D7           & Resposta intuitiva do \emph{menu} de acordo com intenção do usuário. & Acessibilidade dos dados       \\
      \hline
      AR3D8           & Suporte à esquemas de cores alternativos.                            & Melhora na exposição dos dados \\
      \hline
      AR3D9           & Suporte de \emph{zoom in/out} para os principais conteúdos.          & Melhora na exposição dos dados \\
      \hline
      AR3D10          & Suporte para outros métodos entrada além do toque.                   & Acessibilidade dos dados       \\
      \hline
      AR3D11          & Uso de fontes com alta legibilidade.                                 & Aquisição de informação        \\
      % \hline
    \end{tabular}
    \legend{Fonte: \citeonline{Kim20191103}.}
  \end{center}
\end{table}

O estudo realizou testes, conduzidos com 23 PDV e 23 sem DV, comparando \emph{apps} selecionados da área da saúde antes e depois da aplicação do UXDG\@.
Os resultados apontaram que houve um aumento na velocidade de reconhecimento das informações depois de aplicar as diretrizes.
De acordo com o experimento, esse aumento aconteceu tanto para usuários com DV, aumento de 13,68\%, quanto para os sem, de 32,41\%.

% ---
\subsection{\emph{Mobile Device Accessibility for the Visually Impaired: Problems Mapping and Empirical Study of Touch Screen Gestures}}
% ---

Esse artigo, mediante um MSL, apresenta os problemas de acessibilidade enfrentados na utilização de dispositivos móveis por PDV encontrados na literatura.
A \autoref{tab-cat-pro-4} mostra, como categorias, 6 dos 7 grupos de problemas identificados no estudo,
desconsiderando o de ``borda não sensível ao toque'', visto que esse é um problema relativo aos dispositivos físicos.

\begin{table}[htb]
  \begin{center}
    \ABNTEXfontereduzida
    \caption{Categorias dos problemas mapeados na literatura.}
    \label{tab-cat-pro-4}
    \begin{tabular}{p{2.0cm}|p{5.0cm}}
      %\hline
      \textbf{Código} & \textbf{Categoria}   \\
      \hline
      CPM1            & Botões               \\
      \hline
      CPM2            & Comandos de voz      \\
      \hline
      CPM3            & Entrada de dados     \\
      \hline
      CPM4            & Interação por gestos \\
      \hline
      CPM5            & Leitor de tela       \\
      \hline
      CPM6            & Retorno ao usuário   \\
      % \hline
    \end{tabular}
    \legend{Fonte: \citeonline{Damaceno2016}.}
  \end{center}
\end{table}

Na \autoref{tab-pro-1-2-6} são listados os problemas relacionados à botões (CPM1), comandos de voz (CPM2) e retorno do usuário (CPM6),
e o número de citações, este último que corresponde ao número de estudos nos quais o problema foi identificado, sendo que os problemas
relacionados aos botões físicos dos dispositivos foram desconsiderados, por estarem fora do controle da aplicação.

\begin{table}[htb]
  \begin{center}
    \ABNTEXfontereduzida
    \caption{Problemas relacionados às categorias CPM1, CPM2 e CPM6.}
    \label{tab-pro-1-2-6}
    \begin{tabular}{p{1.2cm}|p{10.0cm}|p{1.4cm}|p{1.4cm}}
      %\hline
      \textbf{Código} & \textbf{Problema}                                                                                & \textbf{Categoria} & \textbf{Citações} \\
      \hline
      AR4P1           & A grande proximidade entre os botões virtuais dificulta a interação.                             & CPM1               & 1                 \\
      \hline
      AR4P2           & Os botões virtuais acarretam menor sensibilidade tátil.                                          & CPM1               & 1                 \\
      \hline
      AR4P3           & Apenas um comando de voz é reconhecido por vez.                                                  & CPM2               & 2                 \\
      \hline
      AR4P4           & Há baixa privacidade ao emitir comandos de voz.                                                  & CPM2               & 1                 \\
      \hline
      AR4P5           & Há diminuição do desempenho do reconhecimento em condições de ruído.                             & CPM2               & 1                 \\
      \hline
      AR4P6           & Há diminuição do desempenho do reconhecimento devido à entonação e à acentuação.                 & CPM2               & 1                 \\
      \hline
      AR4P7           & Há dificuldade para ativar comando de voz.                                                       & CPM2               & 1                 \\
      \hline
      AR4P8           & Há necessidade de mentalizar instrução por voz, aumentando carga de memória do indivíduo.        & CPM2               & 1                 \\
      \hline
      AR4P9           & O reconhecimento de voz funciona apenas em alguns aplicativos.                                   & CPM2               & 1                 \\
      \hline
      AR4P10          & O uso de comandos de voz é computacionalmente custoso.                                           & CPM2               & 1                 \\
      \hline
      AR4P11          & Há ausência de retorno ao usuário, ao interagir com alguns elementos de interface.               & CPM6               & 1                 \\
      \hline
      AR4P12          & Há dificuldade para compreender diferentes padrões vibratórios.                                  & CPM6               & 1                 \\
      \hline
      AR4P13          & Há dificuldade para compreender a orientação da interface, utilizando apenas o retorno auditivo. & CPM6               & 1                 \\
      \hline
      AR4P14          & Retorno auditivo é prejudicado em ambientes ruidosos.                                            & CPM6               & 2                 \\
      \hline
      AR4P15          & Usar apenas o retorno auditivo não é o suficiente para a interação.                              & CPM6               & 1                 \\
      % \hline
    \end{tabular}
    \legend{Fonte: \citeonline{Damaceno2016}.}
  \end{center}
\end{table}

\newpage

A \autoref{tab-pro-ent-dad-1} mostra os problemas relacionados à entrada de dados (CPM3) com o número de citações para cada problema.
Os problemas que mencionavam teclado físico de dispositivos móveis foram desconsiderados, pois a aplicação desenvolvida suporta apenas
\emph{smartphones}.

\begin{table}[htb]
  \begin{center}
    \ABNTEXfontereduzida
    \caption{Problemas relacionados à entrada de dados (CPM3).}
    \label{tab-pro-ent-dad-1}
    \begin{tabular}{p{1.2cm}|p{12.0cm}|p{1.3cm}}
      %\hline
      \textbf{Código} & \textbf{Problema}                                                                                                                   & \textbf{Citações} \\
      \hline
      AR4P16          & A digitação de textos é lenta em teclados QWERTY virtuais.                                                                          & 2                 \\
      \hline
      AR4P17          & As teclas mais distantes das bordas são mais difíceis de encontrar do que as mais próximas das bordas, em teclados virtuais QWERTY. & 1                 \\
      \hline
      AR4P18          & É preciso conhecer previamente Braille para ter bom desempenho de digitação utilizando esta modalidade.                             & 2                 \\
      \hline
      AR4P19          & É preciso trocar o modo do teclado virtual, para acessar determinados caracteres.                                                   & 1                 \\
      \hline
      AR4P20          & Há ausência de marca tátil para o número 5, no teclado numérico virtual, e para as letras “F” e “J” no teclado QWERTY virtual.      & 2                 \\
      \hline
      AR4P21          & Há erros ao corrigir caracteres digitados equivocadamente, substituindo por fonemas semelhantes, em teclados virtuais.              & 1                 \\
      \hline
      AR4P22          & Há erros de omissão de caracteres, faltando um ou mais ao digitar palavras em teclados virtuais.                                    & 1                 \\
      \hline
      AR4P23          & Há necessidade de confirmação de cada caractere digitado em teclados virtuais.                                                      & 1                 \\
      \hline
      AR4P24          & Há necessidade de navegar pelo teclado virtual para localizar os caracteres desejados.                                              & 1                 \\
      \hline
      AR4P25          & Há um segundo de espera para entrar com cada tecla em teclados virtuais.                                                            & 1                 \\
      \hline
      AR4P26          & O teclado numérico virtual é denso dificultando, a interação.                                                                       & 1                 \\
    \end{tabular}
    \legend{Fonte: \citeonline{Damaceno2016}.}
  \end{center}
\end{table}

A \autoref{tab-pro-int-ges-1} lista os problemas relacionados à interação por gestos (CPM4) com o número de citações para cada problema encontrado.

\begin{table}[htb]
  \begin{center}
    \ABNTEXfontereduzida
    \caption{Problemas relacionados à interação por gestos (CPM4).}
    \label{tab-pro-int-ges-1}
    \begin{tabular}{p{1.2cm}|p{12.0cm}|p{1.2cm}}
      %\hline
      \textbf{Código} & \textbf{Problema}                                                                   & \textbf{Citações} \\
      \hline
      AR4P27          & Baixa flexibilidade de ângulo e velocidade dos gestos dificultam o reconhecimento.  & 1                 \\
      \hline
      AR4P28          & Gestos com forma da letra “L” são difíceis de fazer.                                & 2                 \\
      \hline
      AR4P29          & Gestos com formas geométricas fechadas (círculo e triângulo) são difíceis de fazer. & 1                 \\
      \hline
      AR4P30          & Gestos com formas geométricas são lentos de se fazer.                               & 1                 \\
      \hline
      AR4P31          & Conflito na desambiguação entre dois toques com um dedo e três toques com um dedo.  & 1                 \\
      \hline
      AR4P32          & Dificuldade para fazer gestos estando em movimento.                                 & 1                 \\
      \hline
      AR4P33          & Dificuldade para fazer gestos próximos à barra superior de sistemas.                & 1                 \\
      \hline
      AR4P34          & Dificuldade para fazer gestos representados por símbolos.                           & 2                 \\
      \hline
      AR4P35          & Dificuldade para fazer o gesto de dois toques com um dedo.                          & 1                 \\
      \hline
      AR4P36          & Dificuldade para se localizar na tela para realizar gestos.                         & 1                 \\
      \hline
      AR4P37          & Erros na detecção de gestos multitoque.                                             & 1                 \\
      \hline
      AR4P38          & Falha de interpretação de gestos em geral, pelo sistema.                            & 4                 \\
      \hline
      AR4P39          & Mudança indevida de foco ao tentar fazer o gesto dois toques com um dedo.           & 1                 \\
      \hline
      AR4P40          & Não é possível alterar mapeamento dos gestos às funções do sistema.                 & 1                 \\
      \hline
      AR4P41          & Não há consistência de gestos entre diferentes sistemas.                            & 1                 \\
      \hline
      AR4P42          & Não há gestos que acionam as principais funções do sistema.                         & 1                 \\
      \hline
      AR4P43          & O toque acidental na tela, com outro dedo, prejudica o reconhecimento de gestos.    & 1                 \\
      \hline
      AR4P44          & Os manuais de explicação de como fazer gestos de toque não são eficientes.          & 3                 \\
      \hline
      AR4P45          & Conflito entre do aplicativo gestos e os do leitor de tela do sistema.              & 1                 \\
      % \hline
    \end{tabular}
    \legend{Fonte: \citeonline{Damaceno2016}.}
  \end{center}
\end{table}

\newpage

Por fim, são listados, na \autoref{tab-pro-lei-tel-1}, os problemas relacionados a leitores de tela (CPM5) com o número de citações.

\begin{table}[htb]
  \begin{center}
    \ABNTEXfontereduzida
    \caption{Problemas relacionados a leitores de tela (CPM5).}
    \label{tab-pro-lei-tel-1}
    \begin{tabular}{p{1.2cm}|p{12.0cm}|p{1.4cm}}
      %\hline
      \textbf{Código} & \textbf{Problema}                                                                                 & \textbf{Citações} \\
      \hline
      AR4P46          & A leitura é linear, demorando para se ter noção global da interface.                              & 2                 \\
      \hline
      AR4P47          & A pronúncia de algumas palavras é problemática.                                                   & 1                 \\
      \hline
      AR4P48          & A voz do leitor de tela é artificial.                                                             & 1                 \\
      \hline
      AR4P49          & Alguns elementos de interface não são lidos.                                                      & 3                 \\
      \hline
      AR4P50          & Há baixa familiaridade com o leitor de tela de dispositivos móveis.                               & 1                 \\
      \hline
      AR4P51          & Há conflito ao usar o leitor de tela do sistema em conjunto com o leitor embutido em aplicativos. & 2                 \\
      \hline
      AR4P52          & Há desconforto ao ouvir o leitor de tela em ambientes ruidosos.                                   & 2                 \\
      \hline
      AR4P53          & Há leitura de apenas o que está em foco.                                                          & 1                 \\
      \hline
      AR4P54          & Não há controle de velocidade de leitura.                                                         & 2                 \\
      \hline
      AR4P55          & Não há um botão para interromper a leitura imediatamente.                                         & 1                 \\
      \hline
      AR4P56          & O foco do leitor de tela muda indevidamente.                                                      & 2                 \\
      \hline
      AR4P57          & O foco do leitor de tela não possui uma ordem de navegação lógica.                                & 2                 \\
      \hline
      AR4P58          & O leitor de tela é lento.                                                                         & 1                 \\
      \hline
      AR4P59          & O texto lido é, por vezes, inadequado.                                                            & 1                 \\
      % \hline
    \end{tabular}
    \legend{Fonte: \citeonline{Damaceno2016}.}
  \end{center}
\end{table}

% ---
\subsection{\emph{Observation Based Analysis on the Use of Mobile Applications for Visually Impaired Users}}
% ---

O estudo realizou uma análise, envolvendo 5 PDV, com o objetivo de validar se a falta dos requisitos de acessibilidade levantados
em um trabalho anterior realmente impactavam na utilização de \emph{apps} móveis por PDV\@.

\begin{table}[htb]
  \begin{center}
    \ABNTEXfontereduzida
    \caption{Categorias dos requisitos encontrados.}
    \label{tab-cat-req-enc-5}
    \begin{tabular}{p{2.0cm}|p{5.0cm}}
      %\hline
      \textbf{Código} & \textbf{Categoria}                \\
      \hline
      CRED1           & \emph{Feedbacks} audíveis         \\
      \hline
      CRED2           & Adaptação das informações visuais \\
      \hline
      CRED3           & Navegação                         \\
      % \hline
    \end{tabular}
    \legend{Fonte: \citeonline{Siebra2016}.}
  \end{center}
\end{table}

Os requisitos foram divididos em 3 categorias, como mostra a \autoref{tab-cat-req-enc-5}.
Baseados na análise dos resultados, o estudo qualificou os requisitos em 3 níveis (Essencial, Desejável e Não observado).

Como os requisitos ``não observados'', de acordo com o artigo, não foram mencionados pelos participantes dos testes, apenas
os requisitos essenciais e desejáveis são listados na \autoref{tab-req-ess-des-1}.
Somente um requisito foi classificado como desejável pelo estudo, o AR5R7, o restante foi classificado como essencial.

\begin{table}[htb]
  \begin{center}
    \ABNTEXfontereduzida
    \caption{Requisitos essenciais e desejáveis focados em DV.}
    \label{tab-req-ess-des-1}
    \begin{tabular}{p{1.2cm}|p{12.1cm}|p{1.4cm}}
      %\hline
      \textbf{Código} & \textbf{Requisito}                                                                                                & \textbf{Categoria} \\
      \hline
      AR5R1           & O nome do caractere que está sendo digitado deve ser ouvido.                                                      & CRED1              \\
      \hline
      AR5R2           & Nomes de elementos e imagens na tela devem ser ouvidos ao serem tocados ou selecionados.                          & CRED1              \\
      \hline
      AR5R3           & \emph{Feedback} de ações/interações devem ser claros e fornecidos de forma tátil, voz ou eventos sonoros.         & CRED1              \\
      \hline
      AR5R4           & Estratégias para o uso de leitores de tela (ex.\@: atalhos para navegar na tela de forma mais eficiente.)         & CRED1              \\
      \hline
      AR5R5           & Prover uma chave ``home'' tátil de acesso fácil e rápido para que um usuário possa retornar a um lugar conhecido. & CRED2              \\
      \hline
      AR5R6           & Prover documentação em formatos alternativos, utilizando fontes grandes.                                          & CRED3              \\
      \hline
      AR5R7           & Permitir customizações pelo usuário e evitar que essas preferências sejam perdidas.                               & CRED3              \\
      \hline
      AR5R8           & Apresentar amplificador com \emph{zoom} ajustável.                                                                & CRED3              \\
      \hline
      AR5R9           & Prover equivalências textuais claras para evitar erros quando os textos são lidos na tela.                        & CRED3              \\
      \hline
      AR5R10          & Brilho, contrate e cores ajustáveis.                                                                              & CRED3              \\
      \hline
      AR5R11          & Prover alertas informativos por outros canais além do visual (ex.\@: voz.)                                        & CRED3              \\
      % \hline
    \end{tabular}
    \legend{Fonte: \citeonline{Siebra2016}.}
  \end{center}
\end{table}

% ---
\subsection{\emph{Prioritization of mobile accessibility guidelines for visual impaired users}}
% ---

O artigo apresenta uma proposta de priorização de diretrizes de acessibilidade que resultaram de estudos anteriores.
Essas diretrizes foram baseadas no eMAG, entretanto, diretrizes como as da BCC (\emph{BBC Mobile Accessibility Guidelines}) e recomendações da plataforma Android também foram consideradas.
Para criação do \emph{ranking}, o estudo utilizou um questionário que foi respondido 103 vezes, sendo 66 dessas respostas de PDV, nas quais a análise se concentrou.

O estudo dividiu as diretrizes em 6 categorias que podem ser visualizadas na \autoref{tab-cat-dir-acc-5}.

\begin{table}[htb]
  \begin{center}
    \ABNTEXfontereduzida
    \caption{Categorias das diretrizes de acessibilidade \emph{mobile} baseadas no eMAG.}
    \label{tab-cat-dir-acc-5}
    \begin{tabular}{p{1.5cm}|p{4.5cm}}
      %\hline
      \textbf{Código} & \textbf{Categoria}         \\
      \hline
      AR6CE           & Estrutura                  \\
      \hline
      AR6CC           & Comportamento              \\
      \hline
      AR6CCI          & Conteúdo/Informação        \\
      \hline
      AR6CAD          & Apresentação/\emph{Design} \\
      \hline
      AR6CM           & Multimídia                 \\
      \hline
      AR6CF           & Formulários                \\
      % \hline
    \end{tabular}
    \legend{Fonte: \citeonline{Quispe2020}.}
  \end{center}
\end{table}

O estudo considerou a priorização para 4 grupos diferentes, baseados no tipo de DV (baixa visão, visão parcial e os 2 tipos de cegueira: legal e total).
E os resultados mostraram que existiam diferenças notáveis na percepção das diretrizes entre os grupos.

\begin{quadro}[htb!]
  \begin{center}
    \ABNTEXfontereduzida
    \caption{\label{qua-pri-acc-gui}Priorização de diretrizes de acessibilidade para usuários com DV.}
    \begin{tabular}{|m{0.5cm} | m{2.4cm} | m{2.4cm} | m{2.8cm} | m{3.0cm} | m{2.4cm}|}
      %\hline
      \hline
      \textbf{Id} & \textbf{Visão parcial} & \textbf{Baixa visão} & \textbf{Cegueira legal}       & \textbf{Cegueira total}              & \textbf{Todas as DV} \\
      \hline
      1           & AR6D7, AR6D28          & AR6D9                & AR6D24, AR6D28                & AR6D8, AR6D22, AR6D23                & AR6D8                \\
      \hline
      2           & AR6D25                 & AR6D11, AR6D28       & AR6D7, AR6D11, AR6D22, AR6D25 & AR6D1, AR6D6, AR6D25, AR6D28         & AR6D25               \\
      \hline
      3           & AR6D24                 & AR6D25               & AR6D1, AR6D27                 & AR6D7, AR6D9, AR6D15, AR6D24, AR6D27 & AR6D7, AR6D9, AR6D22 \\
      \hline
      4           & AR6D9, AR6D22          & AR6D7, AR6D22        & AR6D9                         & AR6D11                               & AR6D11, AR6D24       \\
      \hline
      5           & AR6D11                 & AR6D10               & AR6D10                        & AR6D10                               & AR6D27               \\
      \hline
      6           & AR6D10, AR6D27         & AR6D15               & AR6D8, AR6D23                 & \-                                   & AR6D1                \\
      \hline
      7           & AR6D8, AR6D15          & AR6D1, AR6D6, AR6D27 & AR6D6                         & \-                                   & AR6D8                \\
      \hline
      8           & AR6D6                  & AR6D8                & AR6D15                        & \-                                   & AR6D6, AR6D23        \\
      \hline
      9           & AR6D1                  & AR6D23, AR6D24       & \-                            & \-                                   & AR6D10, AR6D15       \\
      \hline
      10          & AR6D23                 & \-                   & \-                            & \-                                   & \-                   \\
      \hline
      % \hline
    \end{tabular}
    \legend{Fonte: \citeonline{Quispe2020}.}
  \end{center}
\end{quadro}

A partir desses resultados, o trabalho relacionou as diretrizes com as percepções de cada grupo
e criou a lista de priorização que pode ser vista no \autoref{qua-pri-acc-gui}.
No qual a coluna ``Id'' informa a ordem de priorização e os códigos que estão nas demais são listados na \autoref{tab-dir-acc-mob-1} junto com as diretrizes.

\begin{table}[htb]
  \begin{center}
    \ABNTEXfontereduzida
    \caption{Diretrizes de acessibilidade \emph{mobile} baseadas no eMAG.}
    \label{tab-dir-acc-mob-1}
    \begin{tabular}{p{1.2cm}|p{12.0cm}|p{1.5cm}}
      %\hline
      \textbf{Código} & \textbf{Diretriz}                                                                                                                            & \textbf{Categoria} \\
      \hline
      AR6D1           & Elementos de tela devem ser organizados de maneira lógica e semântica.                                                                       & AR6CE              \\
      \hline
      AR6D2           & As telas devem apresentar sequência lógica de leitura para navegação entre \emph{links}, controles de formulário e outros elementos.         & AR6CE              \\
      \hline
      AR6D3           & \emph{Links} na tela devem ser organizados para evitar confusão.                                                                             & AR6CE              \\
      \hline
      AR6D4           & Informações devem ser divididas em grupos específicos para facilitar a procura e leitura dos conteúdos.                                      & AR6CE              \\
      \hline
      AR6D5           & Usuários devem ser informados se \emph{links} abrem novas telas para poderem decidir se querem ou não sair da tela atual.                    & AR6CE              \\
      \hline
      AR6D6           & Todas as funcionalidades na tela devem estar disponíveis a partir do teclado.                                                                & AR6CC              \\
      \hline
      AR6D7           & Todos os elementos de \emph{interface} na tela devem ser acessíveis.                                                                         & AR6CC              \\
      \hline
      AR6D8           & Redirecionamento automático de telas não deve acontecer.                                                                                     & AR6CC              \\
      \hline
      AR6D9           & Em telas com limite de tempo, deve haver opções para desligar ou ajustar o tempo.                                                            & AR6CC              \\
      \hline
      AR6D10          & Não deve haver efeitos visuais piscantes, intermitentes ou cintilantes na tela.                                                              & AR6CC              \\
      \hline
      AR6D11          & Conteúdos animados não devem iniciar automaticamente.                                                                                        & AR6CC              \\
      \hline
      AR6D12          & A linguagem utilizada na tela deve ser especificada.                                                                                         & AR6CCI             \\
      \hline
      AR6D13          & Mudanças na linguagem dos conteúdos sempre devem ser especificadas.                                                                          & AR6CCI             \\
      \hline
      AR6D14          & Títulos de telas devem ser descritivos, informativos e representativos com relação ao conteúdo principal.                                    & AR6CCI             \\
      \hline
      AR6D15          & Deve haver algum mecanismo para indicar ao usuário onde ele está no momento, no conjunto de telas.                                           & AR6CCI             \\
      \hline
      AR6D16          & Alvos de \emph{links} devem ser identificados claramente, incluindo informações sobre se estão funcionando ou se direcionam para outra tela. & AR6CCI             \\
      \hline
      AR6D17          & Todas as imagens devem possuir descrição textual.                                                                                            & AR6CCI             \\
      \hline
      AR6D18          & Documentos em formatos acessíveis devem estar disponíveis.                                                                                   & AR6CCI             \\
      \hline
      AR6D19          & Quando uma tabela é utilizada na tela, título e sumário apropriados devem ser fornecidos.                                                    & AR6CCI             \\
      \hline
      AR6D20          & Os textos nas telas devem ser fáceis de ler e entender.                                                                                      & AR6CCI             \\
      \hline
      AR6D21          & Totos as siglas, abreviações e palavras incomuns na tela devem possuir explicação.                                                           & AR6CCI             \\
      \hline
      AR6D22          & Deve haver uma taxa miníma de contraste entre as cores de fundo e as de frente.                                                              & AR6CAD             \\
      \hline
      AR6D23          & Características sensoriais (ex.\: cores, formas e sons) não podem ser o único significado para distinguir elementos de tela.                 & AR6CAD             \\
      \hline
      AR6D24          & O elemento ou área em foco deve ser evidente visualmente.                                                                                    & AR6CAD             \\
      \hline
      AR6D25          & Vídeos que não incluem áudio devem fornecer alternativas como legendas.                                                                      & AR6CM              \\
      \hline
      AR6D26          & Deve haver alternativas a conteúdo de áudio (ex.\: transcrição ou linguagem de sinais).                                                      & AR6CM              \\
      \hline
      AR6D27          & Conteúdos visuais que não estão disponíveis como áudio devem ser descritos.                                                                  & AR6CM              \\
      \hline
      AR6D28          & Devem haver mecanismos para controlar áudios da aplicação.                                                                                   & AR6CM              \\
      \hline
      AR6D29          & Devem haver mecanismos para controlar animações que iniciam automaticamente.                                                                 & AR6CM              \\
      \hline
      AR6D30          & Botões de imagem ou conteúdos de áudio em formulários devem possuir alternativas textuais.                                                   & AR6CF              \\
      \hline
      AR6D31          & Todos os campos do formulário devem ser identificados.                                                                                       & AR6CF              \\
      \hline
      AR6D32          & Uma ordem lógica na navegação pelo formulário deve ser garantida.                                                                            & AR6CF              \\
      \hline
      AR6D33          & Não devem haver mudanças automáticas quando um elemento do formulário é focado, para não confundir ou desorientar o usuário.                 & AR6CF              \\
      \hline
      AR6D34          & Formulários devem possuir instruções de preenchimento.                                                                                       & AR6CF              \\
      \hline
      AR6D35          & Erros de entrada devem sempre ser descritos e as submissões de dados confirmadas.                                                            & AR6CF              \\
      % \hline
    \end{tabular}
    \legend{Fonte: \citeonline{Quispe2020}.}
  \end{center}
\end{table}
