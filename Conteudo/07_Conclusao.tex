\chapter{Considerações Finais}
\label{ch:conclusion}

Este trabalho buscou contextualizar e fundamentar a problemática identificada pela dificuldade
no acesso à informações sobre o autocuidado com o DM por pacientes com DV\@. Para isso, foram introduzidos
o DM e suas complicações relacionadas à DV, bem como, foram apresentados dados que indicam o crescimento no
número de casos de ambos.

Por meio de estudos anteriores, em um trabalho de mestrado que acarretou na parceira para desenvolvimento deste
projeto, foi possível um aprofundamento sobre as necessidades e requisitos do público-alvo.
Nos quais foram identificadas a importância do autocuidado no tratamento do DM e as principais funcionalidades utilizadas
como solução no mercado \cite{Sobral2021}.

Outra problemática introduzida foi que, mesmo com o aumento da informatização e popularização dos \emph{smartphones},
PDV ainda enfrentam sérias dificuldades devido à falta de acessibilidade à DV em aplicações móveis \cite{Shera2021285}.
Diante dessa problemática, foi realizado um processo de MSL visando identificar as principais soluções que estão sendo adotadas.
Além disso, foram apontadas as principais ferramentas e diretrizes relacionadas à acessibilidade disponibilizadas pelas
plataformas móveis.

Nesse processo de mapeamento, foram extraídas informações relevantes
de trabalhos publicados em bases acadêmicas que apresentaram técnicas para resolver esses problemas
em aplicações móveis.
A partir da análise dos resultados do MSL, estabeleceram-se as técnicas de acessibilidade que seriam utilizadas
na solução proposta.

Por fim, aliando os benefícios do autocuidado na redução das complicações ocasionadas pelo DM
\cite{ADA2019} ao crescimento no acesso à Internet por meio de \emph{smartphones} \cite{CETIC_2021}
e às principais técnicas para solução dos problemas de acessibilidade em \emph{apps} móveis,
propõe-se o desenvolvimento de uma aplicação com a combinação dessas características.