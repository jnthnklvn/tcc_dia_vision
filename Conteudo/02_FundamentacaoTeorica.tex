\chapter{Fundamentação Teórica}

Neste capítulo...

% ---
\section{Diabetes Mellitus}
% ---

De acordo com a Associação Americana de Diabetes (ADA), o Diabetes Mellitus (DM) é um grupo de doenças endocrinológicas crônicas
caracterizado pela elevação da glicose no sangue, devido à deficiência de ação do hormônio insulina, que requer cuidados médicos
contínuos para redução de risco e controle glicêmico \cite{ADA2019}.

O DM possui dois tipos, onde o tipo 1 afeta a produção de insulina devido à uma reação autoimune às proteínas das células das
ilhotas do pâncreas e o tipo 2 afeta o processamento do açúcar no sangue e é causado por fatores genéticos relacionados à
secreção prejudicada de insulina, resistência à insulina e fatores ambientais, como obesidade, alimentação excessiva, falta de
exercício, estresse e o envelhecimento \cite{Ozougwu_2013}.

A diabetes vem se tornando um desafio global de saúde pública cada vez maior por conta do rápido aumento no número de casos.
Estimativas da Federação Internacional de Diabetes (IDF), através do Atlas da Diabetes\footnote{\url{https://diabetesatlas.org/}}
de 2019, apontaram que 463 milhões de pessoas no mundo viviam com DM, o que representa cerca de 9.3\% da população
global adulta, e é esperado um aumento para 10,2\% (578 milhões) em 2030 e 10,9\% (700 milhões) em 2045 \cite{SAEEDI2019107843}.

O Brasil é o 5º país com mais diabéticos no mundo com 16,8 milhões em 2019, na faixa etária de 20 à 79 anos, e estimativas
de 21,5 e 26 milhões de casos para 2030 e 2045, respectivamente \cite{SAEEDI2019107843}. Os custos totais de hipertensão, diabetes
e obesidade no Sistema Único de Saúde (SUS) alcançaram 3,45 bilhões de reais em 2018, sendo 30\% desse custo relacionado ao DM \cite{Nilson2020}.

Já a retinopatia diabética é uma complicação vascular do diabetes, cuja prevalência está diretamente relacionada à duração
do diabetes e ao controle do nível de glicemia \cite{Solomon412}. Essa complicação é a maior causa de novos casos de cegueira
em adultos, na faixa etária de 20 à 74 anos, em países desenvolvidos \cite{ADA2019}. Além disso, outros distúrbios oculares
como o glaucoma e a catarata ocorrem mais cedo e com maior frequência em diabéticos \cite{ADA2019}.

Os resultados de uma metanálise realizada no estudo de \citeonline{AMINUDDIN2021103286} apontaram que as intervenções
de autogerenciamento baseadas em \emph{smartphones} pareceram ter efeitos benéficos sobre atividades de autocuidado para
pacientes com DM tipo 2.

% ---
\section{Deficiência visual}
% ---

De acordo com a Classificação Internacional de Funcionalidade, Incapacidade e Saúde (ICF),
a incapacidade enfrentada por pessoas com deficiência visual (PDV) não é determinada apenas
pela condição ocular, mas também pelo ambiente físico e social em que a pessoa vive, bem como
as dificuldades que pode enfrentar para realização de atividades como autocuidado, os problemas
que sofrem cotidianamente, como em ir para o trabalho ou escola, e o acesso a cuidados, produtos
e serviços oftalmológicos \cite{WHO2019}.

Assim, atualmente o mundo enfrenta um sério problema com relação a saúde da visão. Segundo a Organização Mundial da Saúde (OMS),
pelo menos 2,2 bilhões de pessoas no mundo vivem com deficiência visual (DV) em algum grau, com isso
a necessidade de cuidados com os olhos tende a crescer drasticamente nas próximas décadas \cite{WHO2019}.

O estudo da \citeonline{WHO2019} aponta que mais de 1 bilhão dos casos de pessoas com DV poderiam ser prevenidos ou
tratados. Ainda segundo esse estudo, os principais motivos para esses casos são:

\begin{itemize}
    \item O tempo despendido em ambientes fechados e aumento das atividades \textit{"near work"} (ler, escrever, assistir TV, jogar videogames, etc);
    \item O aumento no número de pessoas vivendo com diabetes, principalmente o tipo 2;
    \item Muitas pessoas não terem acesso a serviços oftalmológicos e verificações de rotina.
\end{itemize}

No Brasil, de acordo com o último censo do Instituto Brasileiro de Geografia e Estatistica (IBGE), realizado em 2010,
cerca de 18,6\% da população era afetada por algum tipo de DV, sendo 3,46\% por DV severa
\cite{IBGE2012}. Embora o próximo censo esteja previsto para 2022\footnote{\url{https://censo2022.ibge.gov.br/}}, outra pesquisa
foi realizada pelo Ministério da Saúde em 2019, a Pesquisa Nacional de Saúde (PNS), e apontou que 3,4\% da população brasileira,
com 2 ou mais anos de idade, possui muita dificuldade ou não enxerga \cite{stopa2020pesquisa}.

% ---
\section{Acessibilidade e Leitores de tela}
% ---

Segundo o Art. 3º da Lei Brasileira de Inclusão da Pessoa com Deficiência, acessibilidade se refere à:

\begin{citacao}
    possibilidade e condição de alcance para utilização, com segurança e autonomia, de espaços, mobiliários, equipamentos urbanos,
    edificações, transportes, informação e comunicação, \textbf{inclusive seus sistemas e tecnologias}, bem como de outros serviços e
    instalações abertos ao público, de uso público ou privados de uso coletivo, tanto na zona urbana como na rural, por pessoa com
    deficiência ou com mobilidade reduzida.
\end{citacao}

Acesso a Tecnologias Assistivas (AT) adequadas e de qualidade por um preço acessível melhora o funcionamento individual e
a independência, ao mesmo tempo que facilita a participação e integração na sociedade \cite{world2019global}.

Visando a inclusão das pessoas com DV, tecnologias conhecidas como Tecnologias Assistivas (TA) se tornam cada vez mais presentes.
\citeonline{Cook2014} utilizam em seu livro, uma definição de TA mundialmente utilizada que foi definida por uma \textit{Public Law}
dos Estados Unidos da América (EUA). Os autores justificam a utilização dessa definição por a mesma contemplar os pontos mais
importantes a respeito de TA, como diz a seguir:

\begin{citacao}
    Qualquer item, parte de equipamento ou sistema adquirido comercialmente, modificado ou customizado que é utilizado para aumentar, manter ou melhorar as capacidades
    funcionais de pessoas com deficiência \cite{Cook2014}.
\end{citacao}

Para que essas TAs funcionassem adequadamente, organizações como a \textit{World Wide Web Consortium} (W3C) definiram diretrizes que deveriam ser seguidas no desenvolvimento
de aplicações \textit{web} \cite{W3C2019}. Já para aplicações \textit{mobile}, como a implementação da tecnologia varia de acordo com o Sistema Operacional (SO), essa
definição se deu pelas próprias proprietárias dos SOs, tais como Google e Apple.

% ---
\section{Diretrizes de acessibilidade}
% ---

Um estudo realizado por \citeonline{Ballantyne2018}, compila um conjunto de diretrizes para acessibilidade \textit{mobile} e realiza
testes em 25 dos \textit{apps} mais populares da \emph{Google Play}. Os resultados do estudo revelaram que apenas 8 dos 25 selecionados
possuiam taxa de conformidade com as diretrizes acima de 75\%. O estudo ainda revela que 63\% das violações encontradas são
relacionadas ao \textit{design} (componentes de tela).

Já \citeonline{Yan2019} elaboram um estudo mais abrangente, realizado com 479
\textit{apps} de 23 categorias da \emph{Google Play}. Os autores utilizaram uma ferramenta automatizada, o IBM
\textit{Mobile Accessibility Checker} (MAC), para encontrar possíveis problemas com acessibilidade nesses \textit{apps},
categorizando-os em V (Violação), PV (Potêncial Violação) e A (Alerta). Os resultados mostraram
que 94.8\%, 97.5\% e 66.4\% dos apps continham problemas realacionados a V, PV e A, respectivamente \cite{Yan2019}.

Para \citeonline{Quispe2020} os principais fatores para a baixa priorização da acessibilidade de aplicações \textit{mobile}
são o desconhecimento, a alta demanda e a falta de tempo das equipes de desenvolvimento, fazendo com que se concentrem nos
requisitos funcionais em detrimento de requisitos não funcionais de usabilidade como o de acessibilidade.

% ---
\section{Desenvolvimento de aplicações multiplataforma}
% ---

Segundo  as estimativas, mais de 5 bilhões de pessoas possuem dispositivos móveis no mundo, sendo mais da metade destes, \textit{smartphones}.
No Brasil, a taxa de adultos que dizem possuir dispositivos móveis é de 83\% no total e 60\% para \textit{smartphones}. Na faixa etária entre 18
e 34 anos, houve um aumento no número de proprietários de \textit{smartphones} de 61\% em 2015 para 85\% em 2018 \cite{Taylor2019}. //CETIC.br, NIC-BR, ITU (ONU)
