\chapter{Fundamentação Teórica}
\label{ch:fundament}

Este capítulo apresenta os conceitos básicos relevantes para compreensão deste trabalho,
com o objetivo de contextualizar e situar o leitor no tema abordado.

% ---
\section{Diabetes Mellitus}
% ---

O DM é um grupo de doenças endocrinológicas crônicas
caracterizado pela elevação da glicose no sangue que ocorre devido à deficiência de ação do hormônio insulina, e requer cuidados médicos
contínuos para redução de risco e controle glicêmico \cite{ADA2019}.

Esse grupo divide-se em dois tipos: o tipo 1 afeta a produção de insulina devido a uma reação autoimune às proteínas das células das
ilhotas do pâncreas, enquanto que o tipo 2 afeta o processamento do açúcar no sangue, e é causado por fatores genéticos relacionados
à secreção prejudicada e resistência à insulina, além de fatores ambientais
como obesidade, alimentação excessiva, falta de exercício,
estresse e o envelhecimento \cite{Ozougwu_2013}.

Segundo a \citeonline{ADA2019}, estudos mostraram que DSMES estão associados ao aumento no conhecimento e comportamentos de
autocuidado sobre o diabetes e no número de auto-relatos de redução de peso, à melhoria de qualidade de vida, à redução de risco de mortalidade
em todas as causas e à redução nos custos de cuidados com a saúde.

Resultados mais recentes, de uma metanálise realizada em \citeonline{AMINUDDIN2021103286}, também apontaram que as intervenções
de autogerenciamento baseadas em \emph{smartphones} teriam efeitos benéficos sobre atividades de autocuidado, principalmente para
pacientes com DM tipo 2.

% ---
\section{Deficiência Visual}
% ---

De acordo com a Classificação Internacional de Funcionalidade, Incapacidade e Saúde (ICF),
a incapacidade enfrentada por pessoas com deficiência visual (PDV) não é determinada apenas
pela condição ocular, mas também pelo ambiente físico e social em que a pessoa vive, bem como
as dificuldades que pode enfrentar para realização de atividades de autocuidado, os problemas
que sofrem cotidianamente e o acesso a cuidados, produtos e serviços oftalmológicos \cite{WHO2019}.

Segundo a \citeonline{WHO2019}, mais de 1 bilhão dos casos de pessoas com deficiência visual (DV) poderiam ser prevenidos ou
tratados. A publicação ainda aponta que os principais motivos para esses casos são:

\begin{itemize}
    \item O tempo despendido em ambientes fechados e aumento das atividades \textit{``near work''} (ler, escrever, assistir TV, jogar videogames, etc);
    \item O aumento no número de pessoas vivendo com diabetes, principalmente o tipo 2;
    \item Muitas pessoas não terem acesso a serviços oftalmológicos e exames de rotina.
\end{itemize}

No Brasil, conforme o último censo do Instituto Brasileiro de Geografia e Estatistica (IBGE), realizado em 2010,
cerca de 18,6\% da população era afetada por algum tipo de DV, sendo 3,46\% por DV severa
\cite{IBGE2012}. Embora o próximo censo esteja previsto para 2022, outra pesquisa
foi realizada pelo Ministério da Saúde em 2019, a Pesquisa Nacional de Saúde (PNS), e apontou que 3,4\% da população brasileira,
com 2 ou mais anos de idade, possui muita dificuldade para enxergar ou não enxerga \cite{stopa2020pesquisa}.

% ---
\section{Acessibilidade e Tecnologia Assistiva}
% ---

Segundo o Art. 3º da Lei Brasileira de Inclusão da Pessoa com Deficiência,
acessibilidade refere-se à:

\begin{citacao}
    possibilidade e condição de alcance para utilização, com segurança e autonomia, de espaços, mobiliários, equipamentos urbanos,
    edificações, transportes, informação e comunicação, \textbf{inclusive seus sistemas e tecnologias}, bem como de outros serviços e
    instalações abertos ao público, de uso público ou privados de uso coletivo, tanto na zona urbana como na rural, por pessoa com
    deficiência ou com mobilidade reduzida \cite{Brasil2015}.
\end{citacao}

Assim, visando essa inclusão, tecnologias conhecidas como Tecnologias Assistivas (TA) se tornam cada vez mais presentes.
\citeonline{Cook2014} adotam em seu livro uma definição de TA criada por uma \textit{Public Law}
dos Estados Unidos da América (EUA) que é mundialmente utilizada, pois ela contempla os pontos mais
importantes a respeito de TA, como diz a seguir:

\begin{citacao}
    Qualquer item, parte de equipamento ou \textbf{sistema} adquirido comercialmente, modificado ou customizado que é utilizado para aumentar, manter ou melhorar as capacidades
    funcionais de pessoas com deficiência \cite{Cook2014}.
\end{citacao}

De acordo com a OMS, o acesso à TAs adequadas e de qualidade por um preço acessível melhora
no desenvolvimento das funções e na independência de indivíduos com deficiências, ao mesmo tempo que facilita
a participação e integração dos mesmos na sociedade \cite{world2019global}.

% ---
\section{Diretrizes de Acessibilidade}
% ---

Para que as TAs funcionassem adequadamente na \emph{web}, a \textit{World Wide Web Consortium} (W3C) definiu, por meio da
\emph{Web Accessibility Initiative} (WAI), um conjunto de diretrizes e recomendações de acessibilidade, chamado
\emph{Web Content Accessibility Guidelines} (WCAG) que deveriam ser seguidas no desenvolvimento de aplicações \emph{web}
\cite{W3C2019}.

Com o advento da navegação na \emph{web} em dispositivos móveis, a W3C lançou, em novembro
de 2006, o \emph{Mobile Web Best Practices}(MWBP)\footnote{O Guia de Boas Práticas
em Web Móvel reune padrões web do W3C para ajudar no desenvolvimento de conteúdos web para
que funcionem adequadamente em dispositivos móveis. \url{http://www.w3.org/TR/mobile-bp/}}, com o objetivo de
melhorar a experiência do usuário ao acessar a \emph{web} nesses dispositivos.

Além da W3C, órgãos de governos também desenvolveram diretrizes e recomendações de acessibilidade, baseados no WCAG e
em suas próprias legislações, como foi o caso dos EUA com a Seção 508
da Lei de Reabilitação de 1973, que exige acessibilidade de todas as agências federais quando desenvolvem,
adquirem, mantêm ou usam tecnologia eletrônica e de informação \cite{JAEGER2006169}. Contudo, estudos apontam que diversos
sites governamentais não estão em total conformidade com a Seção 508 \cite{KING2016715,YI201575}.

Já no Brasil, o Decreto 5.296/04 estabeleceu o cumprimento dos requisitos de acessibilidade pelos
órgãos da administração pública direta, indireta e fundacional, pelas empresas prestadoras de
serviços públicos e instituições financeiras, resultando na criação do Modelo de Acessibilidade
em Governo Eletrônico (eMAG) \cite{EMAG_2007}.

Lançada em 2005, a primeira versão do eMAG também foi desenvolvida com base no WCAG 1.0. Estudos avaliaram as diferenças entre as diretrizes do eMAG
(versões 1.0 e 3.0) e as do WCAG (versões 1.0 e 2.0), observando-se que as versões possuiam poucas
diferenças nas recomendações, sendo as maiores diferenças estruturais, demonstrando vantagens na adoção do eMAG para o
contexto brasileiro \cite{Bach_2009,Rocha_2013}.

Entretanto, para aplicações nativas \textit{mobile}, não existe uma entidade como a W3C para definir essas diretrizes e, embora
ela tenha proposto o MWBP, este refere-se à criação de aplicações \emph{web} para o \emph{mobile}, não contemplando todas
as necessidades das aplicações nativas \cite{W3C_2008}.

Assim, as próprias empresas responsáveis pelos Sistemas Operacionais (SO) \emph{mobile}, Google (Android) e Apple (iOS),
criaram suas diretrizes e recomendações para o desenvolvimento de aplicativos acessíveis para suas plataformas.
Apesar disso, muitos desenvolvedores não possuem conhecimento sobre essas técnicas e recomendações, ou sobre a necessidade delas
para o suporte das aplicações à pessoas com deficiências \cite{Quispe2020,Bi2021}.

Um estudo realizado por \citeonline{Ballantyne2018} compila um conjunto de diretrizes para acessibilidade \textit{mobile} e realiza
testes em 25 dos \textit{apps} mais populares da Google Play. Os resultados do estudo revelaram que apenas 8 dos 25 selecionados
possuiam taxa de conformidade com as diretrizes acima de 75\%. O estudo ainda revela que 63\% das violações encontradas estavam
relacionadas ao \textit{design} (componentes de tela).

Para \citeonline{Quispe2020} os principais fatores para a baixa priorização da acessibilidade de aplicações \textit{mobile}
são o desconhecimento, a alta demanda e a falta de tempo das equipes de desenvolvimento, fazendo com que se concentrem nos
requisitos funcionais em detrimento de requisitos não funcionais de usabilidade como o de acessibilidade.

% ---
\section{Ferramentas Relacionadas à Acessibilidade}
% ---
Nesta seção são apresentadas ferramentas que podem ser utilizadas durante o desenvolvimento do projeto, com o objetivo de melhorar e validar a usabilidade
da aplicação quanto à acessibilidade.

% ---
\subsection{Leitores de tela}
% ---

Com o objetivo de possibilitar a utilização dos \emph{smartphones} por usuários cegos e auxiliar os com DV parcial, o Android e o iOS
fornecem nativamente os leitores de tela chamados TalkBack e VoiceOver, respectivamente.

A descrição do VoiceOver pela Apple:

\begin{citacao}
    Com o VoiceOver – um leitor de tela baseado em gestos – você pode usar o iPhone mesmo que você não possa ver a tela.
    O VoiceOver fornece descrições audíveis do que está na tela — desde o nível da bateria até quem está ligando e em qual
    app o seu dedo está. Você também pode ajustar a velocidade da fala e o tom de voz conforme as suas necessidades \cite{VoiceOver2021}.
\end{citacao}

Descrição do TalkBack pelo Google:

\begin{citacao}
    O TalkBack é o leitor de tela do Google incluído em dispositivos Android.
    Ele permite que você controle o dispositivo sem usar os olhos \cite{TalkBak2021}.
\end{citacao}

Enquanto apenas a Apple fabrica os \emph{smartphones} que utilizam o iOS, o Google mantém o Android, mas cada fabricante
pode escolher se deseja personalizá-lo com as características da marca ou se mantém o sistema fiel as definições padrões. Diante disso,
a documentação ressalta:

\begin{citacao}
    A configuração depende do fabricante do dispositivo, da versão do Android e da versão do TalkBack.
    Estas páginas de ajuda se aplicam à maioria dos dispositivos, mas pode haver algumas diferenças \cite{TalkBak2021}.
\end{citacao}

Esses leitores de tela utilizam uma técnica conhecida como \emph{text-to-speech} (TTS, ``texto para discurso'' em tradução livre)
para narrar as descrições para os usuários. Dessa forma, tanto o Android quanto o iOS disponibilizam Interfaces de Programação de Aplicações
(APIs, do inglês ``\emph{Application Programming Interface}'') para que os desenvolvedores possam integrar essa tecnologia diretamente aos
próprios \emph{apps}, não sendo necessário habilitar os leitores nos dispositivos para funcionar nessas aplicações específicas
\cite{Heesook2017,Biase2018,Oliveira2019,Caballero2020}.

Para o funcionamento adequado desses leitores de tela nas aplicações, é necessário que haja um tratamento
com relação aos componentes de \emph{interface}, com o fornecimento de descrições que possam ser narradas para o usuário. Por conta disso,
problemas associados à falta de descrições desses componentes costumam ser os mais frequentes enfrentados pelos usuários com DV
\cite{Vendome201941,Christoph2020,Shera2021285}.

Uma boa prática para mitigação desses problemas pode ser a utilização constante dos leitores de tela para validação do fluxo das aplicações durante
o processo de desenvolvimento, visando diminuir a quantidade, visto que podem ser identificados e corrigidos mais cedo \cite{Tomlinson2016377}.

% ---
\subsection{Testes Automatizados}
% ---

A utilização de ferramentas automatizadas para realização de testes de usabilidade e acessibilidade pode reduzir o esforço
e o retrabalho, pois identifica diversos problemas ainda em tempo de desenvolvimento, diminuindo os esforços e custos para
realização das correções \cite{Christoph2020}.

Uma dessas ferramentas é a \emph{Mobile Accessibility Testing} (MATE), desenvolvida por brasileiros, que automaticamente explora os \emph{apps}
aplicando diferentes tipos de checagem por problemas de acessibilidade relacionadas à DV, gerando um relatório detalhado para auxiliar
os desenvolvedores na resolução desses problemas \cite{Eler2018AutomatedAT}.

O Google também possui duas ferramentas que permitem a realização desses testes, o \emph{Accessibility Scanner}, uma aplicação instalável no
dispositivo, disponível na Google Play,
que sobrepõe o \emph{app} a ser testado e faz sugestões de melhorias de acessibilidade, e o Test Lab, no qual é possível realizar o \emph{upload}
do \emph{app} para realização dos testes pela ferramenta, disponível no Firebase\footnote{O Firebase é uma plataforma do Google
que oferece serviços de backend, monitoramento e engajamento que facilitam o processo de desenvolvimento
de aplicações móveis e web. \url{https://firebase.google.com/}}.

A ferramenta MAC da IBM possibilita a realização de testes automatizados em aplicações
móveis nativas e em conteúdos \emph{web mobile}, gerando alertas sobre problemas de acessibilidade com recomendações de
correções baseadas em diretrizes padrões da industria e regulamentações governamentais \cite{patil2016enhanced,Yan2019}.

% ---
\section{Desenvolvimento de aplicações móveis multiplataforma}
% ---

Conforme pesquisa realizada pelo Centro de Estudos sobre
as Tecnologias da Informação e da Comunicação (CETIC), 98\% dos usuários de Internet brasileiros, com 16 anos ou mais,
utilizavam telefone celular para acessar a Internet em 2020, essa taxa foi de 97\% em 2018 e 99\% em 2019 \cite{CETIC_2021}.
A pesquisa ainda mostrou que 40\% desses usuários buscaram por informações ou realizaram serviços públicos \emph{online} relacionados
aos direitos do trabalhador ou previdência social em 2019, e esse número aumentou para 72\% em
2020, durante a pandemia de COVID-19 \cite{CETIC_2021}.

Quanto à realização de atividades remotas, 87\% dos usuários que declararam frequentar escola ou universidade, no momento da coleta
dos dados dessa pesquisa, afirmaram que a instituição na qual estudaram ofereceu atividades educacionais remotas, com o telefone
celular sendo o dispositivo mais utilizado para o acompanhamento dessas atividades pelas classes D e E \cite{CETIC_2021}. Os resultados
dessa pesquisa também mostraram que 84\% dos usuários das classes D e E utilizaram principalmente o celular para realização de atividades
profissionais de forma remota durante a pandemia.

Embora aplicativos móveis possam ser desenvolvidos até por desenvolvedores amadores, plataformas móveis são muito complexas, tanto o
SO Android quanto iOS, os principais para \emph{smartphones} da atualidade, contém mais de 12 milhões de linhas de código
(LOC, do inglês ``\emph{lines of code}'') \cite{pressman2014software}.

O alto número de \emph{apps} disponíveis faz com que o processo de desenvolvimento de \emph{software} para esses dispositivos pareça
ter sido bem compreendido, porém ainda existe um grande número de questões que precisam ser resolvidas \cite{pressman2014software,Wasserman2010}.

Os requisitos não funcionais de aplicações móveis como a usabilidade, por exemplo, são diferentes de aplicações \emph{web} ou \emph{desktop} \cite{pressman2014software}. 
Existe um ``ângulo'' \emph{mobile} para praticamente todo aspecto de engenharia de \emph{software}, em que as caracteristicas
das aplicações e seus SOs apresentam um novo ou diferente conjunto de questões que precisam ser consideradas \cite{Wasserman2010}.

Como mencionado por \citeonline{pressman2014software}, diferentes dispositivos móveis utilizam diferentes SOs e, consequentemente, diferentes
ambientes e ferramentas de desenvolvimento para cada plataforma, destacando a importância de se considerar portabilidade ao desenvolver aplicações
para esses dispositivos.

Devido à tais questões, o tempo e o esforço de desenvolvimento necessários para oferecer suporte à múltiplas plataformas aumenta
significativamente, elevando os custos do projeto \cite{Henning2013,Wasserman2010}. Nesse sentido, diversos kits de ferramentas de
desenvolvimento (mais conhecidos pelo termo em inglês: \emph{framework}), foram propostos, visando simplificar esse processo, diminuindo os
custos de desenvolvimento e manutenção \cite{Martinez2017,Francese2015}.

\subsection{\emph{Flutter}}

Com a primeira versão estável sendo lançada em dezembro de 2018, o Flutter é um kit de ferramentas de Interface
de Usuário (UI, do inglês \emph{User Interface}) de código aberto criado pelo Google para construção de aplicações compiladas nativamente
para multiplas plataformas, tais como \emph{mobile} (Android e iOS), \emph{web} e \emph{desktop} (Linux, MacOS e Windows) \cite{kuzmin2020experience}.

O \emph{framework} foi construído com base na linguagem de programação Dart, também criado pelo Google e de código aberto.
A linguagem foi lançada em 2011 com o objetivo inicial de ``substituir'' o JavaScript no desenvolvimento \emph{web},
com diferencias como mecanismos de abstrações e semântica mais clara, visando a coesão e elegância de código e
rodando tanto no lado do cliente (navegadores) quanto no lado servidor
(\emph{backend}) \cite{walrath2012dart}.

Um dos diferencias do Flutter com relação a maioria dos \emph{frameworks} para desenvolvimento multiplataforma é que ele renderiza os próprios
componentes de \emph{interface}, sem utilizar os componentes nativos de cada plataforma como os demais, ou seja, quando você constroi um botão com Flutter,
ele mesmo renderiza, sem a necessidade de uma ``ponte'' com os SOs para solicitar a renderização do componente nativo \cite{zammetti2019practical,boukhary2019clean}.

O motor (do inglês, \emph{engine}) do Flutter é baseado, em sua maior parte, no C++, linguagem nativa utilizada pelos SOs (Android, iOS, Windows etc), assim,
possibilitando alcançar desempenho próximo ao de aplicações nativas \cite{zammetti2019practical,kuzmin2020experience}. Já o motor gráfico utilizado por essa
base de código é o Skia, uma compacta biblioteca gráfica de código aberto que apresenta excelente desempenho nas plataformas
suportadas \cite{zammetti2019practical,boukhary2019clean}.

O Flutter também fornece uma \emph{interface} sobre os kits de desenvolvimento de \emph{software} (SDKs, do inglês ``\emph{Software Development Kit}'')
nativos de ambas as plataformas (Android e iOS), esta que tem como objetivo eliminar as diferenças entre as APIs nativas de cada plataforma \cite{zammetti2019practical}.
Dessa forma, não é necessário que os desenvolvedores se preocupem com o funcionamento nativo das APIs como a de câmera de cada plataforma, podendo utilizar
a do Flutter diretamente, pois ela abstrairá toda essa parte, ficando responsável por fazer a chamada à API adequada \cite{zammetti2019practical}.

No trabalho de mestrado de \citeonline{gonsalves2019evaluating}, um protótipo de aplicação móvel foi desenvolvido para Android e iOS nativamente, e com os
\emph{framework} de desenvolvimento multiplataforma Flutter e Cordova, visando realizar um comparativo de
aspectos do processo de desenvolvimento e dos resultados das aplicações. 

Os resultados do estudo mostraram que, no geral, o Flutter oferece uma melhor experiência de desenvolvimento que o Cordova e as plataformas nativas
(considerando desenvolver para iOS e Android). Ambos os \emph{frameworks} multiplataforma apresentaram menos LOC que os \emph{apps} nativos separados,
sendo pouco mais de 1/3 do total de LOC dos \emph{apps} nativos \cite{gonsalves2019evaluating}.

Um ponto negativo do Flutter foi o tamanho significativamente maior do \emph{app} em \emph{megabytes} (MB)
e maior consumo de RAM\footnote{A memória de Acesso Randômico (do inglês \emph{Random Access Memory}) é onde
ficam armazenados os programas básicos operacionais de sistemas eletrônicos digitais.}
que os \emph{apps} em Cordova e nativos, porém o Flutter também mostrou maior velocidade na inicialização do \emph{app} e na navegação entre as telas
que os demais \cite{gonsalves2019evaluating}. O maior tamanho e consumo de RAM no \emph{app} desenvolvido com Flutter dá-se pela inclusão do seu motor,
das bibliotecas utilizadas e de outros recursos no \emph{app} \cite{gonsalves2019evaluating,zammetti2019practical}.

\section{Considerações Finais}

Este capítulo buscou trazer elucidações sobre o DM e a relação com DV, acessibilidade, diretrizes e ferramentas,
finalizando sobre o desenvolvimento de aplicativos móveis multiplataforma.
O próximo capítulo abordará o processo de revisão da literatura, no qual foi realizado um levantamento das principais soluções
utilizadas na implementação de acessibilidade em aplicações móveis.