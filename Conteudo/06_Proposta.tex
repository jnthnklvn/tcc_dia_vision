\chapter{Proposta e Plano de Continuidade}

Neste capítulo é apresentada a proposta deste trabalho, com os requisitos, estórias de usuário, casos de uso e demais artefatos
que fazem parte do levantamento de requisitos e análise, oferecendo uma visão ampla da aplicação que será desenvolvida.
Além disso, também é apresentado o plano de continuidade com o cronograma para realização das etapas da segunda parte deste
trabalho.

Para o desenvolvimento completo desse projeto é previsto a realização de 4 etapas, ao longo dos meses referentes ao primeiro
período letivo de 2021 da Universidade Federal de Sergipe (UFS), estas são listadas a seguir:
\begin{itemize}
    \item Desenvolvimento do aplicativo com a implementação dos requisitos mais essenciais e viáveis;
    \item Revisão e testes da acessibilidade da aplicação;
    \item Descrição do processo metodológico e análise dos resultados;
    \item Revisão, correção e entrega do trabalho.
\end{itemize}

Este projeto está sendo desenvolvido em parceria com a mestranda, Débora Almeida Silveira Sobral, do
Programa de Pós-graduação Profissional em Gestão e Inovação Tecnológica em Saúde (PPGITS), também orientanda da Profa.
Dra. Adicinéia Aparecida de Oliveira. E tem como objetivo
o desenvolvimento de uma aplicação móvel como ferramenta de auxilio à educação e ao autocuidado de pacientes diabéticos
com acuidade visual prejudica, acessível à PDV\@.

Para tanto, Débora buscou, em seu trabalho, realizar um levantamento de referencial teórico e tecnológico sobre o DM,
aplicativos móveis e deficiência visual. Assim, reunindo as principais funcionalidades e soluções que o aplicativo a ser
desenvolvido deveria adotar como requisitos para que atendesse às necessidades desse público-alvo.

\newpage

\section{Busca de Anterioridade}

Em seu trabalho, Débora realizou uma busca de anterioridade, na base do Instituto Nacional de Propriedade Industrial (INPI)
e nas lojas de aplicativos Google Play (Android) e Apple Store (iOS), visando identificar os \emph{softwares} e funcionalidades
já existentes sobre DM e DV no mercado.

A \autoref{fig_tab_cor_func} exibe uma tabela onde a autora buscou relacionar os \emph{apps} encontrados
nas lojas de aplicativos às principais funcionalidades propostas em seu trabalho para o DiaVision.

\begin{figure}[htb]
    \caption{\label{fig_tab_cor_func}Relação de funcionalidades dos \emph{apps} encontrados nas lojas de aplicativos.}
    \begin{center}
        \includegraphics[scale=0.8]{Imagens/proposta/busca_anterioridade.png}
    \end{center}
    \legend{Fonte: Débora Sobral}
\end{figure}

\newpage

\section{Visão e Análise}

Esta seção resume as necessidades e características esperadas do produto de \emph{software} a ser desenvolvido, identificadas a partir de
reuniões com a \emph{product owner}, Débora Sobral, que identificou a problemática abordada neste trabalho e realizou o levantamento de
funcionalidades e problemas das soluções já existentes no mercado em seu trabalho de mestrado.

Assim, a partir dessas reuniões, o problema que deveria ser resolvido pela aplicação ficou
definido como: a dificuldade de acesso à informações de autocuidado com relação ao DM por deficientes visuais.
Com o objetivo de fornecer conteúdos que auxiliem os diabéticos portadores de deficiências visuais no gerenciamento
do autocuidado com DM\@.

Sendo a acessibilidade ao deficiente visual o principal diferencial da aplicação, já que, como pode ser visto na \autoref{fig_tab_cor_func},
apenas um \emph{app} demonstrou possuir essa característica.

\subsection{Riscos e Impedimentos}

Os seguintes riscos e possíveis impedimentos com relação ao produto foram identificados:

\begin{itemize}
    \item Não adesão por parte do público alvo;
    \item Dificuldades no manuseio do \emph{smartphone} pelo público alvo;
    \item Dificuldade de localizar os possíveis participantes da pesquisa;
    \item Utilização incorreta do aplicativo ou não assimilação das informações adquiridas;
    \item Afastamento do paciente da assistência continuada na rede primária;
    \item Constrangimento do usuário por falta de entendimento das funcionalidades.
\end{itemize}

\newpage

\subsection{Estórias dos usuários}

Também a partir das reuniões com a \emph{product owner}, para o processo de definição e levantamento de requisitos, foram identificadas
as estórias dos usuários que são listadas na \autoref{tab-est-usr}.

\begin{table}[htb]
    \begin{center}
        \ABNTEXfontereduzida
        \caption{Relação de estórias dos usuários.}
        \label{tab-est-usr}
        \begin{tabular}{p{2.0cm}|p{5.0cm}|p{7.0cm}}
            %\hline
            \textbf{Eu, enquanto}                             & \textbf{Quero} & \textbf{Para}           \\
            \hline
            Usuário                                           &
            Encontrar o \emph{app} nas lojas virtuais         &
            Baixar o \emph{app} no meu celular                                                           \\
            \hline
            Usuário                                           &
            Realizar cadastro no aplicativo                   &
            Ter acesso às funcionalidades do \emph{app}                                                  \\
            \hline
            Usuário                                           &
            Realizar login de forma prática                   &
            Para entrar no \emph{app}                                                                    \\
            \hline
            Usuário                                           &
            Poder resetar minha senha                         &
            Caso esqueça, possa altera-la e recuperar acesso ao \emph{app}                               \\
            \hline
            Usuário                                           &
            Ter acesso ao menu principal                      &
            Para escolher qual funcionalidade irei utilizar                                              \\
            \hline
            Usuário                                           &
            Registrar informações das refeições               &
            Acompanhar a quantidade de calorias consumidas por refeição                                  \\
            \hline
            Usuário                                           &
            Ter acesso a aplicativos para deficientes visuais &
            Ajudar a realizar atividades do dia a dia                                                    \\
            \hline
            Usuário                                           &
            Sugerir aplicativos para deficientes visuais      &
            Ajudar a realizar atividades do dia a dia                                                    \\
            \hline
            Usuário                                           &
            Registrar práticas de atividade física            &
            Acompanhar a evolução da rotina de atividade física                                          \\
            \hline
            Usuário                                           &
            Ter acesso a dicas de autocuidado                 &
            Melhorar a qualidade de vida e prevenir complicação do diabetes mellitus                     \\
            \hline
            Usuário                                           &
            Filtrar as dicas por categorias                   &
            Facilitar a busca das dicas sobre assuntos específicos                                       \\
            \hline
            Usuário                                           &
            Consultar locais para acesso à serviços de saúde  &
            Facilitar o acesso e contato com as principais clínicas, hospitais e consultórios da cidade  \\
            \hline
            Usuário                                           &
            Registrar glicemia                                &
            Acompanhamento dos valores de glicemia e ser alertado quando estiver fora do limite desejado \\
            \hline
            Usuário                                           &
            Registrar medicações que faço uso                 &
            Receber alertas com horários para uso da medicação                                           \\
            \hline
            Usuário                                           &
            Registrar medicações que faço uso                 &
            Ter uma lista atualizada com todas as informações das medicações de uso frequente            \\
            \hline
            Usuário                                           &
            Realizar avaliação dos pés                        &
            Acompanhar a evolução dos pés e detectar quando surgir alterações                            \\
            \hline
            Usuário                                           &
            Registrar diurese diária                          &
            Acompanhar a evolução e característica da diurese e detectar quando surgir alterações        \\
            \hline
            Usuário                                           &
            Ter acesso a relatórios dos registros feitos      &
            Visualizar e compartilhar esses dados registrados                                            \\
            \hline
            Usuário                                           &
            Ter acesso aos dados pessoais                     &
            Editar ou acrescentar dados pessoais durante o uso do aplicativo                             \\
            \hline
            Usuário                                           &
            Configurar preferências                           &
            Definir quais notificações ativar ou desativar                                               \\
            \hline
            Usuário                                           &
            Configurar preferências                           &
            Personalizar os limites da glicemia                                                          \\
            \hline
            Usuário                                           &
            Realizar logout                                   &
            Para desvincular minha conta do \emph{app}                                                   \\
            \hline
            Administrador do sistema                          &
            Adicionar dicas de autocuidado para os pacientes  &
            Fornecer conhecimento para o paciente acerca de cuidados com a saúde.                        \\
            \hline
            Administrador do sistema                          &
            Cadastrar centros de saúde no sistema.            &
            Que o paciente possa conhecer os centros de saúde que atendem suas demandas.                 \\
            \hline
            Administrador do sistema                          &
            Avaliar as sugestões os centros de saúde          &
            Assegurar credibilidade ao aplicativo                                                        \\
            % \hline
        \end{tabular}
        \legend{Fonte: Autores}
    \end{center}
\end{table}

\newpage

\subsection{Casos de Uso}

A partir dos requisitos e estórias dos usuários identificados, o diagrama de casos de usos da \autoref{fig_use_cas} foi elaborado.

\begin{figure}[htb]
    \caption{\label{fig_use_cas}Diagrama de casos de uso.}
    \begin{center}
        \includegraphics[scale=0.65]{Imagens/proposta/use_case.jpg}
    \end{center}
    \legend{Fonte: Autor}
\end{figure}

\newpage

\section{Cronograma}

Por fim, o cronograma na \autoref{fig_cro_con} foi definido visando a organização e planejamento das atividades que serão realizadas
na segunda parte desse trabalho.

\begin{figure}[htb]
    \caption{\label{fig_cro_con}Cronograma de continuidade do Projeto.}
    \begin{center}
        \includegraphics[scale=0.65]{Imagens/proposta/cronograma_continuidade.png}
    \end{center}
    \legend{Fonte: Autor}
\end{figure}