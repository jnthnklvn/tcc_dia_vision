% !TeX root = ..\Modelo-TCC-DCOMP.tex
\newpage{}

\section{Resultados Encontrados}

Nesta seção, são apresentados os resumos com as principais características, relacionadas ao tema deste trabalho, dos artigos selecionados na fase de extração, visando encontrar respostas para as questões levantadas na definição do protocolo de MSL\@.

\subsection{\emph{A Mobile Educational Game Accessible to All, Including Screen Reading Users on a Touch-Screen Device}}

O estudo realizado por \citeonline{Leporini2017} teve o objetivo levantar informações e possíveis soluções para as dificuldades levantadas por um grupo composto por 6 pessoas cegas ao responder questões de tarefas interativas.
E investigou, através de tarefas interativas como exercícios e questionários, a acessibilidade e usabilidade de gestos e leitores de tela em dispositivos móveis com \emph{touch-screen}.

No artigo é apresentado um \emph{game mobile} que envolveu duas pessoas cegas com experiência na utilização de \emph{smartphones} na fase inicial do planejamento do protótipo.
O \emph{game} funciona como se fosse um ``sistema solar'' com oito planetas, onde cada planeta representa um conjunto de questões e exercícios.
O jogador recebe determinada pontuação cada vez que joga de acordo com os acertos e erros durante o game.
As principais funcionalidades do \emph{app} relativas à acessibilidade identificadas foram:

\begin{enumerate}
\item Contraste de cor para garantir diferentes níveis de acessibilidade;
\item Apresentações de conteúdos de forma auditiva e visual;
\item Interação via gestos ou toques;
\item Suporte auditivo com descrições dos elementos.
\end{enumerate}

Através da avaliação desse protótipo, por cegos, o estudo investigou o suporte de acessibilidade \emph{mobile} multiplataforma do conjunto de especificações técnicas, \emph{WAI-Aria}\footnote{\url{https://www.w3.org/WAI/standards-guidelines/aria/}}, observando problemas na detecção de elementos, devido às suas posições na tela e conteúdos difíceis de identificar na interação com leitores de tela.
Notando também que houve alguma dificuldade por conta de gestos implementados no \emph{app} diferirem dos habituais utilizados pelos usuários no \emph{VoiceOver} do \emph{iOS}.

Apesar dos problemas encontrados, o artigo aponta que o \emph{feedback} foi positivo e os resultados mostraram que os exercícios puderam ser realizados facilmente, por pessoas cegas, através de simples gestos com auxilio dos leitores de tela.

\textbf{Tecnologia utilizada para desenvolvimento:} \emph{Cordova Framework}.

\textbf{Plataforma alvo do \emph{app} desenvolvido:} multiplataforma (\emph{Android} e \emph{iOS}).

\textbf{Público alvo da aplicação:} PDV\@.

\subsection{\emph{A Model-Driven Approach to Cross-Platform Development of Accessible Business Apps}}

Um procedimento comum no processo de desenvolvimento de \emph{software} é considerar a acessibilidade para PDV apenas na etapa final.
Além disso, muitos desenvolvedores não estão cientes de técnicas de software para atender esse grupo, pois o domínio de apps móveis multiplataforma tem recebido uma atenção limitada por pesquisadores.
Foi nesse sentido, que o estudo de \citeonline{Christoph2020} buscou identificar desafios, requisitos e soluções técnicas de acessibilidade, selecionando 28 requisitos a respeito de acessibilidade para aplicações móveis através de uma RSL\@.

O artigo apresenta uma abordagem orientada a modelos que integra conceitos de acessibilidade no desenvolvimento de aplicações móveis multiplataforma em conjunto com protótipos acessíveis à PDV, construídos com base nessa abordagem.
Uma aplicação com foco nos cidadãos que desejam obter informações sobre chuvas fortes e inundações foi desenvolvida, nela os usuários podem ter uma visão de eventos de inundações próximos e compartilhar novos incidentes.

O estudo comparou uma versão da aplicação desenvolvida nativamente que necessitou de 3,400 linhas de código \emph{Java} e 3,200 linhas de código \emph{XML} (gerado de forma semiautomática) com outra versão, com um conjunto similar de funcionalidades.
A nova versão do \emph{app} consistiu em 445 linhas de código \emph{MD²}, \emph{framework} baseado na abordagem orientada a modelos para desenvolvimento móvel multiplataforma através da linguagem de alto nível \emph{Xtend}\footnote{\url{https://www.eclipse.org/xtend/}}.
Principais funcionalidades sobre acessibilidade identificadas:

\begin{enumerate}
\item Adaptação da \emph{interface} de acordo com as necessidades do usuário;
\item Integração com os leitores de tela através do fornecimento de descrições em texto para elementos não textuais;
\item Personalização do contorno de foco do \emph{TalkBack}.
\end{enumerate}

Segundo o artigo, o estudo de caso mostrou que \emph{apps} acessíveis podem ser gerados a partir do modelo de alto nível \emph{MD²}, implementando as técnicas de integração adequadas em cada ponto.
Embora o autor afirme isso, o estudo também deixa claro que ainda havia uma pendência de validação centrada no usuário, visto que o trabalho não implementou todas as técnicas e a solução proposta não foi testada com PDV\@.

\textbf{Tecnologia utilizada para desenvolvimento:} \emph{Xtend, Java} e \emph{Eclipse}.

\textbf{Plataforma alvo do \emph{app} desenvolvido:} multiplataforma (\emph{Android} e \emph{iOS}).

\textbf{Público alvo da aplicação:} PDV interessadas em saber sobre eventos climáticos locais como chuvas fortes e inundações.

\subsection{\emph{An Accessible Roller Coaster Simulator for Touchscreen Devices: An Educational Game for the Visually Impaired}}

O trabalho de \citeonline{Biase2018} apresenta um \emph{app} simulador de montanha russa, baseado em simuladores educacionais já existentes e adaptado para \emph{smartphones}, para ser utilizado em disciplinas Educação Física por pessoas com e sem DV\@.
A aplicação foi desenvolvida para auxiliar no estudo de Energia Mecânica e trás as interações por áudio e tátil como alternativas à visual.
As principais funcionalidades sobre acessibilidade identificadas no \emph{app} foram:

\begin{enumerate}
\item Os elementos visuais possuem descrições textuais para integração com leitores de tela;
\item \emph{Feedback} através de ``texto para voz'' (TTS, do inglês \emph{text-to-speech}) e vibração ao clicar em determinados elementos na tela, mesmo com o modo de acessibilidade desativado;
\item Efeitos sonoros característicos que ilustram os resultados da simulação ao longo do percurso.
\end{enumerate}

Com taxas de 73\% eficácia, 77\% de eficiência e 66\% satisfação do usuário com relação a aplicação desenvolvida, os testes de usabilidade demonstraram que as estratégias de interação propostas são viáveis, com grande potencial para serem utilizadas em propósitos educacionais.

Alguns problemas de acessibilidade afetaram a taxa de satisfação dos usuários, a mantendo em 66\%, tais como dificuldades em seguir a trilha da montanha com apenas um dedo, não ser possível detectar quando o carro está voltando no trilho e falha no comando que altera o foco dos elementos, alterando para o elemento errado.

\textbf{Tecnologia utilizada para desenvolvimento:} \emph{Unity 3D engine}.

\textbf{Plataforma alvo do \emph{app} desenvolvido:} \emph{Android}.

\textbf{Público alvo da aplicação:} Pessoas com e sem DV\@.

\subsection{\emph{Application for the Configuration and Adaptation of the Android Operating System for the Visually Impaired}}

Apesar das vantagens dos dispositivos móveis, alguns desafios da interação de PDV com os sistemas operacionais (SOs) desses dispositivos precisam ser superados, para que a tecnologia alcance um número significativo nesse grupo.
Assim, o estudo de \citeonline{Oliveira2018} visou planejar e desenvolver uma aplicação que automatize as configurações do SO \emph{Android} de acordo com as preferências de acessibilidade de cada PDV, através de comandos de voz.
O artigo apresenta algumas funcionalidades e técnicas relacionadas a acessibilidade que são listadas a seguir:

\begin{enumerate}
\item Escala de Usabilidade do Sistema (SUS, do inglês \emph{System Usability Scale}) para avaliação de usabilidade da aplicação;
\item \emph{SpeechRecognizer} do \emph{Android} para reconhecimento de voz;
\item Eurísticas de Usabilidade de Nielsen (do inglês, \emph{Nielsen Usability Heuristics}) para evitar problemas de acessibilidade já mapeados.
\end{enumerate}

Um protótipo foi desenvolvido e mostrou potencial para ser utilizado como ferramenta para PDV, trazendo benefícios com a possibilidade do uso de comando de voz.
Os testes foram realizados com seis voluntárias com DV, sendo duas parcial e quatro total.
Onde três delas já possuíam experiência com comandos de voz e apenas duas das seis pessoas já haviam realizado a configuração do dispositivo alguma vez.

As voluntárias expressaram avaliações positivas quanto a autonomia, satisfação e usabilidade da aplicação.
E o tempo gasto para realizar as configurações de acessibilidade foi mais curto no \emph{app} desenvolvido que na aplicação padrão do \emph{Android}.

\textbf{Tecnologia utilizada para desenvolvimento:} \emph{Android Studio 2.0}.

\textbf{Plataforma alvo do \emph{app} desenvolvido:} \emph{Android}.

\textbf{Público alvo da aplicação:} PDV\@.

\subsection{\emph{Blind and visually impaired user interface to solve accessibility problems}}

Este estudo realizou uma RSL e testes em várias aplicações móveis para PDV, e dividiu os problemas encontrados em três categorias: organização, apresentação e comportamento (OAC).
Uma aplicação móvel, chamada ``\emph{Read Master}'', também foi desenvolvida no trabalho de \citeonline{Shera2021285}, incorporando soluções para os principais problemas de OAC.
Por fim, o artigo apresentou diretrizes de \emph{design} e desenvolvimento, baseadas na avaliação prática, para superar problemas na criação de aplicações móveis acessíveis à PDV\@.

O \emph{app} consiste em duas funcionalidades principais: fornecer informações cientificas e \emph{quizzes} de múltipla escolha.
As principais técnicas e funcionalidades identificadas no estudo para o suporte de acessibilidade foram:

\begin{enumerate}
\item \emph{SUS} para avaliação de usabilidade da aplicação;
\item Levantamento e categorização dos principais problemas de acessibilidade em \emph{apps} móveis.
\end{enumerate}

Uma avaliação de usabilidade do \emph{app}, com 56 PDV, foi conduzida e validada com foco na experiência de usuários com DV\@.
Os resultados mostraram que a organização da aplicação estava 100\% efetiva tanto para usuários os cegos quanto para os com DV parcial.
Já quanto a eficiência, a dos usuários com DV parcial se mostrou maior que a dos cegos.
O nível mais alto de satisfação, quanto as 3 categorias de problemas avaliados, para usuários com DV total, estava na apresentação com 87,62\%, enquanto para os com visão parcial estava tanto na organização quanto na apresentação com 89,21\%.
No geral, o estudo indica que a aplicação reduziu a gravidade dos problemas de OPB, oferecendo alta usabilidade.

\textbf{Tecnologia utilizada para desenvolvimento:} Não informado.

\textbf{Plataforma alvo do \emph{app} desenvolvido:} \emph{Android}.

\textbf{Público alvo da aplicação:} PDV\@.

\subsection{\emph{Design and development of a mobile app of drug information for people with visual impairment}}

O trabalho de \citeonline{Amariles2020}, foi desenvolvido na Colombia, onde a falta de acesso à informações, acessíveis, de rótulos de medicamentos como contraindicações, armazenamento, data de validade e dosagem foi indentificada como uma das principais barreiras no uso de medicamentos por PDV\@.

Nesse contexto, uma aplicação \emph{mobile}, chamada \emph{FarmaceuticApp}, foi desenvolvida no estudo.
A principal funcionalidade do \emph{app} é a de buscar por informações de medicamentos, onde essas informações são apresentadas ao usuário de forma acessível e a busca pode ser realizada por vários meios, esses que serão listados adiante.

As principais técnicas e funcionalidades identificadas, relacionadas à acessibilidade e utilizadas no desenvolvimento dessa solução, foram:

\begin{enumerate}
\item Tamanho da fonte das letras personalizável;
\item Vibração e sons para alertar o usuário do resultado da busca;
\item \emph{Tutorial} com possibilidade de ser visto novamente;
\item Possibilidade de busca por \emph{barcode} e \emph{qrcode}, foto, comando de voz e texto;
\item Possibilidade de ativar e desativar o assistente de voz do \emph{app}.
\end{enumerate}

\textbf{Tecnologia utilizada no desenvolvimento:} \emph{Java, Android Studio, Accessibility Scanner App}, e o \emph{Test Lab do Firebase}.

\textbf{Plataforma alvo do \emph{app} desenvolvido:} \emph{Android}.

\textbf{Público alvo da aplicação:} PDV que buscam obter informações de rótulos de medicamentos\@.

O estudo envolveu 48 PDV, das quais 69\% necessitavam de assistência para o uso de medicamentos e 90\% possuíam celulares, sendo 93\%  deles com o SO \emph{Android}.
Na avaliação final, 100\% dos usuários disseram utilizariam o \emph{app} e o avaliaram entre 4 e 5 estrelas (bom e muito bom).


\subsection{\emph{Designing multimodal mobile interaction for a text messaging application for visually impaired users}}

Apesar da inclusão de opções de acessibilidade, os SOs móveis ainda enfrentam uma falta de suporte adequado para alguns tipos de atividades e contextos, como é o exemplo da escrita de textos para PDV, uma tarefa que acaba consumindo muito tempo.
Além disso, os usuários geralmente necessitam utilizar as duas mãos para escrever mensagens, o que mostra ser um problema para cegos que necessitam carregar bengala ou possuem cão guia, assim restando apenas uma mão livre.

Nesse contexto, a abordagem proposta no estudo de \citeonline{Duarte2017}, através do protótipo de um \emph{app} para envio de mensagens, visou uma interação com o \emph{smartphone} com as mãos livres, através de técnicas multimodais, especialmente o uso de gestos em combinação com comandos de voz.

Os gestos são utilizados como gatilhos para ações.
Assim, quando um gesto é reconhecido, ele ativa alguma função, que geralmente ativa o ``reconhecedor de fala'' ou o TTS\@.
Por exemplo, existe um gesto para a ação de adicionar uma nova mensagem, ao reconhece-lo, o \emph{app} ativa o reconhecedor de fala para que o usuário dite o que deve ser escrito na mensagem.
Um outro gesto ativa a função para revisão da mensagem escrita, ao ser reconhecido, o TTS é ativado e a mensagem é lida palavra a palavra.
As principais características relacionadas à acessibilidade identificadas nessa solução foram:

\begin{enumerate}
\item Reconhecimento de voz;
\item Reconhecimento de gestos;
\item Sintese de fala.
\item Possibilidade de revisar as mensagens escritas de maneira acessível;
\item Possibilidade de parar a narração durante a revisão da mensagem e editar palavras especificas;
\item Aplicação de questionário da Escala de Usabilidade do Sistema, SUS\@.
\end{enumerate}

\textbf{Tecnologia utilizada no desenvolvimento:} \emph{Java, Android Studio, Accessibility Scanner App}, e o \emph{Test Lab do Firebase}.

\textbf{Plataforma alvo do \emph{app} desenvolvido:} \emph{Android}.

\textbf{Público alvo da aplicação:} PDV\@.

Uma pesquisa foi realizada com 9 usuários com DV e resultou em \emph{feedbacks} positivos, principalmente a respeito da interação por gestos.
Na avaliação da usabilidade das aplicações, através da escala SUS, ambas atingiram 74 pontos, considerada uma alta pontuação.

O estudo também trouxe comparativo de performance dos usuários na realização de tarefas no \emph{app} de envio de mensagem padrão com o \emph{app} desenvolvido.
Os resultados mostraram que na realização de tarefas fáceis, a performance do \emph{app} era pouco superior a alternativa padrão do sistema.
Porém, passa-se a notar grandes diferenças a favor do \emph{app} desenvolvido em tarefas consideradas normais e difíceis, com cerca de 30\% e 50\% mais performance, respectivamente, para a solução desenvolvida em relação ao \emph{app} padrão.

\subsection{\emph{Do You like My Outfit? Cromnia, a Mobile Assistant for Blind Users}}

O objetivo do estudo de \citeonline{Giuliana2018} foi projetar uma solução assistiva que pudesse prover autonomia à pessoas cegas em suas atividades diárias.
Especialistas na área de deficiência visual, de clínicos à profissionais de reabilitação vocacional e operadores do campo de cuidados sociais, participaram do estudo.

%Uma pesquisa, em forma de questionário, com 10 pessoas foi realizada como parte de um projeto europeu que visa definir um roteiro de compras inovador para PDV\@.

O processo de análise e projeto envolveu, desde o início, a participação de 4 pessoas cegas da \emph{Italian Blind Union}, que se voluntariaram para colaborar com a equipe de \emph{design} de usabilidade.
Entre as tarefas diárias que mais se esperava autonomia a de se vestir com uma combinação de cores e roupas adequadas se mostrou ser o maior interesse para as PDV, essas que geralmente dependem de ajudantes para isso.
Assim, uma aplicação \emph{mobile} foi projetada, visando a autonomia de PDV, total ou parcial, nesse ato cotidiano de se vestir.

\begin{enumerate}
\item Integração com leitores de tela;
\item Tamanho de fontes e \emph{labels} adaptáveis de acordo com o tipo de deficiência;
\item Sistema de notificações simples e imediato;
\item Resposta em tempo real.
\end{enumerate}

\textbf{Tecnologia utilizada no desenvolvimento:} Não informado.

\textbf{Plataforma alvo do \emph{app} desenvolvido:} \emph{iOS}.

\textbf{Público alvo da aplicação:} PDV\@.

Como resultado do estudo uma aplicação chamada de \emph{Cromnia}, que possibilita que os usuários reconheçam cores, padrões e combinações de cores, considerando a iluminação do ambiente foi desenvolvida.
O \emph{app} é bem simples e consiste em uma única \emph{interface}, parecida com a padrão da câmera do sistema \emph{iOS}.
O estudo levantou que já existiam soluções no mercado para esse problema, porém a ideia de uma ferramenta paga não foi bem aceita pelos entrevistados, que observaram que muitos nem poderiam pagar.

Os testes envolveram 6 PDV com parcial e 6 com DV total.
Os participantes gostaram dos benefícios do \emph{app} e se mostraram ansiosos para experimentar novas versões, pensando em quando poderão utilizar o aplicativo de fato no dia-a-dia.
O \emph{app} está disponível na \emph{AppStore} e conta com alto número de \emph{downloads}.

\subsection{\emph{Effect of UX Design Guideline on the information accessibility for the visually impaired in the mobile health apps}}

\lipsum[31]

\subsection{\emph{Improved and Accessible E-Book Reader Application for Visually Impaired People}}

Embora livros digitais já estejam estabelecidos internacionalmente, não são satisfatórios em termos de acessibilidade e \emph{interface}.
Por conta disso, o estudo de \citeonline{Heesook2017} apresenta um aplicativo leitor de \emph{e-book} acessível à PDV, que tem o objetivo de suprimir limitações como falta de novos livros, ausência de textos alternativos e navegação desconfortável dos atuais formatos acessíveis, em áudio e \emph{Braille}.

Um levantamento de requisitos de usuário foi realizado através de questionário e cerca 70\% dos requisitos foram implementados.
O \emph{app} possibilita a realização de busca, \emph{download} e leitura de conteúdos no formato \emph{EPUB3} e possui controles para inciar, parar, avançar e retroceder a leitura.
Quanto à acessibilidade, foram identificadas as seguintes soluções:

\begin{enumerate}
\item Suporte para comandos de voz;
\item Configurações de alto contraste;
\item Sintese de voz para leitura dos \emph{e-books};
\item Tamanho dos botões e espaçamentos adequados à PDV\@.
\end{enumerate}

\textbf{Tecnologia utilizada no desenvolvimento:} Não informado.

\textbf{Plataforma alvo do \emph{app} desenvolvido:} \emph{iOS}.

\textbf{Público alvo da aplicação:} PDV que gostam de livros\@.

Nos resultados dos testes, realizados com 12 PDV (7 experientes e 5 sem experiência), o estudo mostrou que a média de satisfação dos usuários foi de aproximadamente 75\% nos testes de usabilidade, realizados em 3 fases, com usuários com e sem experiência.
Onde tempo médio de execução das tarefas foi de 92 segundos para usuários não experientes e 82 segundos para experientes.
Usuários experientes enfrentaram erros relacionados a \emph{login}, configuração e busca por tentarem utilizar suas próprias abordagens baseadas em outras aplicações.

\subsection{\emph{MathMelodies 2: A Mobile Assistive Application for People with Visual Impairments Developed with React Native}}

Esse artigo apresenta a experiência do desenvolvimento do \emph{MathMelodies 2}, uma aplicação para ajudar crianças de 1 a 5 anos com DV no estudo de matemática.
A aplicação apresenta 13 tipos de exercícios e diferentes níveis de dificuldade.
Esses exercícios se passam dentro de contos de fantasia, onde a criança tem que resolvê-los para avançar na história.

A primeira versão foi desenvolvida em 2013 através de uma campanha de \emph{crowdfunding} e lançada para \emph{iPad} de forma gratuita.
O \emph{design} do novo \emph{app} seguiu princípios que derivados da experiência e do \emph{feedback} dos usuários da versão anterior.
Uma das demandas mais comuns foi a de disponibilização do \emph{app} para outras plataformas, \emph{Android} e \emph{iOS}.
Assim, nesse trabalho, \citeonline{Ducci2018}, desenvolve essa nova versão como um protótipo, utilizando \emph{React Native} para reduzir o esforço de desenvolvimento.

\begin{enumerate}
\item Implementação nativa para \emph{iOS} e \emph{Android} de componentes não acessíveis no \emph{React Native};
\item Elementos chave de interação sempre posicionados na mesma parte da tela, em locais de fácil acesso;
\item Tamanho dos ícones e componentes adaptáveis de acordo com tamanho da tela;
\item Todos os elementos visíveis na tela sem necessidade de rolagem;
\item Cores de fundo uniformes e neutras;
\item Interações por gestos simples.
\end{enumerate}

\textbf{Tecnologia utilizada para desenvolvimento:} \emph{React Native}.

\textbf{Plataforma alvo do \emph{app} desenvolvido:} multiplataforma (\emph{Android} e \emph{iOS}).

\textbf{Público alvo da aplicação:} Crianças com DV\@.

Embora as funcionalidades básicas tenham sido contempladas pelo \emph{framework} utilizado, uma funcionalidade avançada que foi requerida não era suportada.
Por conta disso, foi necessário desenvolver componentes adicionais nativamente, isto é, utilizando as tecnologias especificas para cada plataforma.

Testes preliminares, realizados com duas pessoas (uma com DV parcial e outra total), sugeriram que a aplicação estava totalmente acessível.
Assim, o estudo conclui que \emph{React Native} é uma escolha válida para o desenvolvimento de aplicações acessíveis.

\subsection{\emph{Object Recognition and Hearing Assistive Technology Mobile Application Using Convolutional Neural Network}}

A falta de aplicações móveis que atendam pelo menos as necessidades mais comuns de PDV motivou a realização do trabalho de \citeonline{Caballero2020}, que desenvolveu uma aplicação com objetivo de atender as necessidades desse grupo através de tecnologias de Reconhecimento de Objetos (RO) e TTS\@.

O \emph{app} utiliza algoritmos de \emph{Convolutional Neural Network} (CNN), solução de aprendizado de máquina reconhecida como um poderoso método para reconhecimento de imagens, para identificar detalhes em imagens e narra-los para o usuário através do TTS\@.
O artigo se concentra mais na apresentação da API utilizada para o RO, mostrando pouco sobre a aplicação \emph{mobile}, ainda assim, foram identificadas as seguintes características de acessibilidade no \emph{app}: 

\begin{enumerate}
\item Reconhecimento de detalhes de imagens;
\item Sintese dos resultados do RO por voz.
\end{enumerate}

\textbf{Tecnologia utilizada para desenvolvimento:} Não informado.

\textbf{Plataforma alvo do \emph{app} desenvolvido:} \emph{Android}.

\textbf{Público alvo da aplicação:} PDV\@.

O estudo realizou a revisão de diferentes estudos e tecnologias que utilizam CNN, um dos principais estudos citados foi publicado em 2015 na Conferência Brasileira de Sistemas Inteligentes (BRACIS), este que utiliza RO para um sistema de navegação inteligente que possibilita que robôs interajam e determinem o comportamento de objetos.
Através dos trabalhos relacionados citados, o artigo apresenta o RO sendo utilizado para inclusão social de PDV\@.

Os resultados mostraram que CNN tem potencial para classificar coisas vivas e objetos em ambientes interiores e exteriores com alta precisão, através de imagens públicas que serviram como base para treinamento.
Assim, possibilitando um desempenho funcional e confiável do sistema em beneficio das PDV através do \emph{app} desenvolvido.

\subsection{\emph{QUIMIVOX MOBILE 2.0: Application for Helping Visually Impaired People in Learning Periodic Table and Electron Configuration}}

Muito ainda precisa ser feito quanto a inclusão de PDV no processo de ensino e aprendizagem de química, por requerer de muitos recursos visuais.
E, embora exista uma quantidade significativa de \emph{apps} que auxiliam no ensino de química, os mesmos não são acessíveis aos DV, mesmo com o uso de leitores de tela.

É nesse sentido que o estudo de \citeonline{Oliveira2019} introduziu uma nova versão do \emph{``Quimivox Mobile 2.0''}, aplicativo que apresenta informações acessíveis à DV sobre a tabela periódica e, na nova versão, a configuração eletrônica dos elementos químicos.
A interação do \emph{app} é baseada em gestos e comandos de voz, com as informações sendo apresentadas graficamente e por síntese de voz, através do \emph{TalkBack}.

A aplicação utiliza de técnicas de gestos já utilizadas em outras ferramentas que consistem em deslizar com os dedos em quatro direções.
Esses gestos foram complementados com outros específicos para a realização de ações na aplicação, tais como a ativação do reconhecimento de voz e uma opção para retornar a tela anterior.
Segue abaixo as principais técnicas e funcionalidades para acessibilidade identificadas no estudo:

\begin{enumerate}
\item Interação por reconhecimento de voz e gestos;
\item Tamanhos de fontes de letras ampliados;
\item Alto contraste (fundos pretos e textos brancos);
\item Possibilidade de escolha de cores do \emph{app} para melhorar a legibilidade para pessoas daltônicas;
\item \emph{Feedback} sonoro mesmo com \emph{Talkback} desativado.
\end{enumerate}

\textbf{Tecnologia utilizada para desenvolvimento:} \emph{Java, Android Studio} e \emph{API Airy}.

\textbf{Plataforma alvo do \emph{app} desenvolvido:} \emph{Android} 4.0 ou superior.

\textbf{Público alvo da aplicação:} PDV interessadas no aprendizado de Química\@.

Os usuários apontaram o comando de voz como a funcionalidade que mais facilitou na utilização da \emph{app}.
Na avaliação de uma das PDV, participante dos testes, o desenvolvimento de manual poderia contribuir com melhor entendimento do funcionamento do aplicativo.
Outras sugestões foram a ampliação dos tipos de toques na tela e o aumento na velocidade da voz sintetizada.

O artigo conclui que os participantes aprovaram a nova versão, avaliando positivamente o \emph{app}, indicando que a maior dificuldade estava na pouca prática no uso de dispositivos móveis por parte de alguns DV\@.
E relata que essa dificuldade estava relacionada aos gestos, onde a maioria fez algum comentário negativo, citando 5 desses participantes.
Porém, o texto supõe que, com a prática no uso dos gestos, essa dificuldade poderia ser diminuída significativamente, citando o reconhecimento da falta de experiência na utilização de dispositivos móveis por 4 participantes como justificativa, sendo que apenas um deles, chamado P10, fazia parte dos 5 participantes citados pelos comentários negativos.

\subsection{\emph{``Talkin' about the weather'': Incorporating TalkBack functionality and sonifications for accessible app design}}

Informações a respeito do clima atual e previsões são especialmente importantes para PDV, visto que podem afetar suas as decisões do cotidiano, como escolhas de rotas, roupas e tecnologias assistivas que impactam significativamente seu trajeto.
Porém, essas pessoas enfrentam péssimas experiencias tentando buscar informações sobre o clima nos dispositivos móveis, geralmente por conta dos erros entre as informações na tela e a ordem em que os leitores de tela as apresentam, além dos apps serem cheios de imagens e ícones que costumam não apresentar descrição para o usuário a menos que possa enxerga-las.

Assim, \citeonline{Tomlinson2016377}, nesse estudo, projetou um \emph{app} de clima que visa ser acessível à usuários que dependem de leitores de tela.
O estudo realizou uma análise das necessidades dos usuários com DV, levantando quais eram as informações importantes e em qual ordem eles gostariam de consumi-las.
As principais soluções quanto à acessibilidade identificadas foram:

\begin{enumerate}
\item Alternativa aos ícones padrões utilizados para indicação através dos chamados ``Ícones auditivos'';
\item Utilização constante do \emph{TalkBack} durante o processo de desenvolvimento;
\item Interface com alto contraste (textos brancos em fundo preto), visando a experiência de usuário (UX) de PDV\@;
\item Integração com \emph{Talkback} seguindo as Diretrizes de Acessibilidade do \emph{Google}.
\end{enumerate}

``Ícones auditivos'' emitem sons breves, baseados nos sons reais do cotidiano, e servem alternativa para representação dos ícones visuais de clima, como o ícone de chuva, representado por sons que remetem ao evento.

\textbf{Tecnologia utilizada para desenvolvimento:} Não informado.

\textbf{Plataforma alvo do \emph{app} desenvolvido:} \emph{Android}.

\textbf{Público alvo da aplicação:} PDV que necessitam saber sobre o clima\@.

Nos testes de usabilidade, 7 participantes responderam que utilizaram o \emph{app} por pelo menos seis dias durante a semana e, no geral, reportaram terem obtido experiência tão boa ou melhor que nos \emph{apps} de clima que já utilizaram anteriormente.

\subsection{\emph{Users’ perception on usability aspects of a braille learning mobile application ‘mBRAILLE’}}

Estudantes com DV enfrentam dificuldades ou incapacidade, a depender do nível de DV, para obter informações visuais, o que torna o processo de aprendizagem deles mais difícil que o dos outros.
Nesse artigo, \citeonline{Nahar2019100}, apresenta o \emph{mBRAILLE}, \emph{app} que foi desenvolvido em \emph{Bangladesh} para auxiliar PDV no processo de autoaprendizagem de \emph{Braille}, sem ou com dependência mínima de outras pessoas.
Embora a publicação não apresente muitos detalhes do processo de desenvolvimento, sequer mencionam leitores de tela, algumas características relacionadas à acessibilidade utilizadas na solução foram identificadas, seguem:

\begin{enumerate}
\item \emph{Tutorial} para auxiliar o usuário na utilização do \emph{app};
\item \emph{Feedback} por vibração e áudio;
\end{enumerate}

\textbf{Tecnologia utilizada para desenvolvimento:} Não informado.

\textbf{Plataforma alvo do \emph{app} desenvolvido:} \emph{Android}.

\textbf{Público alvo da aplicação:} Estudantes de Bangladesh com DV\@.

O estudo avaliou 4 aspectos de usabilidade (aprendizagem, interface e funcionalidades, acessibilidade e auto descritividade) do app através de testes com 5 usuários com DV, que realizaram a avaliação após utilizarem a aplicação por 2 semanas, mostrando resultados de avaliação média satisfatórios, de 6 ou acima, numa escala de 0 a 7.

O estudo teve a uma limitação de apenas 5 participantes, sendo todos experientes em \emph{Braille}.
Assim, o artigo menciona que trabalhos futuros concentrar-se-ão em avaliar e testar a efetividade do aprendizado de \emph{Braille} através do \emph{app}, com um grande número de participantes de diferentes escolas.


\subsection{\emph{WordMelodies: Supporting Children with Visual Impairment in Learning Literacy}}

As ferramentas educacionais de escolas primarias frequentemente não são acessíveis para crianças com DV\@.
Além disso, os livros costumam ser ricos em conteúdos gráficos com o intuito de engajar os alunos, impactando na acessibilidade mesmo quando estão disponíveis no formato digital.
Da mesma forma, \emph{apps} educacionais frequentemente possuem conteúdos gráficos interativos de maneira inacessível à PDV\@.

Visando amenizar esses problemas, o artigo de \citeonline{Mascetti2019} apresenta o \emph{WordMelodies}, uma aplicação \emph{mobile} inclusiva e multiplataforma que tem como objetivo ajudar crianças com DV na adquisição de habilidades básicas de literatura com 8 tipos de exercícios.
A aplicação foi projetada e avaliada por 3 especialistas no dominio de tecnologias assistivas e educação para crianças com DV\@.
As principais características relativas à acessibilidade encontradas no artigo foram:

\begin{enumerate}
\item Elementos chave de interação sempre posicionados na mesma parte da tela, priorizando os cantos da tela;
\item Interações por gestos como ``arrastar e soltar'' com descrição auditiva;
\item Descrição alternativa em texto dos elementos de tela para integração com leitores de tela.
\end{enumerate}

\textbf{Tecnologia utilizada para desenvolvimento:} \emph{React Native}.

\textbf{Plataforma alvo do \emph{app} desenvolvido:} multiplataforma (\emph{Android} e \emph{iOS}).

\textbf{Público alvo da aplicação:} Crianças com DV\@.

Na avaliação dos especialistas, o \emph{app} se mostrou totalmente acessível, exceto por um problema que afetou a utilização do usuário ao navegar entre os elementos utilizando leitores de tela.
Nessa navegação, a ordem dos elementos não corresponde com a ordem lógica apresentada na tela, problema que ocorreu por uma limitação do \emph{kit} de ferramentas da plataforma de desenvolvimento utilizada, o \emph{React Native}.

Um dos principais desafios no desenvolvimento foi alcançar uma funcionalidade de ``arrastar e soltar'' acessível e fácil de utilizar.
Pois, no \emph{React Native} esse componente não fornece suporte à acessibilidade, sendo necessário o desenvolvimento de um componente nativo tanto no \emph{iOS} como no \emph{Android}, para prover informações auditivas ao usuário enquanto ele utiliza o componente.
