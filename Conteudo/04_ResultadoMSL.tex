\newpage{}

\section{Resultados Encontrados}

Nesta seção, são apresentados os resumos com as principais características, relacionadas ao tema deste trabalho, dos artigos selecionados na fase de extração, visando encontrar respostas para as questões levantadas na definição do protocolo de MSL\@.

\subsection{\emph{A Mobile Educational Game Accessible to All, Including Screen Reading Users on a Touch-Screen Device}}

O estudo realizado por \citeonline{Leporini2017} teve o objetivo levantar informações e possíveis soluções para as dificuldades levantadas por um grupo composto por 6 pessoas cegas ao responder questões de tarefas interativas.
E investigou, através de tarefas interativas como exercícios e questionários, a acessibilidade e usabilidade de gestos e leitores de tela em dispositivos móveis com \emph{touch-screen}.

No artigo é apresentado um \emph{game mobile} que envolveu duas pessoas cegas com experiência na utilização de \emph{smartphones} na fase inicial do planejamento do protótipo.
O \emph{game} funciona como se fosse um ``sistema solar'' com oito planetas, onde cada planeta representa um conjunto de questões e exercícios.
O jogador recebe determinada pontuação cada vez que joga de acordo com os acertos e erros durante o game.
As principais funcionalidades do \emph{app} relativas à acessibilidade identificadas foram:

\begin{enumerate}
\item Contraste de cor para garantir diferentes níveis de acessibilidade;
\item Apresentações de conteúdos de forma auditiva e visual;
\item Interação via gestos ou toques;
\item Suporte auditivo com descrições dos elementos.
\end{enumerate}

Através da avaliação desse protótipo, por cegos, o estudo investigou o suporte de acessibilidade \emph{mobile} multiplataforma do conjunto de especificações técnicas, \emph{WAI-Aria}\footnote{\url{https://www.w3.org/WAI/standards-guidelines/aria/}}, observando problemas na detecção de elementos, devido às suas posições na tela e conteúdos difíceis de identificar na interação com leitores de tela.
Notando também que houve alguma dificuldade por conta de gestos implementados no \emph{app} diferirem dos habituais utilizados pelos usuários no \emph{VoiceOver} do \emph{iOS}.

Apesar dos problemas encontrados, o artigo aponta que o \emph{feedback} foi positivo e os resultados mostraram que os exercícios puderam ser realizados facilmente, por pessoas cegas, através de simples gestos com auxilio dos leitores de tela.

\textbf{Tecnologia utilizada para desenvolvimento:} \emph{Cordova Framework}.

\textbf{Plataforma alvo do \emph{app} desenvolvido:} multiplataforma (\emph{Android} e \emph{iOS}).

\textbf{Público alvo da aplicação:} PDV\@.

\subsection{\emph{A Model-Driven Approach to Cross-Platform Development of Accessible Business Apps}}

Um procedimento comum no processo de desenvolvimento de \emph{software} é considerar a acessibilidade para PDV apenas na etapa final.
Além disso, muitos desenvolvedores não estão cientes de técnicas de software para atender esse grupo, pois o domínio de apps móveis multiplataforma tem recebido uma atenção limitada por pesquisadores.
Foi nesse sentido, que o estudo de \citeonline{Christoph2020} buscou identificar desafios, requisitos e soluções técnicas de acessibilidade, selecionando 28 requisitos a respeito de acessibilidade para aplicações móveis através de uma RSL\@.

O artigo apresenta uma abordagem orientada a modelos que integra conceitos de acessibilidade no desenvolvimento de aplicações móveis multiplataforma em conjunto com protótipos acessíveis à PDV, construídos com base nessa abordagem.
Uma aplicação com foco nos cidadãos que desejam obter informações sobre chuvas fortes e inundações foi desenvolvida, nela os usuários podem ter uma visão de eventos de inundações próximos e compartilhar novos incidentes.

O estudo comparou uma versão da aplicação desenvolvida nativamente que necessitou de 3,400 linhas de código \emph{Java} e 3,200 linhas de código \emph{XML} (gerado de forma semiautomática) com outra versão, com um conjunto similar de funcionalidades.
A nova versão do \emph{app} consistiu em 445 linhas de código \emph{MD²}, \emph{framework} baseado na abordagem orientada a modelos para desenvolvimento móvel multiplataforma através da linguagem de alto nível \emph{Xtend}\footnote{\url{https://www.eclipse.org/xtend/}}.
Principais funcionalidades sobre acessibilidade identificadas:

\begin{enumerate}
\item Adaptação da \emph{interface} de acordo com as necessidades do usuário;
\item Integração com os leitores de tela através do fornecimento de descrições em texto para elementos não textuais;
\item Personalização do contorno de foco do \emph{TalkBack}.
\end{enumerate}

Segundo o artigo, o estudo de caso mostrou que \emph{apps} acessíveis podem ser gerados a partir do modelo de alto nível \emph{MD²}, implementando as técnicas de integração adequadas em cada ponto.
Embora o autor afirme isso, o estudo também deixa claro que ainda havia uma pendência de validação centrada no usuário, visto que o trabalho não implementou todas as técnicas e a solução proposta não foi testada com PDV\@.

\textbf{Tecnologia utilizada para desenvolvimento:} \emph{Xtend, Java} e \emph{Eclipse}.

\textbf{Plataforma alvo do \emph{app} desenvolvido:} multiplataforma (\emph{Android} e \emph{iOS}).

\textbf{Público alvo da aplicação:} PDV interessadas em saber sobre eventos climáticos locais como chuvas fortes e inundações.

\subsection{\emph{An Accessible Roller Coaster Simulator for Touchscreen Devices: An Educational Game for the Visually Impaired}}

O trabalho de \citeonline{Biase2018} apresenta um \emph{app} simulador de montanha russa, baseado em simuladores educacionais já existentes e adaptado para \emph{smartphones}, para ser utilizado em disciplinas Educação Física por pessoas com e sem DV\@.
A aplicação foi desenvolvida para auxiliar no estudo de Energia Mecânica e trás as interações por áudio e tátil como alternativas à visual.
As principais funcionalidades sobre acessibilidade identificadas no \emph{app} foram:

\begin{enumerate}
\item Os elementos visuais possuem descrições textuais para integração com leitores de tela;
\item \emph{Feedback} através de ``texto para voz'' (TTS, do inglês \emph{text-to-speech}) e vibração ao clicar em determinados elementos na tela, mesmo com o modo de acessibilidade desativado;
\item Efeitos sonoros característicos que ilustram os resultados da simulação ao longo do percurso.
\end{enumerate}

Com taxas de 73\% eficácia, 77\% de eficiência e 66\% satisfação do usuário com relação a aplicação desenvolvida, os testes de usabilidade demonstraram que as estratégias de interação propostas são viáveis, com grande potencial para serem utilizadas em propósitos educacionais.

Alguns problemas de acessibilidade afetaram a taxa de satisfação dos usuários, a mantendo em 66\%, tais como dificuldades em seguir a trilha da montanha com apenas um dedo, não ser possível detectar quando o carro está voltando no trilho e falha no comando que altera o foco dos elementos, alterando para o elemento errado.

\textbf{Tecnologia utilizada para desenvolvimento:} \emph{Unity 3D engine}.

\textbf{Plataforma alvo do \emph{app} desenvolvido:} \emph{Android}.

\textbf{Público alvo da aplicação:} Pessoas com e sem DV\@.

\subsection{\emph{Application for the Configuration and Adaptation of the Android Operating System for the Visually Impaired}}

Apesar das vantagens dos dispositivos móveis, alguns desafios da interação de PDV com os sistemas operacionais (SOs) desses dispositivos precisam ser superados, para que a tecnologia alcance um número significativo nesse grupo.
Assim, o estudo de \citeonline{Oliveira2018} visou planejar e desenvolver uma aplicação que automatize as configurações do SO \emph{Android} de acordo com as preferências de acessibilidade de cada PDV, através de comandos de voz.
O artigo apresenta algumas funcionalidades e técnicas relacionadas a acessibilidade que são listadas a seguir:

\begin{enumerate}
\item Escala de Usabilidade do Sistema (SUS, do inglês \emph{System Usability Scale}) para avaliação de usabilidade da aplicação;
\item \emph{SpeechRecognizer} do \emph{Android} para reconhecimento de voz;
\item Eurísticas de Usabilidade de Nielsen (do inglês, \emph{Nielsen Usability Heuristics}) para evitar problemas de acessibilidade já mapeados.
\end{enumerate}

Um protótipo foi desenvolvido e mostrou potencial para ser utilizado como ferramenta para PDV, trazendo benefícios com a possibilidade do uso de comando de voz.
Os testes foram realizados com seis voluntárias com DV, sendo duas parcial e quatro total.
Onde três delas já possuíam experiência com comandos de voz e apenas duas das seis pessoas já haviam realizado a configuração do dispositivo alguma vez.

As voluntárias expressaram avaliações positivas quanto a autonomia, satisfação e usabilidade da aplicação.
E o tempo gasto para realizar as configurações de acessibilidade foi mais curto no \emph{app} desenvolvido que na aplicação padrão do \emph{Android}.

\textbf{Tecnologia utilizada para desenvolvimento:} \emph{Android Studio 2.0}.

\textbf{Plataforma alvo do \emph{app} desenvolvido:} \emph{Android}.

\textbf{Público alvo da aplicação:} PDV\@.

\subsection{\emph{Blind and visually impaired user interface to solve accessibility problems}}

Este estudo realizou uma RSL e testes em várias aplicações móveis para PDV, e dividiu os problemas encontrados em três categorias: organização, apresentação e comportamento (OAC).
Uma aplicação móvel, chamada ``\emph{Read Master}'', também foi desenvolvida no trabalho de \citeonline{Shera2021285}, incorporando soluções para os principais problemas de OAC.
Por fim, o artigo apresentou diretrizes de \emph{design} e desenvolvimento, baseadas na avaliação prática, para superar problemas na criação de aplicações móveis acessíveis à PDV.

O \emph{app} consiste em duas funcionalidades principais: 1) fornecer informações cientificas; e 2) \emph{quizzes} de múltipla escolha.
As principais técnicas e funcionalidades identificadas no estudo para o suporte de acessibilidade foram:

\begin{enumerate}
\item \emph{SUS} para avaliação de usabilidade da aplicação;
\item Levantamento e categorização dos principais problemas de acessibilidade em \emph{apps} móveis.
\end{enumerate}

Uma avaliação de usabilidade do \emph{app}, com 56 PDV, foi conduzida e validada com foco na experiência de usuários com DV.
Os resultados mostraram que a organização da aplicação estava 100\% efetiva tanto para usuários os cegos quanto para os com DV parcial.
Já quanto a eficiência, a dos usuários com DV parcial se mostrou maior que a dos cegos.
O nível mais alto de satisfação, quanto as 3 categorias de problemas avaliados, para usuários com DV total, estava na apresentação com 87,62\%, enquanto para os com visão parcial estava tanto na organização quanto na apresentação com 89,21\%.
No geral, o estudo indica que a aplicação reduziu a gravidade dos problemas de OPB, oferecendo alta usabilidade.

\textbf{Tecnologia utilizada para desenvolvimento:} Não informado.

\textbf{Plataforma alvo do \emph{app} desenvolvido:} \emph{Android}.

\textbf{Público alvo da aplicação:} PDV\@.

\subsection{\emph{Design and development of a mobile app of drug information for people with visual impairment}}

O trabalho de \citeonline{Amariles2020}, foi desenvolvido na Colombia, onde a falta de acesso à informações, acessíveis, de rótulos de medicamentos como contraindicações, armazenamento, data de validade e dosagem foi indentificada como uma das principais barreiras no uso de medicamentos por PDV.

Nesse contexto, uma aplicação \emph{mobile}, chamada \emph{FarmaceuticApp}, foi desenvolvida no estudo.
A principal funcionalidade do \emph{app} é a de buscar por informações de medicamentos, onde essas informações são apresentadas ao usuário de forma acessível e a busca pode ser realizada por vários meios, esses que serão listados adiante.

As principais técnicas e funcionalidades identificadas, relacionadas à acessibilidade e utilizadas no desenvolvimento dessa solução, foram:

\begin{enumerate}
\item Tamanho da fonte das letras personalizável;
\item Vibração e sons para alertar o usuário do resultado da busca;
\item \emph{Tutorial} com possibilidade de ser visto novamente;
\item Possibilidade de busca por \emph{barcode} e \emph{qrcode}, foto, comando de voz e texto;
\item Possibilidade de ativar e desativar o assistente de voz do \emph{app}.
\end{enumerate}

\textbf{Tecnologia utilizada no desenvolvimento:} \emph{Java, Android Studio, Accessibility Scanner App}, e o \emph{Test Lab do Firebase}.

\textbf{Plataforma alvo do \emph{app} desenvolvido:} \emph{Android}.

\textbf{Público alvo da aplicação:} PDV que buscam obter informações de rótulos de medicamentos\@.

O estudo envolveu 48 PDV, das quais 69\% necessitavam de assistência para o uso de medicamentos e 90\% possuíam celulares, sendo 93\%  deles com o SO \emph{Android}.
Na avaliação final, 100\% dos usuários disseram utilizariam o \emph{app} e o avaliaram entre 4 e 5 estrelas (bom e muito bom).


\subsection{\emph{Designing multimodal mobile interaction for a text messaging application for visually impaired users}}

Apesar da inclusão de opções de acessibilidade, os SOs móveis ainda enfrentam uma falta de suporte adequado para alguns tipos de atividades e contextos, como é o exemplo da escrita de textos para PDV, uma tarefa que acaba consumindo muito tempo.
Além disso, os usuários geralmente necessitam utilizar as duas mãos para escrever mensagens, o que mostra ser um problema para cegos, visto que necessitam carregar bengala ou possuem cão guia, assim restando apenas uma mão livre.
Nesse contexto, a abordagem proposta no estudo, através de uma aplicação protótipo para envio de mensagens, visa uma interação com o smartphone com as mãos livres, através de técnicas multimodais, especialmente o uso de gestos em combinação com comandos de voz.

Os gestos são utilizados como gatilhos para ações.
Assim, quando um gesto é reconhecido, ele ativa alguma função, que geralmente ativa o "reconhecedor de fala" ou o TTS.
Por exemplo, existe um gesto para a ação de adicionar uma nova mensagem, ao reconhece-lo, o app ativa o reconhecedor de fala para que o usuário dite o que deve ser escrito na mensagem.
Um outro gesto ativa a função para revisão da mensagem escrita, ao ser reconhecido, o TTS é ativado e a mensagem é lida palavra a palavra.

\begin{enumerate}
\item Interação por gestos;
\item Text-to-speech;
\item Comando de voz.
\end{enumerate}

\textbf{Tecnologia utilizada no desenvolvimento:} \emph{Java, Android Studio, Accessibility Scanner App}, e o \emph{Test Lab do Firebase}.

\textbf{Plataforma alvo do \emph{app} desenvolvido:} \emph{Android}.

\textbf{Público alvo da aplicação:} PDV\@.

O artigo traz um caso de estudo inicial focado em um app para envio de mensagens, onde uma pesquisa foi realizada com 9 usuários com DV e resultou em feedbacks positivos, principalmente a respeito da interação por gestos.
O estudo também trouxe comparativo de performance dos usuários na realização de tarefas em apps de envio de mensagem padrão com o app desenvolvido.
Os resultados mostraram que na realização de tarefas fáceis, a performance do app era pouco inferior as alternativas padrões do sistema, porém, passa-se a notar grandes diferenças a favor do app em tarefas consideradas normal e difíceis, com cerca de 30\% e 50\% mais performance, respectivamente, para a solução desenvolvida em relação ao app padrão.

\subsection{\emph{Do You like My Outfit? Cromnia, a Mobile Assistant for Blind Users}}

Apresentam uma aplicação mobile assistiva projetada para permitir a autonomia de PDV/cegas nas atividades diárias de se vestir.
Uma pesquisa, em forma de questionário, com 10 pessoas foi realizada como parte de um projeto europeu que visa definir um roteiro de compras inovador para PDV.
Especialistas na área de deficiência visual, de clínicos à profissionais de reabilitação vocacional e operadores do campo de cuidados sociais, participaram do estudo.
O objetivo do estudo foi projetar uma solução assistiva que pudesse prover grande autonomia à pessoas cegas em suas atividades diárias.
O processo de análise e projeto envolveu, desde o inciio, a participação de 4 pessoas cegas da Italian Blind Union, que se voluntariaram para colaborar com a equipe de design de usabilidade.
Entre as tarefas diárias que mais se esperava autonomia a de se vestir com uma combinação de cores e roupas adequadas se mostrou ser o maior interesse para as PDV, essas que geralmente dependem de ajudantes para isso.

Requisitos funcionais:
Detecção de iluminação ambiente;
Detecção de cor e textura;
Combinação de cores apropriadas.
Não funcionais:
Total integração com a ferramenta de sintetização de voz dos dispositivos;
Tamanho de fontes e labels adaptáveis de acordo com o tipo de deficiência.
Resposta em tempo real;
Sistema de notificações simples e imediato.

\textbf{Tecnologia utilizada no desenvolvimento:} Não informado.

\textbf{Plataforma alvo do \emph{app} desenvolvido:} \emph{iOS}.

\textbf{Público alvo da aplicação:} PDV\@.

O estudo levantou que já existiam soluções no mercado para esse problema, porém a ideia de uma ferramenta paga não foi bem aceita pelos entrevistados, que observaram que muitos nem poderiam pagar.
Os testes envolveram 6 pessoas cegas e 6 com DV.
O app é bem simples e consiste em uma única interface, parecida com a padrão da câmera do sistema iOS.
Como resultado do estudo a aplicação chamada de Cromnia que possibilita que os usuários reconheçam cores, padrões e combinações de cores, considerando a iluminação do ambiente foi desenvolvida.
Os participantes gostaram dos benefícios do app e se mostraram ansiosos para experimentar novas versões do app, visando em quando poderão utilizar o app de fato no dia a dia.
O app está disponível na AppStore e conta com alto número de downloads.

\subsection{\emph{Effect of UX Design Guideline on the information accessibility for the visually impaired in the mobile health apps}}

\lipsum[31]

\subsection{\emph{Improved and Accessible E-Book Reader Application for Visually Impaired People}}

Apresenta um estudo com um aplicativo leitor de e-book acessível para PDV para suprir as limitações dos atuais livros em áudio e Braille, tais como falta de novos livros, ausência de textos alternativos e navegação desconfortável.
Embora livros digitais já estejam estabelecidos internacionalmente, não são satisfatórios em termos de acessibilidade e interface.
O principal dispositivo utilizado para leitura foi mobile, sendo iOS a maioria com 22 usuários contra 5 para o Android.

- Searching, downloading and reading of EPUB3 contents
- Play, Stop, Move controls in the reading mode
- Function of bookmark, memo, search and sleep timer
- Accessibility features such as high contrast settings, highlighting, and TTS voice attribute settings

\textbf{Tecnologia utilizada no desenvolvimento:} Não informado.

\textbf{Plataforma alvo do \emph{app} desenvolvido:} \emph{iOS}.

\textbf{Público alvo da aplicação:} PDV que gostam de livros\@.

Média de satisfação dos usuários maior que 75\% nos testes de usabilidade, realizados em 3 fases com usuários com e sem experiência;
Tempo médio de execução das tarefas foi de 92 segundos para usuários não experientes e 82 segundos para experientes;
Usuários experientes enfrentaram erros relacionados a login, configuração e busca por tentarem utilizar suas próprias abordagens baseadas em outras aplicações;

\subsection{\emph{MathMelodies 2: A Mobile Assistive Application for People with Visual Impairments Developed with React Native}}

Apresentar a experiência do desenvolvimento do MathMelodies 2, uma aplicação que ajuda crianças de 1 a 5 anos com DV no estudo de matemática.
Uma versão anterior havia sido desenvolvida com código nativo para iPad apenas, já a segunda versão foi desenvolvida com React Native para Android e iOS.
A aplicação apresenta 13 tipos diferentes de exercícios em diferente níveis de dificuldade.
Esses exercícios dentro de contos de fantasia, onde a criança tem que resolver os exercícios para avançar na história.
A primeira versão foi desenvolvida em 2013 através de uma campanha de crowdfunding e lançada para iPad de forma gratuita, desde então, o app foi baixado mais de 50,000 vezes.
Uma das demandas mais comuns dos stakeholders era a disponibilização do app para outras plataformas, Android e iOS (smartphones e tablets).
Assim, essa aplicação foi desenvolvida como um protótipo, utilizando React Native visando reduzir o esforço para desenvolvimento.

Na segunda versão foi adotado principio de design maxsize, ao invés do fixed-size utilizado na primeira versão.
Isso porque a aplicação anterior era conveniente por rodar apenas em tablets, porém ao desenvolver para diversos tamanhos de tela, foi necessário adaptar o tamanho dos ícones e componentes de acordo a o tamanho da tela.

O design do novo app seguiu princípios que foram derivados da experiência e do feedback dos usuários da versão anterior do app.
Gestos simples: o app deve ser baseado apenas em gestos simples que podem ser lidados com facilidade mesmo com o leitor de telas ativo.
Sem rolagem: todos os elementos devem ser visíveis na tela sem requerer que o usuário role a página, visto que é um incomodo para usuários que utilizam leitores de tela ou lupas.
Pontos de referência: os elementos de interação mais importantes devem sempre ser posicionados na mesma parte da tela, num local de fácil acesso.
Homogeneidade: o app deve apresentar a mesma interface para todos os usuários independente de sua deficiência, e as ferramentas de acessibilidade do sistema devem adaptar a interface para as necessidades do usuário.

\textbf{Tecnologia utilizada para desenvolvimento:} \emph{React Native}.

\textbf{Plataforma alvo do \emph{app} desenvolvido:} multiplataforma (\emph{Android} e \emph{iOS}).

\textbf{Público alvo da aplicação:} Crianças com DV\@.

Testes preliminares com PDV sugerem que a aplicação é totalmente acessível.
O estudo conclui que React Native é uma escolha válida para o desenvolvimento de aplicações acessíveis.
Legibilidade: para manter a legibilidade, apenas pequenos trechos de texto devem ser apresentados por página, assim o texto se encaixará na tela mesmo que o usuário habilite a opção de "textos grandres".
Da mesma forma, as cores de fundo dos textos devem ser uniformes e neutras, visto que deve se manter legivel mesmo com a opção de "inverter cores" ativa.
Embora funcionalidades básicas sejam contempladas pelo framework, uma funcionalidade avançada que é requerida no MathMelodies 2 não é suportada.
Por conta disso, foi necessário desenvolver componentes adicionais nativamente, isto é, utilizando as tecnologias especificas para cada plataforma.
Apesar disso, a maior parte da aplicação é multi plataforma.

Desenvolver o restante das funcionalidades do Math-Melodies 2 para que possa ser publicado.

\subsection{\emph{Object Recognition and Hearing Assistive Technology Mobile Application Using Convolutional Neural Network}}

A falta de aplicações móveis que atendam pelo menos as necessidades mais comuns de PDV motivou a realização do trabalho.
O objetivo do estudo foi desenvolver uma aplicação que atendesse as necessidades de desse grupo através de tecnologias de Reconhecimento de Objetos (RO) e TTS.
O app utiliza algoritmos de Convolutional Neural Network (CNN), solução de aprendizado de máquina reconhecida como um poderoso método para reconhecimento de imagens, que analisa imagens para identificar detalhes que são narrados para o usuário através do TTS.
O estudo realiza a revisão de diferentes estudos e tecnologias que utilizam CNN, um dos principais estudos citados foi publicado em 2015 na Conferência Brasileira de Sistemas Inteligentes (BRACIS), este que utiliza RO para um sistema de navegação inteligente que possibilita que robôs móveis interajam e determinem o comportamento de objetos.
Diferente dos trabalhos relacionados citados, o estudo apresenta o RO sendo utilizado para inclusão social de PDV.

Reconhecimento de imagens;
Leitura dos detalhes das imagens reconhecidas;
TTS;

\textbf{Tecnologia utilizada para desenvolvimento:} Não informado.

\textbf{Plataforma alvo do \emph{app} desenvolvido:} \emph{Android}.

\textbf{Público alvo da aplicação:} PDV\@.

O artigo mostra que CNN tem o potencial de classificar coisas vivas e objetos em ambientes interiores e exteriores com alta precisão através de imagens públicas que serviram como dados de treinamento para o sistema.
Assim, possibilitando desempenho funcional e confiável do sistema em beneficio da comunidade com DV através do app desenvolvido.
O estudo conclui que tecnologias para acessibilidade de deficientes visuais ainda são recentes, necessitando de um maior desenvolvimento.

\subsection{\emph{QUIMIVOX MOBILE 2.0: Application for Helping Visually Impaired People in Learning Periodic Table and Electron Configuration}}

Muito ainda precisa ser feito quanto a inclusão de PDV no processo de ensino e aprendizagem de química, por requerer de muitos recursos visuais.
E, embora exista uma quantidade significativa de apps que auxiliam no ensino de química, os mesmos não são acessíveis aos DV, mesmo com o uso de leitores de tela.
É nesse sentido que o estudo visa introduzir uma nova versão do app Quimivox Mobile 2.0, este que apresenta informações acessíveis à DV sobre a tabela periódica e a configuração eletrônica dos elementos químicos.
A interação do app é baseada em gestos e comandos de voz e as informações são apresentadas graficamente e por síntese de voz.

Síntese e o reconhecimento de voz;
Reconhecimento de gestos;
Busca de informação es na tabela periódica;

Acessibilidade aos cegos: técnicas utilizadas em outras ferramentas que consistem em deslizar com os dedos em quatros direções.
Esses gestos foram complementados com outros específicos para a realização de ações na aplicação, tais como a ativação do reconhecimento de voz.
Acessibilidade à baixa visual: visando a capacidade de leitura dos elementos na tela sem a necessidade do recurso de síntese de voz, foram inseridas opções de contrastes de cores e ampliação de fonte.
Acessibilidade à população daltônica: melhoria nos contrastes e escolha de cores da interface para reforçar a legibilidade da tela.

\textbf{Tecnologia utilizada para desenvolvimento:} \emph{Java, Android Studio} e \emph{API Airy}.

\textbf{Plataforma alvo do \emph{app} desenvolvido:} \emph{Android} 4.0 ou superior.

\textbf{Público alvo da aplicação:} PDV interessadas no aprendizado de Química\@.

O estudo conclui que os participantes aprovaram a nova versão, avaliando positivamente o app, indicando que a maior dificuldade estava na pouca prática no uso de dispositivos móveis por parte de alguns DV.
O artigo relata que essa dificuldade foi relacionada aos gestos, onde a maioria fez algum comentário negativo, citando 5 desses participantes.
Porém, o texto supõe que com a prática no uso dos gestos, essa dificuldade poderia ser diminuída significativamente, citando o reconhecimento da falta de experiência na utilização de dispositivos móveis por 4 participantes como justificativa, sendo que apenas um deles, chamado P10, fazia parte dos 5 participantes citados pelos comentários negativos.
Os usuários apontaram o comando de voz como a funcionalidade que mais facilitou a utilização da app.
Na avaliação de uma das PDV envolvidas nos testes, o desenvolvimento de manual para contribuir com melhor entendimento do funcionamento do aplicativo;
Ampliação dos tipos de toques na tela;
Aumento da velocidade da voz sintetizada;

\subsection{\emph{``Talkin' about the weather'': Incorporating TalkBack functionality and sonifications for accessible app design}}

Informações a respeito do clima atual e previsões são especialmente importantes para PDV, visto que pode afetar as decisões sobre escolhas de rotas, roupas e tecnologias assistivas que impactam significativamente seu trajeto diário.
Porém, PDV enfrentam péssimas experiencias tentando buscar informações sobre o clima nos dispositivos móveis, geralmente por conta dos erros entre as informações na tela e a ordem em que os leitores de tela as apresentam, além dos apps serem cheios de imagens e ícones que costumam não apresentar descrição para o usuário a menos que ele possa enxerga-las.
Assim, o proposito desse estudo foi projetar um app de clima que seja acessível a usuários que dependem dos leitores de tela.
O estudo realizou uma análise das necessidades dos usuários com DV quanto a quais eram as informações eram mais importantes e em qual ordem eles gostariam de consumir.

"Ícones auditivos", os ícones possuem sons breves, baseados nos sons reais do dia a dia, como a representação dos ícones visuais de clima.

Durante o processo de desenvolvimento, os desenvolvedores utilizaram constantemente o TalkBack para garantir que todos os elementos na tela poderiam ser lidos.
Alto contraste (textos brancos em fundo preto) apresenta uma melhor experiência para usuários com DV.

\textbf{Tecnologia utilizada para desenvolvimento:} Não informado.

\textbf{Plataforma alvo do \emph{app} desenvolvido:} \emph{Android}.

\textbf{Público alvo da aplicação:} PDV que necessitam saber sobre o clima\@.

Nos testes de usabilidade, 7 usuários responderam que utilizaram o app por pelo menos seis dias durante a semana e, no geral, reportaram terem obtido experiência tão boa ou melhor que nos apps que já utilizaram anteriormente.
O app foi projetado para apresentar as informações em um layout construído para funcionar efetivamente com o TalkBack, seguindo as diretrizes de acessibilidade do Google, provendo também uma alternativa auditiva para os ícones e imagens visuais utilizados para fornecer informações rápidas para o usuário.

\subsection{\emph{Users’ perception on usability aspects of a braille learning mobile application ‘mBRAILLE’}}

Estudantes com DV estão incapazes de obter informações visuais, o que torna o processo de aprendizagem deles mais difícil que o dos outros.
O artigo apresenta o mBRAILLE, app que foi desenvolvido em Bangladesh para auxiliar PDV no processo de autoaprendizagem de Braille, sem ou com dependência mínima de outras pessoas.

Inglês e Bangla (língua falada em Bangladesh)
Aprender as letras;
Praticar as letras;
E tutorial.
ISO 9241-11;

\textbf{Tecnologia utilizada para desenvolvimento:} Não informado.

\textbf{Plataforma alvo do \emph{app} desenvolvido:} \emph{Android}.

\textbf{Público alvo da aplicação:} Estudantes de Bangladesh com DV\@.

O estudo avaliou 4 aspectos de usabilidade (aprendizagem, interface e funcionalidades, acessibilidade e auto descritividade) do app através de testes com 5 usuários com DV, que realizaram a avaliação após utilizarem a aplicação por 2 semanas, mostrando resultados de avaliação média satisfatórios, de 6/10 ou acima.
O estudo teve a uma limitação de apenas 5 participantes, sendo todos experientes em Braille.
Assim, trabalhos futuros visam concentrar-se em avaliar e testar a efetividade do aprendizado de Braille através do app, com um grande número de participantes de diferentes escolas.
Tradução livre: A nova versão da ISO 9241-11, publicada em 2018, afirma que a usabilidade se aplica a todos os aspectos de uso de quaisquer produtos tecnológicos (sistema ou software), incluindo: Aprendizagem, Uso regular, Acessibilidade e Manutenibilidade.
No entanto, também fornece flexibilidade para escolha aspectos de usabilidade adequados ao software / aplicativo desenvolvido.
Por exemplo, avaliação de tela / recurso / interface, autodescrição, etc.
são importantes para qualquer aplicativo móvel baseado em toque para pessoas cegas.
Portanto, esses aspectos de usabilidade também são considerados neste estudo.

\subsection{\emph{WordMelodies: Supporting Children with Visual Impairment in Learning Literacy}}

As ferramentas educacionais de escolas primarias frequentemente não são acessíveis para crianças com DV.
Além disso, os livros costumam ser ricos em conteúdos gráficos com o intuito de engajar os alunos, impactando na acessibilidade mesmo quando estão disponíveis no formato digital.
Da mesma forma, apps educacionais frequentemente possuem conteúdos gráfico interativos de maneira inacessível a PDV.
Visando resolver esses problemas, o estudo apresenta o WordMelodies, uma aplicação mobile inclusiva e multi-plataforma que ajuda crianças com DV a adquirirem habilidades básicas de literatura com 8 tipos de exercícios.
A aplicação foi projetada e avaliada por 3 especialistas no dominio de tecnologias assistivas e educação para crianças com DV.
Após a avaliação dos especialistas, o app se mostrou totalmente acessível, exceto por uma limitação do kit de ferramentas da plataforma de desenvolvimento utilizada, o React Native.

"Arrastar e soltar" acessível à leitores de tela;
8 exercícios;

Inclusivo: o app deveria ser usável e fácil de aprender para pessoas com e sem DV.
Entretenimento: além possibilitar os usuários a prática literária, o app deve divertir e entreter.
Independência: o app deve ser utilizável por todos os usuários sem a assistencia de terceiros.
Consistencia: elementos chave de interação devem ser estar posicionados na mesma parte da tela, de preferência nas bordas e quinas onde são mais fáceis de serem encontrados.
Além do toque: o app deve utilizar e ensinar interações comuns por gestos para crianças, como o "arrastar e soltar".
Escalável: o app deve possibilitar a adição de novos exercícios e conteúdos com pouco esforço de desenvolvimento.

\textbf{Tecnologia utilizada para desenvolvimento:} \emph{React Native}.

\textbf{Plataforma alvo do \emph{app} desenvolvido:} multiplataforma (\emph{Android} e \emph{iOS}).

\textbf{Público alvo da aplicação:} Crianças com DV\@.

Durante a fase de análise, foi criada uma lista com 59 exercícios para pratica de habilidades literárias no ensino primário, coletados de especialistas, padrões de ensino e análise apps e websites existentes.
A partir de exemplos interativos desses exercícios, os 3 especialistas avaliaram a utilidade dos mesmos.
Então, os 8 exercícios com melhor avaliação foram selecionados para serem desenvolvidos no app.

Um dos principais desafios no desenvolvimento foi alcançar uma funcionalidade de "arrastar e soltar" acessível e fácil de utilizar.
Pois, no React Native esse componente da suporte à acessibilidade, sendo necessário o desenvolvimento de um componente nativo tanto no iOS como no Android, para prover informações ao usuário enquanto ele utiliza o componente.
Um problema na ordem dos elementos, causado pelo fato do React Native ter uma ordem interna os elementos de UI, que aparecem quando o usuário avança ou retorna entre os elementos, utilizando leitor de tela, não foi possível ser resolvido e permanece para correção.