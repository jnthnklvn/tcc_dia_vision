\chapter{Introdução}

Atualmente o mundo enfrenta um sério problema com relação a saúde da visão. Segundo a Organização Mundial da Saúde (\citeonline{WHO2019}), pelo menos 2,2 bilhões de pessoas no mundo possuem deficiência visual em algum grau, com isso a necessidade de cuidados com os olhos tende a crescer drasticamente nas próximas décadas. No Brasil, de acordo com o censo do \citeonline{IBGE2012}, os números dessa deficiência representavam cerca de 18,6\% da população em 2010. // citar dados mais recentes (PNAD, IBGE, estimativas)

Segundo  as estimativas, mais de 5 bilhões de pessoas possuem dispositivos móveis no mundo, sendo mais da metade destes, \textit{smartphones}. No Brasil, a taxa de adultos que dizem possuir dispositivos móveis é de 83\% no total e 60\% para \textit{smartphones}. Na faixa etária entre 18 e 34 anos, houve um aumento no número de proprietários de \textit{smartphones} de 61\% em 2015 para 85\% em 2018 \cite{Taylor2019}. //CETIC.br, NIC-BR, ITU (ONU)

Visando a inclusão das pessoas com deficiência visual numa sociedade cada vez mais conectada, tecnologias conhecidas como Tecnologias Assistivas (TA) se tornam cada vez mais presentes. \citeonline{Cook2014} utilizam em seu livro, uma definição de TA mundialmente utilizada que foi definida por uma \textit{Public Law} dos Estados Unidos da América (EUA). Os autores justificam a utilização dessa definição por a mesma contemplar os pontos mais importantes a respeito de TA, como diz a seguir: 
\begin{citacao}
Qualquer item, parte de equipamento ou sistema adquirido comercialmente, modificado ou customizado que é utilizado para aumentar, manter ou melhorar as capacidades funcionais de pessoas com deficiência \cite{Cook2014}. 
\end{citacao}

Para que essas TAs funcionassem adequadamente, organizações como a \textit{World Wide Web Consortium} (W3C) definiram diretrizes que deveriam ser seguidas no desenvolvimento de aplicações \textit{web} \cite{W3C2019}. Já para aplicações \textit{mobile}, como a implementação da tecnologia varia de acordo com o Sistema Operacional (SO), essa definição se deu pelas próprias proprietárias dos SOs, tais como Google e Apple.

% ---
\section{Motivação e Justificativa}
% ---
A Organização Mundial da Saúde (OMS) aponta que mais de 1 bilhão dos casos de pessoas com deficiência visual poderia ser evitado ou resolvido \citeonline{WHO2019}. De acordo com a OMS, isso ocorre por conta das principais causas desses casos serem as listadas a seguir:\\
a) O tempo despendido em ambientes fechados e aumento das atividades \textit{"near work"} (ler, escrever, assistir TV, jogar videogames, etc);\\
b) O aumento no número de pessoas vivendo com diabetes, principalmente tipo 2;\\
c) Muitas pessoas não terem acesso a serviços oftalmológicos e verificações de rotina.

A maior parte das diretrizes sobre acessibilidade são focadas em \textit{web}. Embora alguns conceitos possam ser aplicados a aplicações móveis, ainda são poucos diante das variações de comportamento dos sistemas \textit{mobile}. Um estudo realizado por \citeonline{Ballantyne2018}, compila um conjunto de diretrizes para acessibilidade \textit{mobile} e realiza testes em 25 dos \textit{apps} mais populares da Google Play. Os resultados do estudo revelaram que apenas 8 dos 25 selecionados possuiam taxa de conformidade com as diretrizes acima de 75\%. O estudo ainda revela que 63\% das violações encontradas são relacionadas ao \textit{design} (componentes de tela). Já \citeonline{Yan2019} elaboram um estudo mais abrangente, realizado com 479 \textit{apps} de 23 categorias da Google Play. Os autores utilizam uma ferramenta automatizada, o IBM \textit{Mobile Accessibility Checker} (MAC), para encontrar possíveis problemas com acessibilidade nesses \textit{apps}, categorizando-os em V (Violação), PV (Potêncial Violação) e A (Alerta). Os resultados encontrados por Yan e Ramachandran mostraram que 94.8\%, 97.5\% e 66.4\% dos apps continham problemas realacionados a V, PV e A, respectivamente \cite{Yan2019}.

Para \citeonline{Quispe2020}, um dos principais fatores para a baixa priorização da acessibilidade no desenvolvimento de aplicações \textit{mobile} é a alta demanda sobre as equipes de desenvolvimento. Os autores ainda apontam que essa falta de tempo faz com que o processo se concentre nos requisitos funcionais em detrimento de requisitos não funcionais de usabilidade como a acessibilidade. Nesse sentido, Quispe, Scatalon e Eler, com base nas respostas de deficientes visuais a um questionário avaliando as diretrizes de acessibilidade do  Modelo de Acessibilidade em Governo Eletrônico (e-Mag), propõem uma estratégia para ajudar os desenvolvedores a contornar os problemas com acessibilidade. Essa proposta consiste na priorização das diretrizes mais importantes de acordo com sua relevância para usuários com baixa acuidade visual.

% ---
\section{Objetivos}
% ---

% ---
\subsection{Objetivo Geral}
% ---

O presente trabalho tem o objetivo de desenvolver uma aplicação de assistência voltada a pessoas com diabetes e acuidade visual prejudicada, seguindo as diretrizes de acessibilidade (DA) do Google e Apple para Android e iOS, respectivamente. Bem como, trazer soluções alternativas baseadas nessas diretrizes.

% ---
\subsection{Objetivos Específicos}
% ---

A fim de atingir o objetivo geral, os objetivos específicos listados a seguir foram definidos:
\begin{itemize}
\item Avaliar aplicações móveis que utilizam os conceitos das DA, em busca de soluções que possam ser aplicadas ao \textit{app} a ser desenvolvido;
\item Realizar revisão sistemática de aplicações médicas com foco acessibilidade, para conhecer o estado atual;
\item Pesquisar quais são os maiores problemas enfrentados pelos usuários com relação a acessibilidade em aplicações \textit{mobile};
\item Pesquisar quais são as funcionalidades que pacientes diabéticos mais usam e/ou necessitam;
\item Desenvolver a aplicação móvel;
\end{itemize}

% ---
\section{Metodologia de Pesquisa}
% ---

Metodologia quantitativa exploratória

\begin{itemize}
\item Avaliar opções para o \textit{backend} da aplicação; //Metodologia
\item Aplicar os conceitos das principais diretrizes de acessibilidade ao aplicativo; //Metodologia
\item Publicar aplicativo na Google Play. //Metodologia
\end{itemize}

% ---
\section{Organização do Documento}
% ---

Organização




Assim, este trabalho propõe o desenvolvimento de uma aplicação, seguindo a priorização das diretrizes de acessibilidade \textit{mobile} mais relevantes, que forneça meios que possam ser úteis na mitigação de algumas das principais causas da baixa acuidade visual.

// tirar daqui

A principio, esses são os meios:\\
a) Informações para incentivo e conscientização dos usuários com relação ao tempo que passam em frente a telas de \textit{smartphones}, TVs, etc;\\
b) Informações sobre autocuidados para que pessoas com diabetes ou tendência a ter, saibam como se prevenir;\\
c) Acompanhamento das medicações prescritas para o usuário, o notificando periodicamente para que lembre de toma-las da maneira indicada pelo profissional da saúde;\\
d) Monitoramento de elementos como a glicemia e a diurese, para que o usuário seja notificado caso atinjam níveis preocupantes;\\
e) Autodiagnostico através de testes visuais, indicando que o usuário consulte um oftalmologista de acordo o resultado.\\
// tirar daqui