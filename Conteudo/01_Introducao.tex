\chapter{Introdução}

% ---
\section{Contextualização e Motivação}
% ---

O Diebetes Mellitus (DM) vem se tornando um desafio global de saúde pública cada vez maior por conta do rápido aumento no número de casos nos últimos
20 anos \cite{ADA2019}.
Estimativas da Federação Internacional de Diabetes (IDF), do Atlas da Diabetes\footnote{\url{https://diabetesatlas.org/}}
de 2019, apontaram que 463 milhões de pessoas no mundo viviam com DM, o que representa cerca de 9.3\% da população
global adulta, e é esperado um aumento para 10,2\% (578 milhões) em 2030 e 10,9\% (700 milhões) em 2045 \cite{SAEEDI2019107843}.

O Brasil é o 5º país com mais diabéticos no mundo com 16,8 milhões em 2019, na faixa etária de 20 à 79 anos, e estimativas
de 21,5 e 26 milhões de casos para 2030 e 2045, respectivamente \cite{SAEEDI2019107843}. Os custos totais de hipertensão, diabetes
e obesidade no Sistema Único de Saúde (SUS) alcançaram 3,45 bilhões de reais em 2018, sendo 30\% desse custo relacionado ao DM \cite{Nilson2020}.

Já a retinopatia diabética é uma complicação vascular do diabetes, cuja prevalência está diretamente relacionada à duração
do diabetes e ao controle do nível de glicemia \cite{Solomon412}. Essa complicação é a maior causa de novos casos de cegueira
em adultos, na faixa etária de 20 à 74 anos, em países desenvolvidos \cite{ADA2019}. Além disso, outros distúrbios oculares
como o glaucoma e a catarata ocorrem mais cedo e com maior frequência em diabéticos \cite{ADA2019}.

Assim, atualmente o mundo enfrenta um sério problema com relação a saúde da visão. Segundo a Organização Mundial da Saúde (OMS),
pelo menos 2,2 bilhões de pessoas no mundo vivem com deficiência visual (DV) em algum grau, com isso,
a necessidade de cuidados com os olhos tende a crescer drasticamente nas próximas décadas \cite{WHO2019}.

Segundo a Associação Americana de Diabetes (ADA), serviços de educação e apoio para o autogerenciamento do diabetes
(DSMES, do inglês \emph{diabetes self-management education and support}) facilitam na aquisição de conhecimento e habilidades
necessárias para o autocuidado, incorporando as necessidades, objetivos e experiências de diabéticos \cite{ADA2019}. Portanto, DSMES
visam auxiliar o empoderamento dos pacientes com diabetes na tomada de decisões informadas de autogerenciamento \cite{Marrero2013}.

De acordo com estimativas de 2019, mais de 5 bilhões de pessoas no mundo possuem dispositivos móveis, sendo mais da metade destes, \textit{smartphones}
\cite{Taylor2019}. Embora, segundo \citeonline{morris2017smartphone}, cerca de 84\% da população estadunidense com DV possua ou utilize telefone celular,
a taxa média de indivíduos que possuem nos países menos desenvolvidos é de apenas 61\% \cite{ITU_2021}.

No Brasil, a taxa de adultos que relataram possuir dispositivos móveis foi de 83\% no total e 60\% para \textit{smartphones},
sendo que, na faixa etária entre 18 e 34 anos, houve um aumento no número de proprietários de \textit{smartphones} de
61\% em 2015 para 85\% em 2018 \cite{Taylor2019}.

\citeonline{Yan2019} elaboram um estudo abrangente, realizado com 479 \textit{apps} de 23 categorias da
\emph{Google Play}, utilizando uma ferramenta automatizada, o IBM \textit{Mobile Accessibility Checker}
(MAC), para encontrar possíveis problemas com acessibilidade relacionadas à DV nesses \textit{apps},
categorizando-os em V (Violação), PV (Potencial Violação) e A (Alerta). Os resultados mostraram que 94.8\%, 97.5\% e 66.4\% dos apps continham problemas
relacionados à V, PV e A, respectivamente \cite{Yan2019}.

Diante da popularização dos \emph{smartphones} e das problemáticas mencionadas, o presente trabalho visa desenvolver uma aplicação móvel
com DSMES, para ao autocuidado de diabéticos, considerando a resolução dos principais problemas de acessibilidade para pessoas com
deficiência visual (PDV).

% ---
\section{Objetivos}
% ---

Nesta seção são apresentados os objetivos, divididos em geral e específicos. Onde o objetivo geral dá uma visão mais ampla sobre o objeto
de estudo e os específicos visam aprofundar as intenções expressas no geral \cite{cervo2006metodologia}.

% ---
\subsection{Geral}
% ---

O objetivo deste trabalho é desenvolver uma aplicação móvel multiplataforma e assistiva voltada ao autocuidado de pessoas com diabetes
e acuidade visual prejudicada. Implementando soluções para os principais problemas de acessibilidade à PDV na utilização de aplicativos móveis.

% ---
\subsection{Específicos}
% ---

Para atingir objetivo deste trabalho, os seguintes objetivos específicos foram definidos:

\begin{itemize}
    \item Identificar os principais problemas enfrentados por PDV na utilização de aplicações móveis;
    \item Identificar as principais técnicas e soluções de acessibilidade à PDV para aplicações móveis;
    \item Relacionar as principais técnicas e soluções identificadas aos principais problemas identificados;
    \item Realizar o desenvolvimento do aplicativo móvel com os principais requisitos levantados;
    \item Aplicar as soluções identificadas para os principais problemas de acessibilidade ao aplicativo.
\end{itemize}

% ---
\section{Metodologia}
% ---

A metodologia adotada neste estudo foi a quantitativa exploratória, visando responder as questões de pesquisa apresentadas
no início do capítulo 3, assim, obtendo maior familiaridade com o problema e possíveis soluções.

Para isso, foi realizado um levantamento bibliográfico por meio de um Mapeamento Sistemático da Literatura (MSL), onde foi possível
realizar uma análise dos resultados dos estudos selecionados (de acordo com o protocolo) e dos relacionados (estudos de mapeamento).

% ---
\section{Organização do Documento}
% ---
Neste capítulo foram apresentados a contextualização, a motivação e os objetivos deste trabalho. Bem como, a metodologia adotada.

Para facilitar a navegação e melhor entendimento, este documento está organizado em capítulos, cujas descrições são listadas a seguir:
\begin{itemize}
    \item Capítulo 2 - Fundamentação Teórica: aborda os principais conceitos relacionados ao trabalho realizado;
    \item Capítulo 3 - Mapeamento Sistemático da Literatura: descreve os estudos selecionados e os relacionados, e apresenta uma análise dos resultados;
    \item Capítulo 4 - Plano de continuidade: descreve o que será realizado e apresenta o cronograma das próximas etapas deste trabalho.
\end{itemize}
