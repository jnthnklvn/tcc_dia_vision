\chapter{Introdução}

% ---
\section{Motivação e Justificativa}
% ---

% ---
\section{Objetivos}
% ---

% ---
\subsection{Objetivo Geral}
% ---

O presente trabalho tem o objetivo de desenvolver uma aplicação de assistência voltada a pessoas com diabetes e acuidade visual prejudicada, seguindo as
diretrizes de acessibilidade (DA) do Google e Apple para Android e iOS, respectivamente. Bem como, trazer soluções alternativas baseadas nessas diretrizes.

% ---
\subsection{Objetivos Específicos}
% ---

A fim de atingir o objetivo geral, os objetivos específicos listados a seguir foram definidos:
\begin{itemize}
    \item Avaliar aplicações móveis que utilizam os conceitos das DA, em busca de soluções que possam ser aplicadas ao \textit{app} a ser desenvolvido;
    \item Realizar revisão sistemática de aplicações médicas com foco acessibilidade, para conhecer o estado atual;
    \item Pesquisar quais são os maiores problemas enfrentados pelos usuários com relação a acessibilidade em aplicações \textit{mobile};
    \item Pesquisar quais são as funcionalidades que pacientes diabéticos mais usam e/ou necessitam;
    \item Desenvolver a aplicação móvel;
\end{itemize}

% ---
\section{Metodologia de Pesquisa}
% ---

Metodologia quantitativa exploratória


% ---
\section{Organização do Documento}
% ---

Organização

