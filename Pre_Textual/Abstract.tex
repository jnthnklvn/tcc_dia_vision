% resumo em inglês
\setlength{\absparsep}{18pt} % ajusta o espaçamento dos parágrafos do resumo
\begin{resumo}[Abstract]
  \begin{otherlanguage*}{english}
    Diabetes Mellitus (DM) is a group of chronic endocrine diseases characterized by elevated blood
    glucose that requires continuous medical care for risk reduction and glycemic control, and has become
    an increasing challenge due to the rapid increase in the number of cases in the last 20 years. As a result,
    complications from DM are already the leading cause of blindness in adults in developed countries, with at
    least 2.2 billion people in the world living with visual impairment (VI) to some level. Previous studies
    have identified the importance of self-care in the treatment of DM, the difficulty in accessing information
    about it by patients with VI and the main features used as a solution in the market. Furthermore, in this
    work it was found that even with the increase in computerization and popularization of smartphones,
    people with VI still face serious difficulties due to the lack of accessibility in mobile applications.
    Therefore, a Systematic Literature Mapping (SLM) process was carried out to establish accessibility
    techniques that could be adopted in the development of a mobile application called DiaVision. Thus,
    the application was developed for Android and iOS, meeting the requirements identified for this target
    audience and accessibility solutions found in SLM\@. An initial version of the app was
    evaluated by 77 patients with VI and the results of this study showed that 90.1\% of the participants
    would like to have the app, with 62.3\% evaluating it with full mark. However, part of the
    accessibility solutions were implemented after this study was carried out, so 35.1\% of the
    participants pointed out that the application needed improvements in terms of accessibility. With
    this, future works must carry out a new evaluation of the application with the new functionalities
    and improvements related to accessibility.

    \textbf{Keywords}: mobile app, diabetes mellitus, self-care, accessibility, visual impairment.
  \end{otherlanguage*}
\end{resumo}