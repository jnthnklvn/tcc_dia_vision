\begin{agradecimentos}

Não haveria como não falar da minha mãe, Diana, em primeiro lugar neste espaço,
esta que abriu mão de diversas coisas para lutar pelo bem de seus 3 filhos, que sempre quis o melhor
pra mim e fez o que pôde para evitar e alertar sobre meus erros. E mesmo não entendendo minhas escolhas,
não as barrou e sempre esteve lá para quando eu quebrasse a cara.

Mãe, obrigado por tudo!

Agradeço também à Sergiane e Walesson, que me confiaram a honra de ser padrinho de Walace.
Aos meus irmãos Daiane e David que, em meio à tantas brigas, sempre nos apoiamos e defendemos uns aos outros.
Ao meu primo, Lucas, que foi uma referência durante a graduação. E, ao restante dessa grande família que, embora
não os nomeie aqui, quero que saibam que foram e são muito importantes para mim.

As amizades que fiz foi o que me manteve firme para seguir até o final dessa jornada. Assim, não posso deixar de mencionar
Abayomi com quem tenho quilômetros de mensagens trocadas no Telegram e que está em constante mudança e revolta (\emph{pelu idi}).
Brunna que talvez não saiba, mas nos momentos em que eu estava cheio de ansiedade, fez-me sentir em paz,
enquanto lia livros no silêncio de sua sala.

Igor e seus argumentos socráticos que foram interpretados como sofistas e Geovanne que está trabalhando feito louco e
me deixando preocupado com um \emph{burnout}. Mayara que sempre notava quando estava desanimado e estava lá para quando
eu precisasse desabafar. Raul de quem eu tinha que esconder a preocupação para evitar que infartasse e, por fim devido
à ordem alfabética, Roberto, o tiozão do pavê.

Embora eu consciência de minhas dificuldades para me expressar, principalmente com relação a sentimentos,
não sei o por que, mas nunca consegui dizer ``Eu te amo'' para ninguém depois que cresci, nem mesmo para minha mãe.
Assim, queria aproveitar esse momento para aqui deixar registrado que embora nunca tenha-os dito:

Amo vocês!

Não posso esquecer dos professores do DComp que apresentaram os caminhos que possibilitaram que eu seguisse nessa jornada.
Também aos professores de outros departamentos como o DPS e DCS que mostraram a importância do pensamento crítico.
E, em especial, a Gilton sempre animado e com novas ideias, Leonardo por seus conselhos e desafios, e Ricardo que nos ouvia
e parecia entender nossas frustrações.

E, por fim, à minha orientadora Adicinéia que, além da orientação e paciência comigo neste trabalho,
foi a única que, transmitindo conhecimento durante as aulas, conseguiu manter minha concentração por tanto tempo,
horas, sem me dar sono.

\end{agradecimentos}
% ---