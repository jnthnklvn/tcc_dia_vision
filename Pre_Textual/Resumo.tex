% resumo em português
\setlength{\absparsep}{18pt} % ajusta o espaçamento dos parágrafos do resumo
\begin{resumo}
    O \emph{Diabetes Mellitus} (DM) é um grupo de doenças endocrinológicas crônicas caracterizado pela elevação
    da glicose no sangue que requer cuidados médicos contínuos para redução de risco e controle glicêmico, e vem se
    tornando um desafio cada vez maior devido ao rápido aumento do número de casos nos últimos 20 anos. Com isso,
    complicações do DM já são a maior causa de cegueira em adultos de países desenvolvidos, sendo que pelo menos 2,2
    bilhões de pessoas no mundo vivem com deficiência visual (DV) em algum grau.
    Estudos anteriores identificaram a importância do autocuidado no tratamento do DM, a dificuldade
    no acesso a informações a respeito por pacientes com DV e as principais funcionalidades utilizadas
    como solução no mercado. Além disso, neste trabalho foi constatado que, mesmo com o aumento da informatização e
    popularização dos \emph{smartphones}, pessoas com DV ainda enfrentam sérias dificuldades devido à falta
    de acessibilidade em aplicações móveis. Perante o exposto, foi realizado um processo de Mapeamento Sistemático da
    Literatura (MSL) para estabelecer as técnicas de acessibilidade que poderiam ser adotadas no desenvolvimento
    de um aplicativo móvel chamado DiaVision. Assim, o aplicativo foi desenvolvido para Android e iOS, atendendo
    aos requisitos identificados para este público-alvo e soluções de acessibilidade encontradas no MSL\@.
    Uma versão inicial do \emph{app} foi avaliada por 77 pacientes com DV e os resultados desse estudo
    mostraram que 90,1\% dos participantes gostariam de ter o \emph{app}, com 62,3\% avaliando-o com nota máxima.
    Contudo, parte das soluções de acessibilidade foram implementadas após a realização desse estudo, assim,
    35,1\% dos participantes apontaram que o aplicativo precisava de melhorias quanto à acessibilidade. Dessarte,
    trabalhos futuros devem realizar uma nova avaliação do aplicativo com as novas funcionalidades e melhorias
    relativas à acessibilidade.

    \textbf{Palavras-chave}: aplicativo móvel. diabetes mellitus. autocuidado. acessibilidade. deficiência visual.
\end{resumo}